
\documentclass[a4paper, 12pt]{article} 
\usepackage{amsmath, amssymb, color, graphicx, enumitem}
\usepackage{fullpage} %smaller margins
\usepackage{hyperref} % hyperlinks
\usepackage{multicol}
\usepackage[framemethod=tikz]{mdframed} %boxes 

%columns separate and line
\setlength{\columnsep}{1.5cm}
\setlength{\columnseprule}{0.2pt}

%font

%\usepackage[sc]{mathpazo}
%\linespread{1.05}         % Palladio needs more leading (space between lines)
%\usepackage[T1]{fontenc}

%font, libertine
\usepackage{libertine}

%word spacing
\usepackage{microtype}

%all equations get full space
\everymath{\displaystyle}

%useful shortcuts
\def\R{\ensuremath{\mathbb{R}}} %\ensuremath adds math mode, if forgotten
\def\Q{\ensuremath{\mathbb{Q}}}
\def\N{\ensuremath{\mathbb{N}}}
\def\Z{\ensuremath{\mathbb{Z}}}
\def\C{\ensuremath{\mathbb{C}}}

%shorcuts with arguments
\newcommand{\abs}[1]{\left\vert#1\right\vert} %nice absolute values
\newcommand{\bt}[1]{\textbf{#1}} %bold
\newcommand{\eq}[1]{\begin{align*}#1\end{align*}} %aligned equations
\newcommand{\norm}[1]{\left\lVert#1\right\rVert} %vector norm
\newcommand{\notimplies}{% does not imply
  \mathrel{{\ooalign{\hidewidth$\not\phantom{=}$\hidewidth\cr$\implies$}}}}
\renewcommand{\eq}[1]{\begin{align*}#1\end{align*}} %aligned equations

%piecewise function

%\begin{displaymath}
%   f(x) = \left\{
%     \begin{array}{lr}
%       1 & : x \in \mathbb{Q}\\
%       0 & : x \notin \mathbb{Q}
%     \end{array}
%   \right.
%\end{displaymath} 

%colors
\definecolor{javagreen}{rgb}{0.25,0.5,0.35} %dark green color
\definecolor{lightblue}{rgb}{0.149,0.545,0.824} %solarized blue
\definecolor{sred}{rgb}{0.863, 0.196, 0.184} %solarized red

\newcommand{\blue}[1]{{\leavevmode\color{lightblue}{#1}}} %solarized blue 
\newcommand{\green}[1]{{\leavevmode\color{javagreen}{#1}}} %command for green
\newcommand{\red}[1]{{\leavevmode\color{sred}{#1}}} %solarized red
\newcommand{\gray}[1]{{\leavevmode\color[gray]{0.5}{#1}}} %gray text

%environment
\newcommand{\tab}{\phantom{ssss}}

\title{}
\date{}
%==tips====
%part
    %section, sub, sub
%\begin{enumerate}[resume] %continues counting
\begin{document}
\begin{center}
\section*{Measure Theory}
Mark Ibrahim \\
*based on Principles of Real Analysis by Aliprantis and Burkinshaw
\end{center}

\tableofcontents
\begin{multicols}{2}

\section{Preliminaries}

a function $f: A \rightarrow B$ is \bt{continuous}
$\iff$ $f^{-1}$(open set) is an open set.

a bounded sequence $a_n$ has a \bt{$\limsup$} defined as 
$\lim_{N \rightarrow \infty} \sup\{a_N, a_{N+1}, \dots \}$
"largest tail"\\
$a_n$ converges if $\limsup = \liminf$.


A \bt{Hausdorff} topological space (T2 space) is a topological space where 
any two points can be seperated by open sets.

$\max\{a, b\} = \frac{a + b}{2} + \frac{|a - b|}{2}$.

union of countably sets is countable.

\section{Algebras and Measures}
\subsection{Semirings and Sigma-algebras of Sets (section 12)}

\subsubsection{semirings}
a collection $S$ of subsets of a set $X$ is called a 
\bt{semiring} if  
\begin{enumerate}
    \item $\emptyset \in S$, 
    \item $A \cap B \in S$, and 
    \item $A - B = C_1 \cup \dots C_n$ for $C_1, \dots C_n \in S$.
\end{enumerate}

Any countable union in $S$ can be written as a countable \bt{disjoint} union.

e.g., $S = \{[a, b) | a \leq b \in \R\}$ is a semiring, not an algebra. \\
* note $[a, a) = \emptyset$.


\subsubsection{algebras}

a nonempty collection $S$ of subsets of a set $X$ is an \bt{algebra} if 
\begin{enumerate}
    \item $A \cap B \in S$ 
    \item and $A^c \in S$. 
\end{enumerate}

Nice properties of algebras are: 
\begin{itemize}
    \item $\emptyset, X \in S$ 
    \item $S$ is closed under finite unions 
and finite intersections as well as subtraction
\end{itemize}

a \bt{$\sigma$-algebra} is an algebra that is closed under countable unions.

\bt{Borel sets} of a topological space (X, T) 
\footnote{(X, T) is a topological space with a set $X$ and subsets $T$ if
$\emptyset, X \in T$, and $T$ is closed under unions (even uncountable), finite intersections.}
is a $\sigma$-algebra 
generated by the open sets.

\subsection{Measures on Semirings (section 13)}
A function $\mu$ from a semiring $S$ to $[0, \infty]$ is a \bt{measure on $S$} if 
\begin{enumerate}
    \item $\mu(\emptyset) = 0$ 
    \item countably additive: $\mu(\cup_{n=1}^\infty A_n) = \sum_{n=1}^\infty \mu(A_n)$.
\end{enumerate}
* $\cup_{n=1}^\infty A_n$ must be in $S$ and each is disjoint. \\
\tab * don't need to check if $S$ is a $\sigma-$algebra!

\noindent $\bullet$ If $A \subseteq B$, ($A, B \in S$), then $\mu(A) \leq \mu(B)$.



Alternatively, can show $\mu$ is a measure if and only if "squeeze" 
\begin{enumerate}
    \item $\mu(\emptyset) = 0$
    \item $\sum_{i=1}^n \mu(A_i) \leq \mu(A)$ if $\cup_{i=1}^n A_i \subseteq A$ and $A_i$ are disjoint.
    \item $\mu(B) \leq \sum_{n=1}^\infty \mu(B_n)$, "subadditive"  if 
    $B \subseteq \cup_{n=1}^\infty B_n$.
\end{enumerate}

\subsubsection{Examples of Measures on S}
\begin{itemize}
    \item \bt{Counting Measure} $\mu(A) = |A|$ 
    \item \bt{Dirac Measure} Fix $a \in X$, $\mu_a(A) = 0$ if $a \not \in A$, else 1.
    \item \bt{Lebesgue Stieltjes} For $f: \R \rightarrow \R$, increasing, left continuous and
    $S = \{[a, b) | a \leq b \in \R \}$, $\mu([a, b)) = f(b) - f(a).$\\
    - \bt{Lebesgue Measure on $S$}, denoted $\lambda$ is defined by $\lambda([a, b)) = b-a.$
\end{itemize}

\subsection{Outer Measures (section 14)}

an \bt{outer measure} is a function $\bar \mu: P(X) \rightarrow [0, \infty$ such that 
\begin{enumerate}
    \item $\bar \mu(\emptyset) = 0$
    \item if $A \subseteq B$, $\bar \mu(A) \leq \mu(B)$
    \item countably subadditive: $\bar \mu( \cup_{n=1}^\infty A_n)$ 
    $\leq \sum_{n=1}^\infty \mu(A_n)$
\end{enumerate}

*an outer measure is not always a measure!

A subset $E$ of $X$ is \bt{measurable} if for all $A \subseteq X$,
$$\bar \mu(A) = \bar \mu(A \cap E) + \bar \mu(A \cap E^c)$$

A nicer equivalent way to show $E$ is measurable is by considering 
all $A$ in $S$ with $\mu^*(A) < \infty$ and showing 
$$\mu(A) \geq \mu^*(A \cap E) + \mu^*(A \cap E^c)$$

Nice Properites
\begin{itemize}
    \item every $A$ in S is $\mu^*$-measurable
    \item if $\bar \mu(E) = 0$, $E$ is measurable
    \item for $E_i$ measurable and any $A \subseteq X$, 
    $\bar \mu(\cup_{i=1}^n A \cap E_i) = \sum_{i=1}^n \bar \mu(A \cap E_i)$
\end{itemize}

the collection of measurable subsets is denoted by $\Lambda$.
This collection is a $\sigma$-algebra! 

Remarkably, the outer measure $\bar \mu$ restricted to $\Lambda$ is a measure!


\subsection{Outer Measures generated by a measure (section 15)}
The outer measure $\mu^*$ generated by a measure $\mu$ is defined for any subset $A$ of $X$, 
\centerline{$\mu^*(A) =$}
$$\inf\{\sum_{n=1}^\infty \mu(A_n) : A \subseteq \cup_{n=1}^\infty A_n 
\text{ for $A_n \in S$}\}$$

$\mu^*$ is called the Cath\'eodory extension of $\mu$.
By convention $\mu^*(A) = \infty$ if no cover exits in $S$.

On semiring $S$, $\mu* = \mu$.


For $E_n$ measurable, if $E_n \uparrow E$, then $\mu^*(E_n) \uparrow \mu^*(E)$.
* $E_n \uparrow E$ means: \\
\tab \tab 1) $E_1 \subseteq E_2 \subseteq \dots$ \\
\tab \tab 2) $\cup_{n =1}^\infty E_n = E$ \\

* note $E$ must be measurable since it's the union of measurable sets \\

For $B_n$ measurable with $\mu^*(B_n) < \infty$, if $B_n \downarrow B$, then
$\mu^*(B_i) \downarrow \mu^*(B)$. \\

a measure space if \bt{finite} if $\mu^*(X) < \infty$.


For $X$ a \bt{finite measure} space $E$ is measurable, if and only if 
$$\mu^*(E) + \mu^*(E^c) = \mu^*(X)$$

For all $A \subseteq X$, there is a measurable set $E$ such that 
$A \subseteq E$ and $\mu^*(A) = \mu^*(E)$.


\subsubsection{Cantor Set}
Cantor set $C = \cap_{n=1}^\infty c_n$, where \\
$c_1 = [0, 1] - (1/3, 2/3) $\\
$c_2 = c_1 - ((1/9, 2/9) \cup (7/9, 8/9))$ \\

each $c_n$ is closed, because it's a closed set minus open sets.

\begin{itemize}
    \item $C$ has measure 0
    \item $|C| = |\R|$
    \item every point of $C$ is an accumulation point of $C$
\end{itemize}

Vitali set is an example of a \bt{non-measurable} subset of $\R$.

\subsection{Lebesgue Measure (section 18)}

\bt{Outer Lebesgue measure} $\lambda^*$ is defined as 
$\lambda^*(A) = \inf \{ \sum_{i=n}^\infty \lambda(a_n, b_n) : A \subset \cup_{n=1}^\infty (a_n, b_n)\}$ 

* note $\lambda(a, b) = b - a$.\\
* often, we say Lebesgue measure instead of outer Lebesgue measure. 

By result about $E_n \uparrow E$ from section 15, we can show 
$(a, b)$, $[a, b]$, and $(a, b]$ are all measurable with same measure.

$E \subseteq \R$ is \bt{Lebesgue measurable} $\iff$ there is open $O \subseteq \R$ for each $\epsilon$ such that 
$E \subseteq O$ and $\lambda(O - E) < \epsilon$.

Every Borel set in $\R$ is $\lambda-$measurable

\subsubsection{What are the Borel sets in the reals?}
By definition, it's the $\sigma-$algebra generated by open sets in $\R$.
(Borel $\sigma-$algebra is generated by intervals of the form $(-\infty, a]$, for 
$a \in \Q$). \\
Borel sets contain:
\begin{itemize}
    \item all closed sets 
    \item union of all open sets or closed sets
    \item intersection of all open/closed sets
\end{itemize}

* we can write any open set in $\R$ as disjoint countable union of open intervals!
\subsubsection{Regular Borel Measure}
\label{Regular Borel Measure}
For $X$, a Hausdorff topological space and $B$ the borel sets in $X$, 
a measure $\mu$ on $B$ is called a \bt{regular borel measure} if 
\begin{enumerate}
    \item $\mu(K) < \infty$ if $K$ is compact 
    \item for $B$ a borel set, $\mu(B) =$\\ $\inf \{\mu(O) | O \text{ is open } B \subseteq O\}$ 
    \item for $O$ open, $\mu(O) = \\$  $\sup\{\mu(K) | K \text{ is compact and }$ $K \subseteq O\}$
\end{enumerate}

\begin{enumerate}
    \item $\lambda$ is a regular borel measure
    \item Dirac measure is a regular borel measures
    \item Counting measure is not\\
    \gray{for example $[0, 1]$ is compact, but has infinite measure}
    \item any \bt{translation invariant} regular borel measure on $\R$ is 
    $c \lambda$ for some $c \in \R^+$
\end{enumerate}

\section{Integration: functions}

\subsection{Measurable Functions (section 16)}

a relation holds \bt{almost everywhere} if set where it fails has measure 0.

$f: X \rightarrow \R$ is a \bt{measurable function} if 
\begin{itemize}
    \item $f^{-1}(O)$ is measurable, for all open sets $O$
    \item $f^{-1}(a, \infty)$ is measurable, for all a in $\R$
\end{itemize}

If $f, g: X \rightarrow \R$,
\bt{$f = g$ almost everywhere}
and $f$ is measurable, then
$g$ is measurable too!\\
"= a.e. means measurability carries over"

If $f, g: X \rightarrow \R$ are \bt{measurable} then $\{x \in X | f(x) > g(x)\}$
is measurable.

Sum, product, constant multiple, $||$, $\max$, and $f^+$ 
\footnote{$f^+ = f(x)$ if $f(x) \geq 0$ or $0$ otherwise.
}
of 
measurable functions is also measurable!

\subsubsection{Sequences of Functions and Measurability}
recall (from analysis): $f_n \rightarrow f$ \bt{uniformly} means $|f_n(x) - f(x)| < \epsilon$ for all $x$ if you go out far enough in the sequence.



\bt{Key Theorem}: If $f_n \rightarrow f$ \bt{uniformly} and $f_n$ are continuous, 
    then $f$ is continous.

We can define $\limsup$ ($\liminf$) for any \bt{bounded} sequence.

For a sequence of measurable functions $\{f_n\}_{n=1}^\infty$
\begin{itemize}
    \item If $f_n \rightarrow f$ a.e., then $f$ is measurable func.
    \item If $\{f_n\}_{n=1}^\infty$ is bounded, then $\limsup$ is a measurable function (so is $\liminf$)
\end{itemize}

A sequence of functions, $\{f_n\}_{n=1}^\infty$ 
($f_n: X \rightarrow \R$) converges 
\bt{almost uniformly} on $X$ if 
for any $\epsilon$, there exists a measurable set $F$ where 
$\mu(F) < \epsilon$ and $\{f_n\} \rightarrow f$ 
\bt{uniformly} on $X-F$.


If $f_n \rightarrow f$ \bt{almost uniformly} on $X$ and $\mu(X) < \infty$
then, $\abs{f_n(x) - f(x)} < \epsilon$ for all $n > $ some $N \in \N$, 
and all $x$ in a set $J$ where $\mu(J^c) < \delta$.


\subsubsection{Ergov's Theorem (16.7)}
If $f_n \rightarrow f$ \bt{almost uniformly} on $X$, 
then $f_n \rightarrow f$ pointwise \bt{almost everywhere} on $X$.

Also, if $\mu(X) < \infty$ and $f_n \rightarrow f$ pointwise on $X$,
then $f_n \rightarrow f$ uniformly on $X$.

\gray{counter example: if $\mu(X)$ is not finite,
consider $X = \R$, $\mu=\lambda$ and $f_n = \chi_{[n, n+1)}$.\\
Then, $f_n \rightarrow 0$, but not almost uniformly
}


\subsection{Simple and step functions (section 17)}

nice properties of $\chi_A$
\begin{itemize}
    \item $A \subseteq B \iff \chi_A \leq \chi_B$
    \item $\chi_{A \cap B} = \chi_A \chi_B$ (equivalently $\min\{\chi_A, \chi_B\}$)
    \item $\chi_{A \cap B} = \chi_A + \chi_B - \chi_{A \cap B} (= \max\{\chi_A, \chi_B\})$
\end{itemize}

    a measurable function $f: X \rightarrow \R$ is a \bt{simple function} if it takes on finitely many values.\\
    the standard representation of a simple function is 
    $$\sum_i^n a_i \chi_{Ai} $$
     where $a$ is are distinct nonzero outputs and $A$ inputs

     If each $A_i$ has finite measure, then $f$ is called a \bt{step function}.

     The \bt{integral} of a step function $\phi$ is 
     \eq{\int \phi du = \sum_i^n a_i \mu^*(A_i)}

     *it turns out any representation, even when $A_i$ are not disjoint (or $a_i$ distinct)
     yield the same integral value

     addition and scalar multiplication can be split over integrals as expected. 

     If $\phi \geq \psi$ a.e., then $\int \phi \geq \int \psi$ \\
     *holds if $\psi = 0$ or $\geq$ is $=$

     If $\phi_n$ is a \bt{sequence of step functions} with 
     $\phi_n \downarrow 0$ a.e., then $\int \phi_n \downarrow 0$.\\
    \gray{(similarly if $\phi_n \uparrow \psi$ a.e.)}

     \noindent *careful, $\uparrow \psi$, but $\downarrow 0$ \\
     * also $\phi_n \rightarrow \phi$ isn't good enough!

    If $\phi_n \uparrow f$ a.e. and $\psi_n \uparrow f$ a.e., then 
    $$\int \phi_n = \int \psi_n \text{ as } n \rightarrow \infty$$

    We can show $A$ is \bt{measurable} if we can find 
    step functions $\phi_n \uparrow \chi_A$. 
    In this case, $\mu^*(A) = \lim \int \phi_n$

    For any measurable $f \geq 0$, there exists \bt{simple} $\psi_n$  such that 
    $$ 0 \geq \psi_n \uparrow f$$

    \subsubsection{sigma-finite}
    $X$ is a \bt{$\sigma-$finite measure space} if there exists 
    $E_i$ such that $\cup_{i=1}^\infty E_i = X$, $\mu(E_i) < \infty$, 
    and $E_1 \subseteq E_2 \subseteq \dots$.

    Who cares? Well if, $X$ is $\sigma-$finite then 
    for a \bt{measurable} $f \geq 0$ a.e., then there exists 
    \bt{step} $\phi_n \uparrow f$ a.e.

\section{Lebesgue Integral}


\subsection{Upper Functions (section 21)}

$f: X \rightarrow \R$  is an \bt{upper function} if there exist step $\phi_n$ such that  
\begin{itemize}
    \item $\phi_n \uparrow f$ a.e.
    \item $\lim \int \phi_n du < \infty$
\end{itemize}

$\phi_n$ is called a \bt{generating sequence} for $f$.

* all step functions are upper functions\\
* $f$ upper does \bt{not} imply $-f$ is upper

The integral of $f$ an \bt{upper function} is defined as 
$$\int f du = \lim \int \phi_n du$$
* the value is independent of our choice of $\phi_n$ because if any other 
$\psi_n \uparrow f$ too, then $\int \phi_n = \int \psi_n$ as $n \rightarrow \infty$


\bt{sums, scalar multiples, maxes} of upper functions are upper functions.


If $f \geq g$ a.e. (both upper) then $\int f \geq g$\\
(same for g = 0)

If a \bt{sequence of upper} functions $f_n \uparrow f$ a.e. 
and $\lim \int f_n < \infty$ then $f$ is upper 
and $\int f = \lim \int f_n$
(similarly if $f_n \downarrow 0$)

\subsection{Integrable Functions (section 22)}


a function $f$ is \bt{integrable} if $f = u - v$, both upper functions.

We define $\int f$ as $\int u - \int v$\\
* well-defined no matter the representation of $f$

\subsubsection{How does integrable relate to other properties?}
\begin{itemize}
    \item \bt{upper} functions are integrable
    \item \bt{step} functions are integrable (b/c step are upper)
    \item integrable implies \bt{measurable}
        \begin{itemize}
            \item measurable does \bt{not} imply integrable\\
            \gray{e.g., constant functions are measurable, but only integrable
            when $\mu(X) < \infty$.}
        \end{itemize}
\end{itemize}

Canoncial way to write integrable 
$$f = f^+ - f^-$$
\gray{b/c: both $f^+$ and $f^-$ are upper if $f$ is integrable}

\subsubsection{When is f integrable?}

If integrable $f$ = $g$ a.e., then $g$ is integrable (and integrals are equal).

sums, scalar multiples, max, $| |$ of integrable are integrable.

* $|f|$ integrable does \bt{not} imply $f$ is integrable.

If $f$ is measurable and $h \leq f \leq g$ a.e. for $h, g$ integrable, then 
$f$ is \bt{integrable}.\\
\gray{``measurable sandwiched between integrable is integrable"}

nice properties of $f$ integrable:
\begin{itemize}
    \item if $f \geq 0$ a.e. then $f$ is \bt{upper}
    \item $A = \{x | |f(x)| \geq \epsilon\}$ has \bt{finite measure} ($A$ is also measurable) \\
               \gray{b/c: $|f|$ is measurable so $|f|^{-1}(\epsilon, \infty)$}
\end{itemize}

For $f, g$ integrable, 

    \begin{enumerate}
        \item $\int |f| = 0 \iff f = 0$ a.e.
        \item If $f \geq g$ a.e., then $\int f \geq \int g$
        \item $\int |f| \geq \abs{\int f}$
    \end{enumerate}

If $E$ is \bt{measurable}, $f$ is \bt{integrable}, then 
$$\int_X f = \int_E f + \int_{X-E} f$$

\subsubsection{Big: Levi, Fatou, and Lebesgue Dominated Convergence}

\centerline{\bt{Levi's Theorem}}
For $f_n$ a sequence of \bt{integrable} functions such that 
$f_n \leq f_{n+1}$ a.e. for all $n$ and $\lim \int f_n < \infty$, then 
there exists $f$ integrable such that $f_n \uparrow f$ a.e.\\
(and $\lim \int f_n = \int f$)\\
\gray{``an integrable function waits at the top of an increasing sequence''}

* $f$ is defined a.e. on $X$ \\

nice consequence: 
If integrable $f_n > 0$ a.e., with $\sum_{n=1}^\infty \int f_n < \infty$, then 
$\sum f_n$ defined an integrable function and 
$$\int \sum_{n=1}^\infty f_n = \sum_{n=1}^\infty \int f_n$$

*not true in Riemann land!  \\
* trick: when $f_1 \leq f_2 \leq \dots$ can make a positive sequence by considering
$f_1 - f_1, f_2 - f_1, \dots$



\centerline{\bt{Fatou's Lemma}}
For integrable $f_n \geq 0$ a.e. for all n and 
$\lim \inf \int f_n < \infty$, then 
$$\int \lim \inf f_n \leq \lim \inf \int f_n$$
where $\lim \inf f_n$ defines an integrable function a.e. on $X$. \\
\gray{``$\lim \inf$ of integrable is integrable and less than integral of parts"}

\centerline{\bt{Lebesgue Dominated Convergence}}
\begin{enumerate}
    \item \bt{measurable} $f_n \rightarrow f$ a.e.
    \item $|f_n| \leq g$ a.e. for $g$ \bt{integrable}
\end{enumerate}
then 
$$\int f = \lim_{n \rightarrow \infty} \int f_n$$
where $f_n$ and $f$ are integrable (for all $n$)\\
\gray{``interchange $\lim$ and $\int$ for measurable functions bounded by an integrable function"}





\subsection{Riemann Integrals (section 23)}

a partitions $P$ is just a collection of points inside an interval. \\
A second partition $Q$ \bt{refines} $P$ if $P \subseteq Q$.

The \bt{Upper} Riemann sum is 
\eq{U(f, P) = \sum_{i}^n M_i (x_i - x_{i-1})}

where $M_i$ is the sup of $f$ on $[x_{i-1}, x_i]$.\\
\gray{(similarly lower sum is defined with $m_i$, the inf on the interval)}

If $Q$ refines $P$, then $U(f, Q) \leq U(f, P)$. \\
\gray{($L(f, Q) \geq L(f, P)$).}

$f$ is \bt{Riemann integrable} if 
\eq{\lim_{||P_i|| \rightarrow 0} U(f, P_i) 
& = \lim_{||P_i|| \rightarrow 0} L(f, P_i)  \\
& = \int f(x) dx
}
where $||P_i||$ the length of the largest subinterval.


* The Riemann integral of $f$ can also be defined when $\sup L = \inf S$; this value 
is said to be the integral of $f$.

* \bt{Riemann's Critereon} $f$ is integrable if if $L$ and $U$ can be made arbitrarily close
by selecting a sufficiently fine partition.

Every \bt{Riemann} integrable function is \bt{Lebesgue} integrable.

A \bt{bounded function} $f: [a, b] \rightarrow \R$ is \bt{Riemann} integrable 
$$\iff$$
$f$ is \bt{continuous} a.e.

\subsection{Product Measures and Iterated Integrals (section 26---only a sketch)}

If $S$ and $T$ are semirings, then their cross-product: $S \times T$ is also a semiring.\\
(similarly, for measures $\mu$ and $v$)

a function $f: X \times Y$ is $\mu \times v$ integrable by computing cross-sections: 
\eq{
\int_{X \times Y} f d (\mu \times v) = \int_X \int_Y f d \mu d v
}
"Fubini" says the order of $\int_X \int_Y$ doesn't matter!

\bt{Tonelli's Theorem} f is $\mu \times v$-measurable and $\int_X \int_Y |f| d v d \mu$ exists (or other order) then $\int \int f$ exists.

\section{Function Spaces (Chapter 5)}

\subsection{norms on vector spaces (section 27)}
A real valued function $||\ ||$ on a vector space $V$ is a \bt{norm} if \\
for all v in V, 
\begin{enumerate}
    \item $||v|| > 0$ and $||v|| = 0 \iff v =0$
    \item $||\alpha v|| = |\alpha| ||v||,$ for all $\alpha \in \R$
    \item $||v + w || \leq ||v|| + ||w||$ (triangle)
\end{enumerate}


* a norm space $\implies$ a metric space (but not the converse)

Can show $\abs{||v|| - ||w|| \leq ||v - w||}$.
\gray{by triangle}

Examples of norms in different spaces: 

\begin{itemize}
    \item "Euclidean norm": $\sqrt{v_1^2 + v_2^2 + \dots}$ 
    \item "sup norm": $||f||_{sup} = sup |f(x)|$ over all x. \\
        (only valid in space of bounded, real-valued functions) 
    \item "$L^p$" norm: $||f||_p = \left(\int |f|^p \right )^{1/p}$ \\
        only valid in $L^P(X)$ space = $\{ f | f$ is measurable and $|f|^p$ is integrable$\}$
        * note on $\R^n$, the $L^p$ norm is: 
        $||(a_1, \dots, a_n)||_p = (|a_1|^p + \dots + |a_n|^p)^{1/p}$
\end{itemize}

A \bt{bounded normed} space is one where $||v|| \leq M$ for some constant $M$.

A normed space (a vector space with a norm) is a \bt{Banach} space if every Cauchy sequence
converges (aka \bt{complete}).

Two norms are equivalent if there are $K, M > 0$: 
$K ||x||_1 \leq ||x||_2 \leq M ||x||_1$ for all x.

*In a finite dimensional vector space, all norms are equivalent

\subsection{Linear Operators (section 28)}

A \bt{Linear Operator} (or transformation) is a map T between two 
vector spaces V and W such that: 
$$T(aV + bW) = aT(V) + bT(W)$$

The \bt{Operator norm} of T, $||T||$ is 
$$\sup\{ ||T(v) ||: ||v|| = 1\}$$

we say $T$ is bounded if $||T||$ is finite.

What's is equivalent to $T$ being \bt{bounded}?
\begin{enumerate}
    \item $||T(v)|| \leq M ||v||$ for all $v$ ($M \geq 0$)
    \item T is continuous at zero
    \item T is continous
    \item ker(T) is a closed set
\end{enumerate}

The \bt{collection} of \bt{bounded} linear operators from $V$ to $W$ is 
denoted $L(V, W)$ \\

Nice Properties:
\begin{itemize}
    \item $L(V, W)$ is a normed linear space using $|| ||_{operator}$
    \item If dim V $< \infty$ then every operator is a bounded operator. 
    \item If W is complete (w.r.s to $|| ||_w$), then $L(V, W)$ is complete
\end{itemize}

$L(V, \R)$ is always a \bt{complete normed} linear space ("Banach space").

\subsubsection{Dual Spaces}

For any normed linear space V, the \bt{dual space} of V is 
$L(V, \R) = V^*$\\
\gray{"bounded linear transformations from V to $\R$"}

For any $1/p + 1/q = 1$ ($p, q > 1$), $(L^p)^* = L^q$. \\
\gray{"dual of $L^p$ is equivalent to $L^q$"}

* we get $L^2$ is self-dual

Also note the double dual $((L^p)^*)^* \xrightarrow L^p$, isometrically 
isomorphic.


\subsection{Lp Spaces (section 31)}

$L^p$ is the collection functions $f$ such that \\
1) $f$ is measurable\\
2) $|f|^p$ is integrable

This collection, $L^p$ forms a space. \\
We can define a norm on $L^p$ $|| ||_P$ by 
$$|| f||_p = (\int |f|^p)^{1/p}$$

Proof of triangle inequality is called \bt{Minkowski's Inequality}
only holds for finite $p > 1$, 
$$||f + g ||_p \leq ||f||_p + ||g||_p$$
for $f, g \in L^p$.

\bt{Holder's Inequality} says if $1/p + 1/q = 1$ (called "conjugate exponents")
and $f \in L^p, g \in L^q$ then 
$$\int |fg| \leq ||f||_p ||g||_q$$
\gray{*implies $fg \in L^1$}

\bt{Risz-Fischer} $L^p$ is complete (every Cauchy seq converges) for all $p \geq 1$ (with respect to $L^p-$norm)

*Careful: $|| f - f_n ||_p \rightarrow 0$ doesn't imply $f_n \rightarrow f$ \\
\gray{we don't get pointwise convergence}

* we do get that some subsequence converges pointwise \\
\gray{$\lim || f - f_n||_p = 0 \notimplies f_n \rightarrow f$}

However we do get: \\
If $\lim || f - f_n ||_p = 0$ (in Lp, $p \geq 1$), then 
there exists a subsequence of $f_n$, $f_{k} \rightarrow f$ a.e.

* cauchy $\iff$ some subsequence converges pointwise


\gray{e.g., $X = [0, 1]$ and $\mu = \lambda$,  \\
$A_1 = [0, 1]$ \\
$A_2, = [0, 1/2], [1/2, 1]$ \\
$A_3 = [0, 1/3], [1/3, 2/3], [2/3, 1]$ \\
$\dots$ \\
Let $f_1 = \chi_{A_1}, f_2 = \chi_{A_2} \dots$ \\
Then for all $n$ and any $p \geq 1$, 
$$f_n^p = f_n$$
but $|| f_n ||^p \rightarrow 0$. \\
(there is a subsequence of $f_n$ that converges pointwise to 0 a.e.)
}

another example is \\
\gray{$X = \R$, $f_n = \chi_{[n, n+1]}$, $f_n^p = f_n \rightarrow 0 = f$, \\
but \\
$f_n \not \rightarrow f$ in Lp, because $||f _n ||_p =1$ for any $n$.
}


If $f_n \rightarrow f$ a.e. and $\lim || f_n ||_p = || f ||_p$, then 
$\lim|| f - f_n||_p = 0$
\gray{"pointwise and norms approaching implies convergence in Lp"}


\subsubsection{Essentially Bounded Functions}

If $|fg| \leq h$, some integrable function, then $fg \in L^1$.

a function $f$ is \bt{essentially bounded} if 
$$| f(x) | \leq M$$ 
for almost all $x$.

The \bt{essential supremum}, denoed by $||f||_\infty$ is 
$$||f||_\infty = \inf{M |\ |f(x)| \leq M \text{for almost all x}}$$

e.g., all constant functions are in $L^\infty$.

Also note:  \\
* $L^\infty \not \subseteq L^p$ for any $p < \infty$ when $\mu(X) = \infty$ \\
* $L^\infty \subseteq L^p$ for any p if $\mu(X) < \infty$. \\

If $X$ is $\sigma-$finite and $F \in (L^1)^*$ then there exists 
$g \in L^\infty$ such that for any $f \in L^1$
$$F(f) = \int fg$$ 


\subsubsection{What is in Lp?}

$C_b(K)$ is the set of continous and bounded functions on $K$. 

\gray{e.g., If K is compact, then every continuous function is bounded 
(Extreme Value Theorem).}\\
*can show $C_b(K)$ for K compact is a complete n.l.s using $|| ||_{sup}$.

Any \bt{continuous} function with \bt{compact support} is in $L^p(\R)$ 
($p \geq 1$)

*a function $f: \R \rightarrow \R$ has \bt{compact support} if 
the closure of $f$ ($\{x | f(x) \neq 0\}$) is compact.

*recall: uniform limit of continuous functions is continuous

On $[1, \infty)$,
$$1/x^a \in L^p \iff ap > 1$$

On $(0, 1]$
$$1/x^a \in L^p \iff ap < 1$$

In a finite measure space, $L^q \subseteq L^p$ if $1 \leq p \leq q$.

\subsubsection{Dense Functions in Lp}

$A$ is \bt{dense} if $B$ if for any $b \in B$, there is an $a$ such that $d(a, b) < \epsilon$

nice fact: For $A \subseteq B \subseteq M$, if $A$ is dense in $B$ and $B$ is dense in $M$,
then $A$ is dense in $M$.

* The collection of step functions is \bt{dense} in $L^p$ (for $1 < p < \infty$).

Also linear combinations of $\chi_{(a, b)}$ are dense in $L^p$ ($p \geq 1$).

For $\mu$ a \bt{regular Borel measure} on a Hausdorff locally compact topological space $X$, 
the collection of continuous functions with compact support is \bt{dense} in $L^p$ 
(for $1 \leq p < \infty$).\\

* Hausdorff space is one where two points can be separated by open sets. \\
* locally compact means every point lives in a compact neighborhood.\\
* remember a regular borel measure has additional requirements on compact and borel sets 
see: \ref{Regular Borel Measure}. \\

notation: $C_c(X)$ is the set of continuous real-valued functions on $X$ with 
compact support


What's more? The collection of Riemann integrable functions on $[a, b]$ is
dense in $L^1([a, b])$


\subsubsection{Counting Measure and lp}
$l^p = L(\N,$counting)

For a sequence $f_n \in l^p$, $f \in l^p$\\
$$\iff$$ 
\centerline{$\int |f|^p$ converges}\\
\gray{meaning $\sum |f_n(\N)|^p$ converges.}


For $p \leq q$, if $f \in l^p$ then $f \in l^q$.
\gray{why? $f \in l^p \implies \sum |a_n|^p$ converges, so $|a_n|^p < 1$ \\
hence $|a_n|^q \leq |a_n|^p$}



\section{Hilbert Spaces (Chapter 6)}

\subsection{Inner Product Spaces (section 32)}
an \bt{inner product} on a vector space $V$ is a function from  $V \times  V$ to $\R$ such that
\begin{itemize}
    \item "linear" $(ax + by, z) = a(x, z) + b(y, z)$ (for $x,y, \in V$ and $a, b \in \R$)

    \item "symmetric" $(x, y) = (y, x)$
    \item "positive definite" $(x, x) \geq 0$ and $(x, x) = 0 \iff x = 0$.
\end{itemize}

In an \bt{inner product space} (space with an inner product), the \bt{norm} 
(induced by the inner product) of a vector $v \in V$ is 
$$|| v || = \sqrt{(v, v)}$$

Two vectors $x, y$ are \bt{orthogonal} ($x \bot y$) if $(x, y) = 0$.
In a real vector space, we also have a notion of angles between vectors: 
$$\cos(\theta) = \frac{(x, y)}{|| x || || y ||}$$

\subsubsection{Useful Inequalities}

\begin{itemize}
    \item Cauchy-Schwarz: $|(x, y)| \leq || x || || y ||$
    \item Parallelogram Law: $|| x + y ||^2 + || x - y||^2 = 2 || x ||^2 + 2 || y ||^2$
    \begin{itemize}
        \item a norm $|| * ||$ is induced by an inner product $\iff$ Parallelogram holds
        \item in a real vector space we then have: 
        $(x, y) = 1/4 || x + y ||^2 - 1/4 || x - y ||^2$
    \end{itemize}
    \item Bessel's Inequality: If $\{x_i\}$ is a collection of orthonormal vectors in an 
    inner product space, $\sum_i |(x, x_i)|^2 \leq || x ||^2$.
    \item Pythagorean Theorem: for $x_1, x_2, \dots, x_n$ pairwise orthogonal, 
    $||x_1 + \dots + x_n||^2 = ||x_1||^2 + \dots ||x_n||^2$
\end{itemize}


\subsection{Hilbert Spaces (section 33)}

A \bt{Hilbert space} is a complete \bt{inner product space}.  \\
* the norm is induced by the inner product (i.e. $|| x || = \sqrt{(x, x)}$)

Examples: 
\begin{itemize}
    \item $l^2$ under the inner product: $(x, y) = $ "dot product"
    \item $L^2(u)$ is another example
\end{itemize}

\subsubsection{Orthogonal Complement and Spans}
The \bt{orthogonal complement} of a subset $A$  of an inner product space $X$ is 
$$A^\bot = \{x \in X: x \bot y \text{ for all } y \in A\}$$
\gray{"set of vector orthogonal to all vectors in A"}

*$A^\bot$ is a vector subspace 

*careful: $(A^\bot)^\bot \subseteq A$, but not necessarily equal to $A$

When $X$ is a \bt{Hilbert space} and $A$ is a closed subspace, 
then $A$ and $A^\bot$ span the entire Hilbert space.\\
\gray{in this case, we have $(A^\bot)^\bot = A$}

In a Hilbert space, a subspace of $M$ is \bt{dense} $\iff$
only zero vector is orthogonal to $M$.

The elements of a subset of a Hilbert space are \bt{linearly independent} if $0$ is not in the set and the 
elements are pairwise $\bot$.

\subsection{Orthonormal Bases}

In a Hilbert space, for a family of orthonormal vectors $\{e_i\}$ 
the following are equivalent
\begin{itemize}
    \item $0$ is not in $\{e_i\}$ and $\{e_i\}$ spans a dense subset of $H$
    \item If $x \bot e_i$ for each $i$, then $x = 0$
    \item Parseval's Identity: for each vector $x$, $|| x ||^2 = \sum_i |(x, e_i)|^2$
    \item for each vector, $(x, e_i) \neq 0$ at most countably many times and 
    $x = \sum_i (x, e_i)e_i$ converges
\end{itemize}

For an \bt{orthonormal basis} $\{e_i\}$ in a Hilbert space, 
the family of scalars $\{(x, e_i)\}$ are the \bt{Fourier coefficients}.

*Fourier coefficients are always with respect to a basis



Every Hilbert space H is \bt{linearly isometric} (a linear, norm preserving map exists)
to a Hilbert space of the form $l_2(\Q)$.

Specifically, $L: H \rightarrow l_2(I)$ defined by $L(x) = \{(x, e_i)\}_i$ is such a map.

An infinite dimensional Hilbert space is \bt{separable} $\iff$ it's \bt{linearly isometric} to $l^2$.

In an infinite dimensional Hilbert space $H$,  \\
\centerline{H is \bt{separable}} \\
\centerline{$\iff$}  \\
\centerline{it has a countable orthonormal basis}

*in this case, every orthonormal basis is countable \\
\gray{"if one then all"}

\subsection{Fourier Analysis}

\subsubsection{Best Approximation Theorem}

Let $e_1, \dots$ be an orthonormal set 
\gray{(not necessarily a basis)} \\
Define $H_N = span \{e_1, \dots$. 

The map $\pi_N: H \rightarrow H_N$ defined by 
$$\pi_N(v) = \sum_{i} (v, e_i) e_i$$
is the \bt{orthogonal projection of v into $H_N$}.\\
\gray{it's also a linear transformation and $(v - \pi_N(v)) \bot H_N$.}

"Best Approximation Theorem": If $v \in H, w \in H_N$, then 
$$|| v - w || \geq || v - \pi_N(v) ||$$
\gray{"$\pi_N(v)$ is the unique closest point to v in $H_N$"}

If $H_1 \subseteq H_2 \subseteq \dots H$ (finite dimensional) and $v \in H$, 
then 
$$|| v - \pi_1(v) || \geq || v - \pi_2(v) || \geq \dots$$
and 
$\pi_{N+1}(v) = \pi_N(v) + (v, e_{N+1})e_{N+1}$.

Stone-Weirstrauss \\
\gray{tells you about the collection continuous functions and 
density of vector subspace}

Applying Stone-Weirstrauss to $L^2([0, 2 \phi])$ we get: \\
for any continous $f: [0, 2\pi] \rightarrow \R$, there is a polynomial $p(x)$ such that
$$|f(x) - p(x)| < \epsilon \text{ for any x}$$

\subsubsection{Fourier coefficients}

Let $e_1, \dots$ be an orthonormal set in a separable Hilbert space $H$.

Define the ith Fourier coefficient as 
$$\hat{e}_i (v) = (v, e_i)$$

\gray{* we can think about Best approx, Bessel, 
and Parseval's in terms of $\hat{e}_i$}


If $e_1, \dots$ is an orthonormal basis then for any $v$ there exists 
$\alpha_1, \dots, \alpha_N$ in $\R$ such that 
$$|| v - \sum_{i=1}^N \alpha_i e_i || < \epsilon.$$

Using the above with Best approximation Theorem we get 
$$\lim_{N \rightarrow \infty} \pi_n(v) = v$$

(implying $\lim_{N \rightarrow \infty} || \pi_N(v) ||^2 = || v ||^2$)
\gray{*in H-norm}

\subsubsection{Riesz-Fischer}
Riesz-Fischer: Let $e_1, \dots$ be an orthonormal basis of $H$, then 
the map 
$$v \rightarrow \{\hat{e}_i (v)\}_i^\infty$$
is an \bt{isometric isomorphism} of $H$ with $l^2$.

*isometric: is norm preserving

* up to isometric isomorphism, $l^2$ is the only separable countably finite
dimensional Hilbert space.


\subsubsection{Fourier on L2}

For $H = L^2(I)$ (I is an interval $[a, b]$), 
$$\int_I | f |^2 = \sum_{i=1}^\infty \abs{\int f e_i} $$
for all $f \in L^2(I)$.\\
\gray{* note: $\int f e_i = (f, e_i)$ in $L^2$}

Every $f \in L^2(I)$ has a Fourier series that converges to $f$ 
(not pointwise!)

\bt{Classical Fourier} basis for $L^2$ are for $n = 1, 2, \dots$: \\
On $[0, 1]$: $1, \sqrt{2}\sin(2 \pi nx), \sqrt{2}\cos(2\pi nx)$ \\
On $[0, 2\pi]$ or $[-\pi, \pi]$: $\frac{1}{\sqrt{2\pi}}$, 
$\frac{\sin(xn)}{\sqrt{\pi}}$, $\frac{\cos(nx)}{\sqrt{\pi}}$

For pointwise convergence, we restrict our attention to nicer functions:\\
If $f, f' \in C([0, 1])$ and $f(0) = f(1)$, then the classical Fourier series
\bt{converges absolutely and uniformly} to $f$.

* $C^1$ functions are dense in $L^p(I)$ (for example $L^2([0, 1])$)


Jordan's Theorem for $f$ \bt{sectionally continuous} on $[0, 1]$, then 
Fourier series converges to 
$$\frac{1}{2}f(x_0^+) + \frac{1}{2} f(x_0^-)$$
at $x_0$ if left/right hand derivatives exist. \\
Specifically, Fourier series converges to $f(x_0)$ if $f$ is continuous 
at $x_0$.


Riemann-Lebesgue Lemma: If $f(x) \in L^p([a, b])$ then 
$$\lim_{m \rightarrow \infty} \int_a^b f(t) \sin(mt) = 0$$
(same for $\cos$).

*this implies for any $f \in L^p$, the Fourier coefficients $\rightarrow 0$, 
as $m \rightarrow \infty$.

\subsection{Risz-Representation Theorem}

$f$ is a \bt{positive} function if $f(x) \geq 0$ for all $x$.

A continuous \bt{linear functional} $F$ (map from $C_c(X) \rightarrow \R$)
is \bt{positive} if $F(f) \geq 0$ for all positive $f$

*note $F$ is in the dual space of $C_c(X)$

\bt{Risz-Representation} Theorem: If $F$ is any positive linear functional 
on $C_c(X)$ then there exists a regular Borel measure $\mu$ such that 
$F_u(f) = \int f = F(f)$. \\
*\gray{can show map is linear: $F(af + bg) = aF(f) + bF(g)$.}


\section{Miscellaneous Examples}

\begin{itemize}
    \item In $X = \R, \mu= \lambda$, $f(x) = 1$ isn't integrable, but 
    if $f(x) = 1$ on $[a, b]$ and zero elsewhere it is! 
    \item $L^p(I)$ is a \bt{separable} metric space
    \item $\sum_{n=1}^\infty \frac{1}{n^2} = \frac{\pi^2}{6}$ \\
    \gray{why? compute fourier for $f(x) = x$ on $[-\pi, \pi]$}
    \item For $l^2 = L^2(\N)$ with counting measure, 
    $\{\chi_{\{i\}}\}_{i \in \N}$ is an orthonormal basis of $l^2$.
    \item $\sum \frac{a_n}{n}$ (fourier coefficients) converge absolutely 
    for $f \in L^2([0, 1])$.
\end{itemize}
\section{Questions}
\begin{enumerate}
    \item
\end{enumerate}



\end{multicols}

\end{document}


