
\documentclass[a4paper, 12pt]{article} 
\usepackage{amsmath, amssymb, color, graphicx, enumitem}
\usepackage{fullpage} %smaller margins
\usepackage{hyperref} % hyperlinks
\usepackage{multicol}

%columns separate and line
\setlength{\columnsep}{1.5cm}
\setlength{\columnseprule}{0.2pt}

%font

%\usepackage[sc]{mathpazo}
%\linespread{1.05}         % Palladio needs more leading (space between lines)
%\usepackage[T1]{fontenc}

%font, libertine
\usepackage{libertine}

%word spacing
\usepackage{microtype}

%all equations get full space
\everymath{\displaystyle}

%useful shortcuts
\def\R{\ensuremath{\mathbb{R}}} %\ensuremath adds math mode, if forgotten
\def\Q{\ensuremath{\mathbb{Q}}}
\def\N{\ensuremath{\mathbb{N}}}
\def\Z{\ensuremath{\mathbb{Z}}}
\def\C{\ensuremath{\mathbb{C}}}

%shorcuts with arguments
\newcommand{\abs}[1]{\left\vert#1\right\vert} %nice absolute values
\newcommand{\bt}[1]{\textbf{#1}} %bold
\newcommand{\eq}[1]{\begin{align*}#1\end{align*}} %aligned equations
\newcommand{\cb}[1]{\centerline{\fbox{#1}}} %centered box
\newcommand{\bp}[1]{\fbox{\parbox{0.8\textwidth}{#1}}} %box paragraph
\newcommand{\norm}[1]{\left\lVert#1\right\rVert} %vector norm
\newcommand{\notimplies}{% does not imply
  \mathrel{{\ooalign{\hidewidth$\not\phantom{=}$\hidewidth\cr$\implies$}}}}
\renewcommand{\eq}[1]{\begin{align*}#1\end{align*}} %aligned equations

%piecewise function

%\begin{displaymath}
%   f(x) = \left\{
%     \begin{array}{lr}
%       1 & : x \in \mathbb{Q}\\
%       0 & : x \notin \mathbb{Q}
%     \end{array}
%   \right.
%\end{displaymath} 

%colors
\definecolor{javagreen}{rgb}{0.25,0.5,0.35} %dark green color
\definecolor{lightblue}{rgb}{0.149,0.545,0.824} %solarized blue
\definecolor{sred}{rgb}{0.863, 0.196, 0.184} %solarized red

\newcommand{\blue}[1]{{\leavevmode\color{lightblue}{#1}}} %solarized blue 
\newcommand{\green}[1]{{\leavevmode\color{javagreen}{#1}}} %command for green
\newcommand{\red}[1]{{\leavevmode\color{sred}{#1}}} %solarized red
\newcommand{\gray}[1]{{\leavevmode\color[gray]{0.5}{#1}}} %gray text

%environment
\newcommand{\tab}{\phantom{ssss}}

\title{}
\date{}
%==tips====
%part
    %section, sub, sub
%\begin{enumerate}[resume] %continues counting
\begin{document}
\begin{center}
\section*{Measure Theory}
Mark Ibrahim \\
*based on Principles of Real Analysis by Aliprantis and Burkinshaw
\end{center}

\tableofcontents
\begin{multicols}{2}

\section{Preliminaries}

a function $f: A \rightarrow B$ is \bt{continuous}
$\iff$ $f^{-1}$(open set) is an open set.

a bounded sequence $a_n$ has a \bt{$\limsup$} defined as 
$\lim_{N \rightarrow \infty} \sup\{a_N, a_{N+1}, \dots \}$
"largest tail"\\
$a_n$ converges if $\limsup = \liminf$.

A \bt{Hausdorff} topological space (T2 space) is a topological space where 
any two points can be seperated by open sets.

$\max\{a, b\} = \frac{a + b}{2} + \frac{|a - b|}{2}$.

union of countably sets is countable.

\section{Algebras and Measures}
\subsection{Semirings and Sigma-algebras of Sets (section 12)}

\subsubsection{semirings}
a collection $S$ of subsets of a set $X$ is called a 
\bt{semiring} if  
\begin{enumerate}
    \item $\emptyset \in S$, 
    \item $A \cap B \in S$, and 
    \item $A - B = C_1 \cup \dots C_n$ for $C_1, \dots C_n \in S$.
\end{enumerate}

Any countable union in $S$ can be written as a countable \bt{disjoint} union.

e.g., $S = \{[a, b) | a \leq b \in \R\}$ is a semiring, not an algebra. \\
* note $[a, a) = \emptyset$.


\subsubsection{algebras}

a nonempty collection $S$ of subsets of a set $X$ is an \bt{algebra} if 
\begin{enumerate}
    \item $A \cap B \in S$ 
    \item and $A^c \in S$. 
\end{enumerate}

Nice properties of algebras are: 
\begin{itemize}
    \item $\emptyset, X \in S$ 
    \item $S$ is closed under finite unions 
and finite intersections as well as subtraction
\end{itemize}

a \bt{$\sigma$-algebra} is an algebra that is closed under countable unions.

\bt{Borel sets} of a topological space (X, T) 
\footnote{(X, T) is a topological space with a set $X$ and subsets $T$ if
$\emptyset, X \in T$, and $T$ is closed under unions (even uncountable), finite intersections.}
is a $\sigma$-algebra 
generated by the open sets.

\subsection{Measures on Semirings (section 13)}
A function $\mu$ from a semiring $S$ to $[0, \infty]$ is a \bt{measure on $S$} if 
\begin{enumerate}
    \item $\mu(\emptyset) = 0$ 
    \item countably additive: $\mu(\cup_{n=1}^\infty A_n) = \sum_{n=1}^\infty \mu(A_n)$.
\end{enumerate}

\noindent $\bullet$ If $A \subseteq B$, ($A, B \in S$), then $\mu(A) \leq \mu(B)$.



Alternatively, can show $\mu$ is a measure if and only if "squeeze" 
\begin{enumerate}
    \item $\mu(\emptyset) = 0$
    \item $\sum_{i=1}^n \mu(A_i) \leq \mu(A)$ if $\cup_{i=1}^n A_i \subseteq A$ and $A_i$ are disjoint.
    \item $\mu(B) \leq \sum_{n=1}^\infty \mu(B_n)$, "subadditive"  if 
    $B \subseteq \cup_{n=1}^\infty B_n$.
\end{enumerate}

\subsubsection{Examples of Measures on S}
\begin{itemize}
    \item \bt{Counting Measure} $\mu(A) = |A|$ 
    \item \bt{Dirac Measure} Fix $a \in X$, $\mu_a(A) = 0$ if $a \not \in A$, else 1.
    \item \bt{Lebesgue Stieltjes} For $f: \R \rightarrow \R$, increasing, left continuous and
    $S = \{[a, b) | a \leq b \in \R \}$, $\mu([a, b)) = f(b) - f(a).$\\
    - \bt{Lebesgue Measure on $S$}, denoted $\lambda$ is defined by $\lambda([a, b)) = b-a.$
\end{itemize}

\subsection{Outer Measures (section 14)}

an \bt{outer measure} is a function $\bar \mu: P(X) \rightarrow [0, \infty$ such that 
\begin{enumerate}
    \item $\bar \mu(\emptyset) = 0$
    \item if $A \subseteq B$, $\bar \mu(A) \leq \mu(B)$
    \item countably subadditive: $\bar \mu( \cup_{n=1}^\infty A_n)$ 
    $\leq \sum_{n=1}^\infty \mu(A_n)$
\end{enumerate}

*an outer measure is not always a measure!

A subset $E$ of $X$ is \bt{measurable} if for all $A \subseteq X$,
$$\bar \mu(A) = \bar \mu(A \cap E) + \bar \mu(A \cap E^c)$$

A nicer equivalent way to show $E$ is measurable is by considering 
all $A$ in $S$ with $\mu^*(A) < \infty$ and showing 
$$\mu(A) \geq \mu^*(A \cap E) + \mu^*(A \cap E^c)$$

Nice Properites
\begin{itemize}
    \item every $A$ in S is $\mu^*$-measurable
    \item if $\bar \mu(E) = 0$, $E$ is measurable
    \item for $E_i$ measurable and any $A \subseteq X$, 
    $\bar \mu(\cup_{i=1}^n A \cap E_i) = \sum_{i=1}^n \bar \mu(A \cap E_i)$
\end{itemize}

the collection of measurable subsets is denoted by $\Lambda$.
This collection is a $\sigma$-algebra! 

Remarkably, the outer measure $\bar \mu$ restricted to $\Lambda$ is a measure!


\subsection{Outer Measures generated by a measure (section 15)}
The outer measure $\mu^*$ generated by a measure $\mu$ is defined for any subset $A$ of $X$, 
\centerline{$\mu^*(A) =$}
$$\inf\{\sum_{n=1}^\infty \mu(A_n) : A \subseteq \cup_{n=1}^\infty A_n 
\text{ for $A_n \in S$}\}$$

$\mu^*$ is called the Cath\'eodory extension of $\mu$.
By convention $\mu^*(A) = \infty$ if no cover exits in $S$.

On semiring $S$, $\mu* = \mu$.


For $E_n$ measurable, if $E_n \uparrow E$, then $\mu^*(E_n) \uparrow \mu^*(E)$
For $B_n$ measurable with $\mu^*(B_n) < \infty$, if $B_n \downarrow B$, then
$\mu^*(B_i) \downarrow \mu^*(B)$. \\k

a measure space if \bt{finite} if $\mu^*(X) < \infty$.

For $X$ a \bt{finite measure} space $E$ is measurable, if and only if 
$$\mu^*(E) + \mu^*(E^c) = \mu^*(X)$$

For all $A \subseteq X$, there is a measurable set $E$ such that 
$A \subseteq E$ and $\mu^*(A) = \mu^*(E)$.


\subsubsection{Cantor Set}
Cantor set $C = \cap_{n=1}^\infty c_n$, where \\
$c_1 = [0, 1] - (1/3, 2/3) $\\
$c_2 = c_1 - ((1/9, 2/9) \cup (7/9, 8/9))$ \\

each $c_n$ is closed, because it's a closed set minus open sets.

\begin{itemize}
    \item $C$ has measure 0
    \item $|C| = |\R|$
    \item every point of $C$ is an accumulation point of $C$
\end{itemize}

Vitali set is an example of a \bt{non-measurable} subset of $\R$.

\subsection{Lebesgue Measure (section 18)}

\bt{Outer Lebesgue measure} $\lambda^*$ is defined as 
$\lambda^*(A) = \inf \{ \sum_{i=n}^\infty \lambda^*(a_n, b_n) : A \subset \cup_{n=1}^\infty (a_n, b_n)\}$ 

* often, we say Lebesgue measure instead of outer Lebesgue measure. 

$E \subseteq \R$ is \bt{Lebesgue measurable} $\iff$ there is open $O \subseteq \R$ for each $\epsilon$ such that 
$E \subseteq O$ and $\lambda(O - E) < \epsilon$.


\subsubsection{Regular Borel Measure}
For $X$, a Hausdorff topological space and $B$ the borel sets in $X$, 
a measure $\mu$ on $B$ is called a \bt{regular borel measure} if 
\begin{enumerate}
    \item $\mu(K) < \infty$ if $K$ is compact 
    \item for $B$ a borel set, $\mu(B) =$\\ $\inf \{\mu(O) | O \text{ is open } B \subseteq O\}$ 
    \item for $O$ open, $\mu(O) = \\$  $\sup\{\mu(K) | K \text{ is compact and }$ $K \subseteq O\}$
\end{enumerate}

\begin{enumerate}
    \item $\lambda$ is a regular borel measure
    \item Durac measure is a regular borel measures
    \item Counting measure is not
    \item any \bt{translation invariant} regular borel measure on $\R$ is 
    $c \lambda$ for some $c \in \R^+$
\end{enumerate}

\section{Integration: functions}

\subsection{Measurable Functions (section 16)}
\subsection{Simple and step functions (section 17)}


\end{multicols}

\section{Questions}
\begin{enumerate}
    \item If $A \subseteq B$, is $\mu^*(B-A) = \mu^*(B) - \mu^*(A)$?
\end{enumerate}
\end{document}

