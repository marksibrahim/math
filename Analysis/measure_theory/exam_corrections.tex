

\documentclass[a4paper, 12pt]{article} 
\usepackage{amsmath, amssymb, color, graphicx, enumitem}
\usepackage{fullpage} %smaller margins
\usepackage{hyperref} % hyperlinks
\usepackage{mdframed} %for boxes

%font
%\usepackage[sc]{mathpazo}
%\linespread{1.05}         % Palladio needs more leading (space between lines)
%\usepackage[T1]{fontenc}

%font, libertine
\usepackage{libertine}

%word spacing
\usepackage{microtype}

%all equations get full space
\everymath{\displaystyle}

%useful shortcuts
\def\R{\ensuremath{\mathbb{R}}} %\ensuremath adds math mode, if forgotten
\def\Q{\ensuremath{\mathbb{Q}}}
\def\N{\ensuremath{\mathbb{N}}}
\def\Z{\ensuremath{\mathbb{Z}}}
\def\C{\ensuremath{\mathbb{C}}}

%shorcuts with arguments
\newcommand{\abs}[1]{\left\vert#1\right\vert} %nice absolute values
\newcommand{\bt}[1]{\textbf{#1}} %bold
\newcommand{\eq}[1]{\begin{align*}#1\end{align*}} %aligned equations
\newcommand{\norm}[1]{\left\lVert#1\right\rVert} %vector norm
\newcommand{\notimplies}{% does not imply
  \mathrel{{\ooalign{\hidewidth$\not\phantom{=}$\hidewidth\cr$\implies$}}}}
\renewcommand{\eq}[1]{\begin{align*}#1\end{align*}} %aligned equations

%piecewise function

%\begin{displaymath}
%   f(x) = \left\{
%     \begin{array}{lr}
%       1 & : x \in \mathbb{Q}\\
%       0 & : x \notin \mathbb{Q}
%     \end{array}
%   \right.
%\end{displaymath} 

%colors
\definecolor{javagreen}{rgb}{0.25,0.5,0.35} %dark green color
\definecolor{lightblue}{rgb}{0.149,0.545,0.824} %solarized blue
\definecolor{sred}{rgb}{0.863, 0.196, 0.184} %solarized red

\newcommand{\blue}[1]{{\leavevmode\color{lightblue}{#1}}} %solarized blue 
\newcommand{\green}[1]{{\leavevmode\color{javagreen}{#1}}} %command for green
\newcommand{\red}[1]{{\leavevmode\color{sred}{#1}}} %solarized red
\newcommand{\gray}[1]{{\leavevmode\color[gray]{0.5}{#1}}} %gray text

%environment
\newcommand{\tab}{\phantom{ssss}}

\title{}
\date{}
%==tips====
%part
    %section, sub, sub
%\begin{enumerate}[resume] %continues counting
\begin{document}
\begin{center}
\section*{Exam Corrections}
Measure Theory\\
Mark Ibrahim \\
\end{center}

\begin{enumerate}
    \item[3.] 
    If $\mu$ is a function from $S$ to $[0, \infty]$ such that $\mu(\emptyset) = 0$ and 
    $\mu$ is countably additive, meaning 
     if  $\cup_{i=1}^\infty A_i$ is in $S$ with all $A_i$ pairwise disjoint, then 
     $\sum_{i=1}^\infty \mu(A_i) = \mu(\cup_{i=1}^\infty A_i)$.

    \item[ 11. b)] 
    Since $\emptyset$ is finite, $\mu(\emptyset) = 0$ as needed.
    Next consider any pairwise disjoint sets $A_1, A_2, \dots$ in $S$ such that 
    $\cup_{i=1}^\infty A_i$ is in $S$. 
    Then $\cup_{i=1}^\infty A_i$ must be finite or $\cap_{i=1}^\infty A_i^c$ must be finite.
    If $\cup_{i=1}^\infty A_i$ is finite, then 
    $$\mu(\cup_{i=1}^\infty A_i) = 0 = \sum_{i=1}^\infty \mu(A_i), $$
    because each $A_i$ must be finite if the union of all $A_i$ is finite.
    If $\cap_{i=1}^\infty A_i^c$ is finite, then at least one 
    $A_i$ must be infinite, because $\cap_{i=1}^\infty A_i^c$ is finite
    (otherwise $\cup_{n=1}^\infty A_i$ would be at most countably infinite,
    implying $\cap_{i=1}^\infty A_i^c$ is infinite, a contradiction).
    If some $A_i$ is infinite, then $A_i^c$ is finite, because $A_i$ is in 
    $S$, implying all other $A_j \neq A_i$ are finite. Thus, 
    $\sum_{i=1}^\infty \mu(A_i) = 1 = \mu(\cup_{i=1}^\infty A_i)$.

    \item[11. c)]
    If $B$ is countable or finite, then $B = \cup_{i=1}^\infty \{a_i\}$, where
    $a_i$ are single points or empty sets.
    Thus, $0 \leq \mu^*(B) \leq \sum_{i=1}^\infty \mu(a_i) = 0$, because
    $B \subseteq \cup_{i=1}^\infty \{a_i\}$, implying $\mu^*(B) = 0$.
    Otherwise, $B$ is uncountable, meaning $B$ can't be covered by a union of 
    countable sets in $\R$. Thus, $\mu^*(B) > 0$. 
    Furthermore, $B \subseteq \R$, meaning $\mu^*(B) \leq \mu(\R) + \mu(\emptyset) + \dots = 1$.    Since $\mu-$values are in $\N \cup 0$, 
        \begin{displaymath}
           \mu^*(B) = \left\{
             \begin{array}{ll}
               0 &  \text{ if B is countable or finite} \\
               1 &  \text{ if B is uncountable}
             \end{array}
           \right.
        \end{displaymath} 
        are the possible measures of an arbitrary subset $B$.

    \item[14.]
    The set of points in infinitely sets $E_n$, 
    $E = \lim_{n \rightarrow \infty} E_n.$
    Each $\lambda(E_n) > 0$ implying $\sum_{n=k}^\infty E_n = 0$ 
    as $k \rightarrow \infty$, because $\sum_{n=1}^\infty \lambda(E_n) < \infty$.
    For any $k \in \N$, $E \subseteq \cup_{n=k}^\infty E_n$. 
    Thus as $k \rightarrow \infty$, 
    $$\lambda(E) \leq \sum_{n=k}^\infty \lambda(E_n) = 0.$$
    Since $\lambda(E) \geq 0$, $\lambda(E) = 0$ as desired.

    \item[15.]
    Let $B = \{x | f(x) \geq \epsilon\}$. Note $A \subseteq B$.
    Furthermore, $f(x) = |f(x)|$ except on a set of measure 0, call it $C$, because 
    $f(x) \geq 0$ a.e..
    Therefore, $B \subseteq \{x |\ |f(x)| \geq \epsilon\} \cup C$.
    Since $f$ is an integrable function, 
    by Theorem 22.5, 
    $\{x |\ |f(x)| \geq \epsilon\}$ has finite 
    measure.
    Thus, 
    \eq{ \mu(A) &\leq \mu(B) \\
    & \leq \mu( \{x |\ |f(x)| \geq \epsilon\} \cup C) \\
    & \leq \mu( \{x |\ |f(x)| \geq \epsilon\} + \mu(C) \\
    & = \mu(\{x | |f(x) \geq \epsilon\}) \\
    & < \infty.
    }

    \item[16.]
    Let $f_n(x) = n (e^{x/n} - 1).$
    Since $e^x = 1 + x + \frac{x^2}{2!} + \frac{x^3}{3!} + \dots$, 
    \eq{\lim_{n \rightarrow \infty} f_n(x) 
        & = \lim_{n \rightarrow \infty} n ( \frac{x}{n} + \frac{x^2}{n^2 2!} 
        + \frac{x^3}{n^33!} + \dots) \\
        & = \lim_{n \rightarrow \infty} x + \frac{x^2}{n 2!} + \frac{x^3}{n^2 3!} + \dots \\
    }
    Define $g_m(x) = \frac{x^m}{n^{m-1} m!}$. Then on $[0, 1]$ each $|g_m(x)|$ is bounded 
    by $\frac{1}{m!}$. Since $\sum_{m=1}^\infty \frac{1}{m!}$ converges (to $e$), 
    by the Weierstrass M-test, $\sum_{m=1}^\infty g_m(x)$ converges uniformly.
    Thus, 
    \eq{\lim_{n \rightarrow \infty} x + \frac{x^2}{n 2!} + \frac{x^3}{n^2 3!} + \dots 
    &= \lim_{n \rightarrow \infty} x + \lim_{n \rightarrow \infty}\frac{x^2}{n 2!} + \lim_{n \rightarrow \infty} \frac{x^3}{n^2 3!} + \dots \\
    & = x.
    }
    Furthermore, on $[0, 1]$, for all $n \in \N$, 
    \eq{|f_n(x)| &= \abs{x + \frac{x^2}{n 2!} + \frac{x^3}{n^2 3!} + \dots}\\
    & = x + \frac{x^2}{n 2!} + \frac{x^3}{n^2 3!} + \dots \\
    & \leq x + \frac{x^2}{2!} + \frac{x^3}{3!} + \dots 
    \tag{since $n \in \N$}\\
    & \leq 1 + x + \frac{x^2}{2!} + \dots = e^x.
    }
    Note each $f_n(x)$ is measurable on $[0, 1]$ because $f_n(x)$ is continuous.
    Since $e^x$ is Riemann integrable, by Lebesgue Dominated Convergence, 
    $$\lim_{n \rightarrow \infty} \int_0^1 f_n(x)\ dx = \int_0^1 x\ dx = \frac{1}{2}.$$
    

\end{enumerate}
\end{document}

