

\documentclass[a4paper, 12pt]{article} 
\usepackage{amsmath, amssymb, color, graphicx, enumitem}
\usepackage{fullpage} %smaller margins

%font
%\usepackage[sc]{mathpazo}
%\linespread{1.05}         % Palladio needs more leading (space between lines)
%\usepackage[T1]{fontenc}

%font, libertine
\usepackage{libertine}

%word spacing
\usepackage{microtype}


%useful shortcuts
\def\R{\ensuremath{\mathbb{R}}} %\ensuremath adds math mode, if forgotten
\def\Q{\ensuremath{\mathbb{Q}}}
\def\N{\ensuremath{\mathbb{N}}}
\def\Z{\ensuremath{\mathbb{Z}}}
\def\C{\ensuremath{\mathbb{C}}}

%shorcuts with arguments
\newcommand{\abs}[1]{\left\vert#1\right\vert} %nice absolute values
\newcommand{\bt}[1]{\textbf{#1}} %bold
\newcommand{\eq}[1]{\begin{align*}#1\end{align*}} %aligned equations
\newcommand{\cb}[1]{\centerline{\fbox{#1}}} %centered box
\newcommand{\bp}[1]{\fbox{\parbox{0.8\textwidth}{#1}}} %box paragraph
\newcommand{\norm}[1]{\left\lVert#1\right\rVert} %vector norm
\newcommand{\notimplies}{% does not imply
  \mathrel{{\ooalign{\hidewidth$\not\phantom{=}$\hidewidth\cr$\implies$}}}}
\renewcommand{\eq}[1]{\begin{align*}#1\end{align*}} %aligned equations


%colors
\definecolor{javagreen}{rgb}{0.25,0.5,0.35} %dark green color
\newcommand{\green}[1]{\textcolor{javagreen}{#1}} %command for green
\newcommand{\gray}[1]{\textcolor[gray]{0.5}{#1}} %gray text

%environment
\newcommand{\tab}{\phantom{ssss}}


\title{}
\date{}
%==tips====
%part
    %section, sub, sub
%\begin{enumerate}[resume] %continues counting
\begin{document}
\begin{center}
\section*{Measure Theory: A Primer for Dummies}
Mark \\
\end{center}

\section{$\sigma$-algebras}
A \bt{$\sigma$-algebra} is a collection of subsets of $\Omega$ with nice properties:
\begin{itemize}
    \item $\emptyset \in \sigma$
    \item if $A \in \sigma$ then $A^c \in \sigma$
    \item closed under countable unions 
\end{itemize}
(an \bt{algebra} requires the above but only under finitely many operations)



\section{Measurable Space}
A \bt{measurable space} is a collection of \bt{events}, $\beta$, and a sample space $\Omega$ (outcomes).


\gray{A sample space makes it possible to talk about complements of an event.}

\textcolor{red}{still a bit unclear}


\section{Measure}
A \bt{measure} $\mu$ on a set $A \subseteq \Omega$ is a map from $A \rightarrow \R^+$.\\
\gray{*idea of a measure is to generalize the notion of volume or length}

A \bt{measure space} is a measure space is a measureable space + a measure, written ($\Omega, \beta, \mu$).

A \bt{support} of a measure is all sets with nonzero measure. \\
a \bt{measure with compact support} means the sets of nonzero measure form a compact set.
(in $\R$ this is closed and bounded, by Heine-Borel)

\section{Lebesgue}

The \bt{Lebesgue Measure} $\mu_L(A)$ = volume or length of a set A. \\
e.g., $\mu_L([0,1]) =1$.


A huge result from measure theory is \bt{Lebesgue's Dominated Convergence Theorem}
For $f_n \rightarrow f$, 
\eq{
\int f\,dx = \lim \int f_n\,dx
}
which doesn't hold for the Riemann integral.

\section{Inequalities}
For $a_i, b_i \in \C$, 

\bt{Cauchy-Schwarz}
\eq{
(\sum a_ib_i)^2 \leq \sum a_i^2 \sum b_i^2
}

\bt{Holder's Inequality}
Generalized of Cauchy-Schwarz and used to prove Minkowski's inequality.
\eq{
(\sum |a_i + b_i|^p)^{1/p} \leq (\sum |a_i|^p)^{1/p} (\sum |b_i|^q)^{1/q}
\tag{for $1/q + 1/p = 1$ and $1\leq p$.}
}

\bt{Minkowski's Inequality}
\eq{
(\sum |a_i + b_i|^p)^{1/p} \leq (\sum |a_i|^p)^{1/p} (\sum |b_i|^q)^{1/q}
}

\section*{Resources}
Frank Jones, "Lebesgue Integration on Euclidean space"


\section*{Analysis Facts}

Cauchy in $\R$ $\iff$ convergent. 
(in general convergent $\implies$ cauchy, but not the other way!)

\bt{Complete} means every Cauchy sequence converges.


\textcolor{red}{continue http://webbuild.knu.ac.kr/~trj/Analysis/Chandalia.pdf}
\end{document}

