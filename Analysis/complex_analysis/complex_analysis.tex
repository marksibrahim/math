\documentclass[a4paper, 12pt]{article} 
\usepackage{amsmath, amssymb, color, graphicx, enumitem}
\usepackage{fullpage} %smaller margins
\usepackage{hyperref} % hyperlinks
\usepackage{multicol}
\usepackage[framemethod=tikz]{mdframed} %boxes 

%columns separate and line
\setlength{\columnsep}{1.5cm}
\setlength{\columnseprule}{0.2pt}

%font

%\usepackage[sc]{mathpazo}
%\linespread{1.05}         % Palladio needs more leading (space between lines)
%\usepackage[T1]{fontenc}

%font, libertine
\usepackage{libertine}

%word spacing
\usepackage{microtype}

%all equations get full space
\everymath{\displaystyle}

%useful shortcuts
\def\R{\ensuremath{\mathbb{R}}} %\ensuremath adds math mode, if forgotten
\def\Q{\ensuremath{\mathbb{Q}}}
\def\N{\ensuremath{\mathbb{N}}}
\def\Z{\ensuremath{\mathbb{Z}}}
\def\C{\ensuremath{\mathbb{C}}}

%shorcuts with arguments
\newcommand{\abs}[1]{\left\vert#1\right\vert} %nice absolute values
\newcommand{\bt}[1]{\textbf{#1}} %bold
\newcommand{\eq}[1]{\begin{align*}#1\end{align*}} %aligned equations
\newcommand{\norm}[1]{\left\lVert#1\right\rVert} %vector norm
\newcommand{\notimplies}{% does not imply
  \mathrel{{\ooalign{\hidewidth$\not\phantom{=}$\hidewidth\cr$\implies$}}}}
\renewcommand{\eq}[1]{\begin{align*}#1\end{align*}} %aligned equations

%piecewise function

%\begin{displaymath}
%   f(x) = \left\{
%     \begin{array}{lr}
%       1 & : x \in \mathbb{Q}\\
%       0 & : x \notin \mathbb{Q}
%     \end{array}
%   \right.
%\end{displaymath} 

%colors
\definecolor{javagreen}{rgb}{0.25,0.5,0.35} %dark green color
\definecolor{lightblue}{rgb}{0.149,0.545,0.824} %solarized blue
\definecolor{sred}{rgb}{0.863, 0.196, 0.184} %solarized red

\newcommand{\blue}[1]{{\leavevmode\color{lightblue}{#1}}} %solarized blue 
\newcommand{\green}[1]{{\leavevmode\color{javagreen}{#1}}} %command for green
\newcommand{\red}[1]{{\leavevmode\color{sred}{#1}}} %solarized red
\newcommand{\gray}[1]{{\leavevmode\color[gray]{0.5}{#1}}} %gray text

%environment
\newcommand{\tab}{\phantom{ssss}}

\title{}
\date{}
%==tips====
%part
    %section, sub, sub
%\begin{enumerate}[resume] %continues counting
\begin{document}
\begin{center}
\section*{Complex Analysis}
Mark Ibrahim \\
based on Conway 's Functions of One Complex Variable
\end{center}

\tableofcontents
\begin{multicols}{2}
\section{Complex Number System (I.1 - I.4)}

Bombelli discovered it's useful to consider negative square roots
after studying the equation: $x^3 = 5x+3$

The set of complex numbers, $\C$, is sometimes called 
the \bt{Complex Plane}.
Since any complex number $z$ can be written as $x + iy$, we can 
identify any complex number in $\R^2$.

\subsection{Arithmetic in Complex Plane}

Addition is akin to adding vectors in $\R^2$, because we add the 
real and imaginary parts.
Multiplication is as expected; also the associative, commutative, and 
distributive properties hold. In fact, $\C$ is a field

The \bt{modulus} of a complex number $z = x + iy$ is the length of the vector: 
$$| z | = \sqrt{x^2 + y^2}$$

Modulus Properties
\begin{itemize}
    \item |zw| = |z| |w| \\
        \gray{write $z$ as $z = |z| \hat{z}$, where $\hat{z}$ is the unit
            vector in the direction of $z$\\
        then you can expand $zw$ to obtain result.}
    \item |$\bar{z}$| = |z| \\
        \gray{because you're squaring entries to find the length}
\end{itemize}


The \bt{complex conjugate} of $z$, denoted $\bar{z} = x - iy$.
Note $z \bar{z} = |z|^2$.

Trick: Re $z = \frac{ z + \bar{z}}{2}$.

Note to find the multiplicative inverse of $z$, it's 
$$\frac{1}{z} = \frac{x - iy}{x^2 + y^2}$$
\gray{(equivalently $\frac{\bar{z}}{| z |^2}$)}

The distance between two vectors $z, w$ is $| z - w |$.

Can show Triangle on modulus holds: $| z + w | \leq | z | + | w |$.
\gray{* can extend to $| z_1 + z_2 + z_3 + \dots | \leq |z_1| + |z_2| + \dots$ by 
induction.}

Note $| z + w | = | z | + | w | \iff |z \bar{w}| = Re(z\bar{w})$

A useful variant of the triangle inequality is 
$$\abs{|z| - |w|} \leq |z - w|$$

\subsection{Polar Coordinates}

We can express a complex number $z$ as a vector in $\R^2$ using Polar 
Coordinates: $r = | z |$ and an angle $\theta$. \\
\gray{note $x = | z | \cos(\theta)$}

The \bt{principal argument} of $z$ is $\theta$ restricted to values between
$-\pi$ and $\pi$.

\blue{e.g., Arg(1 - i) = $\frac{-\pi}{4}$}

Often we write $z$ in terms of cos and sin: $z = r(\cos(\theta) + i\sin(\theta))$\\
\gray{in exponential notation, $z = r e^{i\theta}$ by euler's identity}.

\blue{e.g., $e^{i\pi} = \cos(\pi) + i \sin(\pi) = -1$.}
* note conway denotes $\cos(\theta) + i \sin(\theta)$ as $cis(\theta)$

nice properties:\\
arg($\bar{z}$) = - arg($z$) \\
\gray{can see this by drawing vector in $\R^2$}\\
arg($z_1z_2$) = arg($z_1$) + arg($z_2$)

\bt{Multiplication} of $z_1 z_2$ results in: adding angles and multiplying lengths

\bt{De Moivre's Formula}: $(e^{i\theta})^n = e^{in\theta}$, meaning
$$(\cos(\theta) + i \sin(\theta))^n = \cos(n\theta) + i \sin(n\theta)$$

\subsection{Nth Roots}

An nth root of $w$ is a number $z$ such that $z^n = w$.

There are $n$ distinct roots for a number $w \neq 0$.
For $w = \rho e^{i\phi}$, they are $\sqrt[n]{\rho}e^{i(\frac{\phi}{n} + 
\frac{2kpi}{n})}$
for $k = 0, 1, \dots, n-1$.

\bt{nth roots of unity} are the nth roots of 1.

\section{Metric Spaces and Topology in C}

\subsection{Review}
a set (in a metric space) is \bt{compact} if every open cover has a finite
subcover.

A metric space is \bt{sequentially compact} if every sequence has a convergent 
subsequence.

It turns out a metric space is compact $\iff$ it's sequentially compact.

\end{multicols}

\end{document}

