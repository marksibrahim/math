

\documentclass[12pt]{article}

\usepackage{amsmath, amssymb, color, enumitem, framed}
%for box use \begin{mdframed}

\title{Analysis Final Review}
\author{Mark}
\date{}

\definecolor{indigo(dye)}{rgb}{0.0, 0.25, 0.42}
\color{indigo(dye)}

\begin{document}
\maketitle
\centerline{\Large \textbf{Tools: set theory, functions}}
\medskip

a \textbf{set} is a collection of things. \\

\textbf{Powerset} of $A$ is the set of all subsets of $A$.\\

\textbf{De Morgan's Law: }$\complement (A \cap B) = \complement(A) \cup \complement(B)$\\
\indent \textcolor[gray]{0.5}{think of rows and columns of students in a room.}\\
\indent Also, 
$\complement(A \cup B) = \complement(A) \cap \complement(B)$\\

\noindent an \textbf{equivalence relation} partions a set into "equivalence classes"; it's \\
    \indent reflexive: $a \equiv a$ \\ 
    \indent symmetric: $a \equiv b \iff b \equiv a$ \\
    \indent transitive: $if a \equiv b \ and \ b \equiv c, then\ a \equiv c$ \\


\noindent a \textbf{function} is a rule that sends each element of one set to a specific element of another set. 

$$f: X \rightarrow Y$$ \\
$X, Y$ are sets: $X$ is domain, $Y$ co-domain\\

\noindent one-to-one or \textbf{injective}: "no two inputs are sent to the same output"\\
onto or \textbf{surjective}: "each item in $Y$ is hit with some $X$"\\
\textbf{bijective}: injective and surjective

\section*{Real Numbers}

Begin with numbers as objects and three axioms, \\
1. \textbf{Field}: \textcolor[gray]{0.5}{+, * associative, commutative, inverses and distribution}\\
2. \textbf{Order}: + is closed over $\mathbb{R}^+$ and any $a \in \mathbb{R}$ is in $\mathbb{R}^+$, is 0, or $\mathbb{R}^-$\\
3. \textbf{Upper Bound property}: every \textbf{nonempty} set that is bounded above has a \textbf{least} upper bound.  \\ 

\subsection*{Field Properties}


For $a, b$ in a field, \\
\begin{center}
\fbox{\parbox{0.8\linewidth}{
\centerline{\textbf{Master Tool}}\\ 
\ \\
\centerline{$x + a = b$ and $ax = b$ have one \textbf{unique solution}.}\\
\centerline{\textcolor[gray]{0.5}{use $-a$ and $a^{-1}$ to show}}}}
\end{center}
\medskip

$a0 = 0$ for all $a$ is a consequence, as are others we expect.\\

\subsection*{Order}
\textbf{Trichotomy} (any $a, b \in \mathbb{R}$): $a<b$, $a=b$, or $a>b$.\\ 




\noindent \textbf{Archimedean Property}: there is an $n \in \mathbb{Z}$ for any $\varepsilon$ such that $\frac{1}{n} < \varepsilon$

useful\\
for proofs like $a^ma^n= a^{m+n}$ consider all cases $\pm m$ and $n$.\\
$|x-a| < \epsilon$ implies $a - \epsilon < x < a + \epsilon$.\\
\subsection*{Existence of Square Roots}
Every positive number has a unique square root. \\
\textcolor[gray]{0.5}{modify proof below to show}\\

There is an $x \in \mathbb{R}^+$ such that $x^2 = 3$.
\textcolor[gray]{0.5}{
Consider the $lub$, call it $r$, of $\{ x \in \mathbb{R} : x^2 < 3\}$. Show 10 is an upper bound and that $r : r^2=3$ is the $lub$. We negate the possibilities that $r^2 < 3$ and $r^2 > 3$.\\
Suppose $r^2<3$, then we want to show $(r+\epsilon)^2 \in$ set (for $0<\epsilon<1$, but is greater than $r^2$. KEY: $r^2 + 2r\epsilon + \epsilon ^2 < r^2 + 2r\epsilon + \epsilon$. Then, it becomes obvious $\epsilon = \frac{3-r^2}{2r+1}$ to yield the result we want. \\
Next suppose $r^2>3$. Then show $(r-\epsilon)^2$ is an upper bound smaller than the least upper bound.
}
\subsection*{Triangle Inequality}

For $a, b \in \mathbb{Z}$,
$$|a| - |b| \leq |a -b| \leq |a| + |b|$$
\textcolor[gray]{0.5}{
$\pm a \leq |a|, \text{ means both } a \text{ and } -a \text{are} \leq a$\\
then, $+- a + +-b \leq |a| + |b|$\\
we only use two of these cases: $a + b$ and $-a -b$\\
to get $|a + b| \leq |a| + |b|$ \\
left part\\
trick $|a| = |b - (b - a)| \leq |b| + |b - a|$ using triangle \\
}

\fbox{idea: $|a| -|b|$ is a definitive decrease; $|a| + |b|$ increase}

\subsection*{Cauchy-Schwarz}
For $a_1,..., a_n$ and $b_1, ..., b_n \in \mathbb{R}$, 
$$(\sum_{k=1}^n a_k b_k)^2 \leq \sum_{k=1}^n a_k^2 \sum_{k=1}^n b_k^2$$

In vector land, 

$$( \vec a \cdot \vec b\ )^2 \leq ||\vec a||^2\  ||\vec b||^2$$
$||\vec a||$ is the length of a.

sometimes also written as $|\vec a \vec b| \leq ||\vec a ||\ ||\vec b||$.


\subsection*{Complex Numbers}

\noindent complex numbers $\mathbb{C}$ have no ordering since $i^2$ is negative.\\

\textbf{Absolute Value} of a + bi is defined as 
$$|a + bi| = \sqrt{a^2 + b^2}$$

Triangle Inequality holds for complexy numbers.\\

$$e^{iy} = \cos y + i\sin y $$
\centerline{\textcolor[gray]{0.5}{use derivatives to intuit why}}
\medskip

We can also represent complex numbers using polar coordinates with $\theta$ and $d$ as 
$$r \cos \theta + i r \sin \theta$$

Hyperbolic $\sin$ and $\cos$ are ways of extending trig to complex numbers. They are defined as 
$$\sinh x = \frac{e^{ix} - e^{-ix}}{2}$$
$$\cosh x = \frac{e^{ix} + e^{-ix}}{2i}$$
\section*{Metric Spaces}

To speak of points near each other, continuity or limits, we need to understand the space in which these objects live: metric spaces. \\

metric spaces are a generalization of the real line, plane, or 3-d space. \\

\noindent a \textbf{metric space} a set $E$ together with a rule $d$ associating a pair to elements to a real number such that for any $a, b \in E$:

\begin{enumerate}
    \item $d(a, b) \geq 0$ 
    \item $d(a, b) = 0,$ if and only if $a = b$, definite 
    \item $d(a, b) = d(b, a)$, symmteric
    \item $d(a, c) \leq d(a, b) + d(b, c)$, Triangle inequality \\
\end{enumerate}

\subsection*{Examples}
\begin{itemize}
    \item discrete metric space: 0 if same point; 1 otherwise
    \item $E: d(a, b) = |a-b|$
    \item $E^n (\text{in } \mathbb{R}^n): d(a, b) = \sqrt{|a_1-b_1|^2 + ...+|a_n-b_n|^2}$\\
    "Euclidean": $\mathbb{R}^n \times \mathbb{R}^n \rightarrow \mathbb{R}$
    \item Taxicab: $d(a, b) = |a_1-b_1| + |a_2 - b_2|$\\
    "distance one would travel on streets by taxi", $\mathbb{R}^2 \times \mathbb{R}^2 \rightarrow \mathbb{R}$

    \item Sup ("max metric"): $d(a, b) = max\{|a_1 - b_1|, |a_2 - b_2|\}$ \\
    $d: \mathbb{R}^2 \times \mathbb{R}^2 \rightarrow \mathbb{R}$
\section*{Closeness}
\subsection*{Open Ball}
Let $(X, d)$ be a metric space. An \textbf{open} ball of radius $r$ ($r \in \mathbb{R}^+$) and center $p \in X$ is 
$$B(p, r) = \{ y \in X: d(y, p) < r\}$$

\subsection*{Open Set G}
\begin{framed}
\begin{itemize}
    \item every point is center of open ball in $G$
    \item union of open balls
\end{itemize}
\end{framed}


Properties
\begin{itemize}
    \item the entire metric space is open; empty set is too!
    \item the union of any collection of open sets is open
    \item the interesection of a \textit{finite} collection of open sets is open
\end{itemize}
\bigskip
Examples \\
Open in $E^1$ is just an open interval \\
$E^2$ is a circle \\
Taxicab in $\mathbb{R}^2$ is a square pointing up

\subsection*{Closed Set G}
\begin{framed}
\begin{itemize}
    \item $\complement G$ is open
    \item every sequence in $G$ converges in $G$
\end{itemize}
\end{framed}

Properties 
\begin{itemize}
    \item entire set is closed (both closed and open); empty set too
    \item intersection of closed sets is closed
    \item union of \textbf{finite} closed sets is closed\\
\end{itemize}
\bigskip
Subsets of a metric can be \textbf{neither closed nor open}. 
e.g., $[a, b)$\\

A subset of a metric space is \textbf{bounded} if it's contained in some ball (open or closed).

\section*{Completeness}

{\bf Cauchy}: elements as close as we wish beyond a certain point \\ 

\noindent {\bf Complete}: every Cauchy sequence converges. \\

\textbf{Convergent $\rightarrow$ Cauchy}
\textcolor[gray]{0.5}{use triangle}

\begin{itemize}
    \item subsequence of Cauchy is Cauchy\\
        \textcolor[gray]{0.5}{any two elements are as close as we wish, so any subcollection has to be too.}
    \item Cauchy with convergent subsequence is convergent \\ 
        \textcolor[gray]{0.5}{For $p_m$ the convergent subsequence:\\
        $d(p, p_n) \leq d(p, p_m) + d(p_m, p_n) < \epsilon$\\}
\end{itemize}
\bigskip

$\mathbb{R}^n$ is complete (that is $E^n$).\\

Any closed subset of complete space is complete\\
\textcolor[gray]{0.5}{complete so cauchy converges in space. However, subset is closed, so must converge in subset.\\}

accumulation point = limit point.\\


\section*{Compactness}
\fbox{every open cover of $S$, has a \textbf{finite subcover}.}\\
\medskip

\noindent subcover is a finite subset of the open cover.\\
\textbf{open cover}: union of open balls.\\

\noindent Any \textbf{closed subset} of compact is compact\\
\textcolor[gray]{0.5}{$K$ compact; $S$ closed subset. \\
Any open cover of $S$ can be expanded by another open cover of $\complement S$, which has a finite subcover.\\}

\noindent \textbf{closed cell} in $E^m$ set of points with each coordinate inside some interval $[a, b]$.\\

\textbf{cluster point} of a set, need not be in the set.

\subsection*{Bolzano Weirstrauss}
Every bounded sequence in $R^n$ has a convergent subsequence\\

\textcolor[gray]{0.5}{
For any sequence $a_n$, \\
$a_n \subset$ interval [a, b] since bounded\\
infinitely many points of $a_n$ in either $[\frac{a+b}{2}, b]$ or $[a, \frac{a+b}{2}]$.\\
Continue shrinking interval to construct a convergent subsequence.
}

\bigskip

Compact is 
\begin{itemize}
    \item complete \textcolor[gray]{0.5}{every cauchy has a convergent subsequence$\rightarrow$ every cauchy converges by theorem}
    \item closed \textcolor[gray]{0.5}{every convergent sequence converges to the same limit as its subsequence, so limit is in the set.}
\end{itemize}

\subsection*{Heine-Borel}

Any closed, bounded subset of $\mathbb{R}^n$ is compact \\
\textcolor[gray]{0.5}{
unclear, but use contradiction}

\subsection*{Nested Cells}
In compact metric space, \\
$...\subset S_3 \subset S_2 \subset S_1 $\\
for $S_i$ closed set, at least one point in all $S_i$.\\
\textcolor[gray]{0.5}{Alternatively stated as the $\cap$ infinite nested closed cells is nonempty.\\}

Intuitively: nested [a, b] intersect at the least at one point.\\

counter example: nested rays

\section*{Connectedness}

Space $E$ is connected if only subsets that are both open and closed are:\\
$E$ itself and empty-set. \\

A subset is connected if it forms a connected subsapce.

Any interval in $\mathbb{R}$ is connected (open or closed).\\

Connected $\rightarrow$ can't be written as \textbf{two disjoint open} sets.\\

Two disjoint open, imply a hole, so space is not connected.

\section*{Continuous Functions}
$$f: E \rightarrow E' \text{ is \textbf{continuous}}$$
$$\iff$$
$$\text{for every \textbf{open} set } S \text{ in } E', f^{-1}(S) \textbf{\text{ is open}}$$
\centerline{***}
"inverse image of open is open if and only if f is continuous"
(also works with closed sets)

\subsection*{Continuity on Compact Sets}
$$f: E \rightarrow E' \text{ continuous}$$
$$E \text{ compact}$$

Big: \\
\centerline{image of compact is compact }
\ \\
\textcolor[gray]{0.5}{
\noindent Let $U$ be an open cover of $f(E)$. \\
Then, $f^{-1}(U)$ is open, as $f$ is continuous.\\
Hence, there is a finite cover $f^{-1}(U)$ containing $E$\\
Thus, 
$$f(E) \subset f(\text{ finite cover of $f^{-1}(U)$ }) \subset \text{ finite cover of } U$$ }

\noindent consequences, 
\subsubsection*{Extreme Value Theorem}
\textcolor[gray]{0.5}{$f([a, b])$ compact $\rightarrow$ closed, bounded. \\
Thus, l.u.b. exists.}
\ \\
\noindent For $K$ a compact set, 

\begin{enumerate}
    \item continuous on $K \rightarrow$ uniformly continuous
    \item there is a neast point in $K$ to any point
    \item $f$ continuous and bijective in $K \rightarrow$ $f^{-1}$ continuous
\end{enumerate}


\subsection*{Continuity on Connected Sets}

$$f: E \rightarrow E' \text{ connected}$$
$$E \text{ connected}$$

Big: \\
\centerline{image of connected is connected}

Consequences, 
\begin{enumerate}
    \item convex set $\rightarrow$ connected
    \item $f_i$ a sequence of \textbf{uniformly} continuous functions\\ 
$\rightarrow$ $f$ the \textbf{limit is continuous}
\textcolor[gray]{0.5}{proof: $\epsilon / 3$ theorem}
    \item Cauchy Criterion \\
    $f_n$ is \textbf{uniformly Cauchy} if there is $N$ such that for all $m, n >N$, 
$$d(f_n(x), f_m(x) < \epsilon \text{ for all x}$$
$f_n$ uniformly cauchy in a complete space $\rightarrow$ $f_n$ converges uniformly 

\end{enumerate}


Curiosities:
"Coffee Cup Theorem," Brower Fixed Point: \\
for $f([0, 1]) \rightarrow [0, 1]$ continuous, there is $p$ such that $f(p) = p$.

\section*{Series}

series with terms $a_k$ \textbf{converges} if $\lim_{n \rightarrow \infty} \sum_{k=1}^n a_k$ exists


\subsection*{Geomteric Series}


$$\sum_{k=0}^\infty r^k = \frac{1}{1-r} \text{ for } |r| < 1$$

\textcolor[gray]{0.5}{compute $s_n - rs_n$ where $s_n$ is the nth partial sum for proof}

\subsection*{Cauchy Sequence Criterion}

series converges $\iff |S_m - S_n| < \epsilon$, \ \ \  for all $m, n > $ some $N$.\\

\noindent consequence: if series converges, $\lim_{n \rightarrow \infty} a_n = 0$.

\subsection*{Comparison Test}

if $|b_n| < a_n$ for all $n$ and series of $a_n$ converges, so does $\sum b_n$.



\subsection*{Alternating Series Test}

The $\sum (-1)^k a_k$ converges if, 
$$1. a_k > 0$$
$$2. a_k > a_{k+1}$$
$$3. \lim_{n \rightarrow \infty} a_k = 0$$


conditional convergence: series $|a_n|$ diverges, but series of $a_n$ converges.

\subsection*{Rearrangement}

For an absolutely convergent series, any rearrangement conveges to same value.\\

Conditional convergent series, may have a divergent rearrangement!

\subsection*{Ratio Test}

Let $R = \lim_{n \rightarrow \infty} |\frac{a_{n+1}}{a_n}|$. \\
Then, 
\begin{center}
series converges for $R < 1$\\
diverges for $R > 1$. \\
*no info if $R =1$.
\end{center}


\section*{Series of Functions}

\subsection*{Weierstrauss M-test}

For a series of real-valued function on a set $S$, if 
$$|f_k(x)| \leq M_k \text{ \ \ \ for all x } \in S$$
where $M_k$ is a \textbf{convergent} series of constants, 

$$\sum f_k(x) \text{ converges uniformly (and absolutely) }$$

\noindent * if each $f_k$ is continuous, the series converges to a continuous function.


\textcolor{red}{look at apostle for sequences\\
index cards \\
practice problems \\
add proofs?}

\subsection*{Exercises From Apostle}
http://www.math.ucla.edu/~hendricks/Math131B.html

 Apostle Problems: 2.9, 2.15, 2.18, 2.19, 3.27, 3.28, and 3.29\\
3.26, 3.30, 3.31, 3.2(a)-(d),(f),(g), 3.12(b),(c),(f), 3.43, and 3.46\\

\begin{itemize}

    \item[apostle 2.15] a number is called \textbf{algebraic} if it's the root of a polynomial with integer coefficients. Algebraic numbers are coutably infinite, since an nth degree polynomial has at most n roots, thus the total is countable.
    (non-algebraic numbers like $\pi$ are \textbf{transcendental})

    \item[apostle 2.18] Is the set of sequences made up of $0$ or $1$ countable?
    \textcolor[gray]{0.5}{No. Suppose you were, then you can create a list of all sequences in the set. Then construct a sequence whose terms differ from each in the list...(see solution); it's a bit weird}

\end{itemize}

\end{document}
