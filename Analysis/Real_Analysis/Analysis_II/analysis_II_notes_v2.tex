

\documentclass[a4paper, 12pt]{article} 
\usepackage{amsmath, amssymb, color, graphicx, enumitem}
\usepackage{fullpage} %smaller margins

%font
\usepackage[sc]{mathpazo}
\linespread{1.05}         % Palladio needs more leading (space between lines)
\usepackage[T1]{fontenc}

%word spacing
\usepackage{microtype}

\usepackage{multicol}

%useful shortcuts
\def\R{\ensuremath{\mathbb{R}}} %\ensuremath adds math mode, if forgotten
\def\Q{\ensuremath{\mathbb{Q}}}
\def\N{\ensuremath{\mathbb{N}}}
\def\Z{\ensuremath{\mathbb{Z}}}
\def\C{\ensuremath{\mathbb{C}}}

%commands with arguments
\newcommand{\abs}[1]{\left\vert#1\right\vert} %nice absolute values
\newcommand{\bt}[1]{\textbf{#1}} %bold
\newcommand{\eq}[1]{\begin{align*}#1\end{align*}} %aligned equations
\newcommand{\cb}[1]{\centerline{\fbox{#1}}} %centered box
\newcommand{\bp}[1]{\fbox{\parbox{0.8\textwidth}{#1}}} %box paragraph
\newcommand{\notimplies}{% does not imply
  \mathrel{{\ooalign{\hidewidth$\not\phantom{=}$\hidewidth\cr$\implies$}}}}


%colors
\definecolor{javagreen}{rgb}{0.25,0.5,0.35} %dark green color
\newcommand{\green}[1]{\textcolor{javagreen}{#1}} %command for green
\newcommand{\gray}[1]{\textcolor[gray]{0.5}{#1}} %gray text

%environment
\renewcommand{\eq}[1]{\begin{align*}#1\end{align*}} %aligned equations
\newcommand{\sn}[1]{\reversemarginpar \marginpar{{\small\gray{#1}}}}

\definecolor{indigo(dye)}{rgb}{0.0, 0.25, 0.45}
\color{indigo(dye)}

\title{}
\date{}
%==tips====
%part
    %section, sub, sub
%\begin{enumerate}[resume] %continues counting
\begin{document}
\begin{center}
\section*{Analysis Continued}
Mark \\
\end{center}


\section*{Power Series}
\sn{def}
A \bt{power series} is of the form 
\eq{
\sum_{n=0}^\infty a_n x^n
}

*can be centered at $x_0$ we can write it as $\sum_{n=0}^\infty a_n(x-x_0)^n$. \\

\sn{uniform convergence}
If power series converges at a \bt{single point} say $z$, then the power series \bt{converges uniformly} on for all $ r < |x_0|$.\\

\sn{converge to diff func}
The power series converges to a \bt{differentiable $f$} inside the radius of convergence.


\section*{Taylor's Theorem}

$f: [a, b] \rightarrow \R$, infinitely differentiable, derivatives cont., and $f^{(k)}$ is finite, \\ then \\
there exists $x_1$ for any $c \in [a, b]$ and all $x \neq c$.
\eq{
f(x) = \underbrace{\sum_{k=0}^{n-1} \frac{f^{(k)}(c)}{k!} (x-c)^k}_{\text{Taylor Poly}} + \underbrace{\frac{f^{(n)}(x_1)}{n!} (x-c)^n}_{\text{Taylor Remainder}}
}
*note: $x_1$ depends on $n, x, c$

see notes for general form with $f(x)$ and $g(x)$.

\textcolor{red}{take better notes on Power series and Taylor using book}

\section*{Multivariable Derivatives}
Let $f: \R^n \rightarrow \R^m$ \\

Two bad attempts: partial derivative, \bt{directional derivative}. \\

\bt{partial derivative} denoted $D_k f(c) = \frac{\partial f(c)}{\partial x_k}$)\\

both are bad because they don't imply continuity as we'd like.\\
Instead we define the \bt{Total Derivative}, which works as we wish.\\

\subsection*{directional derivative}
The derivative of $f$ in the direction of $u$ is
\eq{
f'(\vec c, \vec u) = \lim_{h \rightarrow 0} \frac{f(\vec c + h \vec u) - f(\vec c)}{h}
}
*often other books require $\vec u$ to be a unit vector, but not here

\textcolor{red}{linear algebra review in notebook}

\subsection*{total derivative}

\sn{correct def}
The function $f$ is \bt{differentiable} at $a$ if there exists a linear transformation $T_a$ such that
\eq{f(a + v) = f(a) + T_a(v) + ||v|| E_c(v)
}
where $E_c(v) \rightarrow 0$ as $v \rightarrow 0$.\\

*$||v||E_c(v)$ can be written using "little o" notation as $o(||v||)$. \\
\bt{little o} notation: $f = o(g)$ as $x \rightarrow c$ if $\lim_{x \rightarrow c} \frac{f(x)}{g(x)} = 0$.\\

\sn{cont}
If $f$ is differentiable at $a$, then $f$ is continous at $a$. \\

\sn{directional deriv}
If $f$ is differentiable at $a$, then $f'(a; u)$ exists and $f'(a; u) = Au$ for any $u$

\subsection*{Derivatives in Matrices}



\end{document}

