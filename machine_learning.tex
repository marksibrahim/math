


\documentclass[12pt]{article}

\usepackage{amsmath, amssymb, color, enumitem, graphicx, hyperref, fullpage}
% \fbox{\parbox{0.8\linewidth}{
%\newcommand{\bt}[1]{\textbf{#1}}
\title{Machine Learning}
\author{Mark}
\date{}



%useful shortcuts
\def\R{\ensuremath{\mathbb{R}}} %\ensuremath adds math mode, if forgotten
\def\Q{\ensuremath{\mathbb{Q}}}
\def\N{\ensuremath{\mathbb{N}}}
\def\Z{\ensuremath{\mathbb{Z}}}

%shorcuts with arguments
\newcommand{\abs}[1]{\left\vert#1\right\vert} %nice absolute values
\newcommand{\bt}[1]{\textbf{#1}} %bold
\newcommand{\eq}[1]{\begin{align*}#1\end{align*}} %aligned equations
\newcommand{\cb}[1]{\centerline{\fbox{#1}}} %centered box
\newcommand{\notimplies}{% does not imply
  \mathrel{{\ooalign{\hidewidth$\not\phantom{=}$\hidewidth\cr$\implies$}}}}
\newcommand{\eq}[1]{\begin{align*}#1\end{align*}} %aligned equations
\newcommand{\piecewise}[4] %piecewise function
{
	\left\{
		\begin{array}{ll}
			#1 & \mbox{if } #2 \\
			#3 & \mbox{if } #4
		\end{array}
	\right.
}
\newcommand{\gray}[1]{\textcolor[gray]{0.5}{#1}} %gray text

\definecolor{indigo(dye)}{rgb}{0.0, 0.25, 0.42}
\color{indigo(dye)}

\begin{document}
\maketitle
\centerline{\LARGE \textbf{Coursera}}
\medskip

Two types: supervised and unsupervised.

Supervised is most commonly applicable, where we know the data we're getting.\\
comes in two flavors:\\
classification based: yes/no values\\
regression based: continuim of values.\\

Unsupervised: model is not provided with the correct results during training; key is to find structure or clusters in data.\\

\subsection*{Gradient Descent}

technique to find \textbf{ local } minimum of function by taking small steps in steepest direction.\\
Take $f(x,y)$ with step size (learning rate) $\alpha$. \\ Then,
update $x, y$ (simultaneously) using \\
$$\text{ new } x = x - \alpha \frac{d}{dx} f(x, y) * f(x, y)$$

if $\alpha$ is too large, technique may not yield optimal value; too small, technique will be slow.\\

\subsection*{Linear Regression using gradient}

can use gradient to find the least squares, minimize the squared error (called cost function).\\

in practice useful to rescale variables btw -1 and 1

*the squared error function supposedly only has one global min, so technique above succeeds at minimizing the error.

\textbf{normal equation } is an analytical method to find same result, but does not scale as nicely as gradient method.

\newpage

\centerline{\LARGE \textbf{Machine Learning}}
Robert Snapp\\

\subsection*{Intro}
AI - rational machines (ML is often a subset)\\

Machine Learning improves performance as more data is fed into Algo.\\

\subsection*{Types of Machine Learning}
\textbf{Supervised Learning} learn a mapping from input to output based on training set.\\
\begin{itemize}
    \item Classification: mapping inputs to \textit{discrete} outputs (categories)
    e.g., classifying a document as a web page or email.
    \item Regression: mapping inputs to \textit{continuous} outputs.
\end{itemize}
\textbf{Unsupervised Learning} only output given with goal of finding "interesting patterns" (called "Knowledge Discovery").\\
\begin{itemize}
    \item clustering
    \item Dimensionality Reduction. e.g., PCA
    \item Imputation ("matrix completion"): finding plausible values for missing data points.
\end{itemize}

*K-nearest neighbors (nearest K points to a given input) is a non-parametric classifier model.\\
\textbf{Reinforcement Learning} learning behavior based on reward/punishment signals.

\section*{Probability Review}

\subsection*{Axioms}
 Formulated by Andrey Kolmogorov\\
 \ \\
a \textbf{probability space} is comprised of :\\
\ \\
$\Omega$: sample space, meaning the set of possible elementary events.\\
\indent e.g., $\Omega = \{1, 2, ..., 6\}$\\
$\mathcal{A}$: all measurable sets of events\\
\indent *can be something like odd numbers if we're rolling a die.\\
$\mathcal{P}$: probability\\


$\mathcal{A}$ must satisfy:\\
1. $\Omega \in A$\\
2. $E \in A \iff E^c \in A$ \\
\indent *complement in $\Omega$\\
3. $E, F \in A \rightarrow E \cup F \in A$\\

$\mathcal{A}$ is a "sigma-algebra", $\sigma$-algebra \\
(simply properties/laws for operations on sets)\\

$\mathcal{P}$ assigns a numerical value for each $E \in A$ such that
natural requirements:\\
$\mathcal{P}$(E) $ \geq 0$, $\mathcal{P}$($\Omega$) = 1, and
if $E, F \in A$ and $E \cap F = \emptyset$, then \\
$$\mathcal{P}(E \cup F) = \mathcal{P}(E) + \mathcal{P}(F)$$
\ \\

$(\Omega, \mathcal{A}, \mathcal{P})$ defines a \textbf{probability space}.\\

\noindent two underlying assumptions: \\
pattern indicates a unique state of nature,\\
problem is driven by an underlying probability space.\\

\subsection*{Distributions}
\noindent a \textbf{random variable} map from  $\Omega$ to a "space of interest"?

\noindent a \textbf{probability mass function}  of a random variable, $X$, is
(where $k$ is in "space of interest" produced by $x$)
$$P_x(k) = \mathcal{P} \{ \omega \in \Omega : x(\omega) = k\}$$

e.g., $X$ is the result of summing two dice. \\
(probability mass is like a probability density function,but for a discrete variable)\\

Then $P_x(7)$ the sum of the probability of all inputs (aka $\omega \in \Omega$) summing to 7.


The \textbf{cumulative probability distribution} (CDF) of a random variable $X$,
$$F_X(r) = \mathcal{P} \{ \omega \in \Omega : X(\omega) \leq r\}$$
"cumulative density function"

Instantaneous probability, aka \textbf{probability density} is the derivative of the \textbf{probability distribution}.

\subsection*{Dice Example}
Two dice: \\
$\Omega = \{(i, j) : i, j \in \{1,..., 6\}\}$.\\
\indent $|\Omega| = 36$\\
$\mathcal{A} =$ \textbf{power set} (all possible subsets) of $\Omega$.\\
\indent $|\mathcal{A}| = 2^{36}$ (each member of $\Omega$ has two states on/off)\\

Define a random variable $X(i,j) = \max(i, j)$. Then, \\
$P_x(1) = \mathcal{P} \{(i, j): X(i, j) = 1\} = \frac{1}{36}.$\\
$P_x(2) = \mathcal{P} \{(i, j): \max(i, j) = 2\} = \mathcal{P} \{(2, 1), (1, 2), (2, 2)\} = \frac{3}{36}.$\\

For the rest, think about $L$ shape in outcomes table.\\

Notation: $[ x = 1] =\mathcal{P} \{(i, j): X(i, j) = 1\}$.


\subsection*{Pattern Classification}

Feature Space: observable trait like diameter of cell\\

\subsection*{Joint Distributions}
\subsubsection*{Discrete}
Two random variables: \\
$X_1$: max of rolling two dice\\
$X_2$: min.\\

$P_{x_1, x_2} (x_1, x_2) = \mathcal{P} \{\omega \in (i, j): X_1(\omega) = x_1 \text{ and } X_2(\omega) = x_2\}$.\\

Then,
 \begin{displaymath}
    P_{x_1, x_2}(x_1, x_2) = \left\{
        \begin{array}{lr}
    P(x_1, x_2) + P(x_2, x_1) & \text{if } x_2 < x_1 \\
    P(x_1, x_2) & \text{ if } x_1 = x_2
    \end{array}
    \right.
\end{displaymath}



\subsubsection*{Continuous}

CDF: all variables $\leq$ number.\\

Let CDF = $F_{x_1, x_2}(x_1, x_2)$. \\
Then,
PDF = $\displaystyle \frac{d^2}{dx_1dy_1}$\\
(partial with respect to $x_1$ then differentiate with respect to $y_1$; order doesn't matter).

\subsubsection*{Combined: Discrete and Continuous}

CDF as expected with discrete condition for discrete and similarly for continuous.\\

PDF: same as above with partial dervitives for contious and piecewise defined function for each discrete class.

\subsection*{Baye's Theorem}
In a probability space, $(\Omega, \mathcal{A}, \mathcal{P})$. \\
For E, F events in $A$,
$$P(E | F) = \frac{P(E \cap F)}{P(F)}$$
(provided $P(F) \neq 0$).\\
*no independence necessary.\\

\noindent \textbf{Why is $E \cap F$ in $\mathcal{A}$?}\\
\textcolor[gray]{0.5}{
We know $E^c$ and $F^c$ in $A$. \\
So, $E^c \cup F^c$ in $A$. \\
Thus, so is $E \cap F$ by De Morgan's Law.\\
}

\textbf{Baye's Rule}
$$P(F | E) = \frac{P(E | F) P(F)}{P(E)}$$

\noindent e.g., \\
ebola = E\\
H = high temp\\

$P(E | H) \approx 1$\\
$P(H|E)$ low, say $10^{-3}$\\

can compute probability of ebola given fever.

\subsection*{HW}
Snapp's paper before generating functions (4.4)

\subsection*{Independence}

Events $E, F$ are \textbf{independent events} if:
$$P(E \cap F) = P(E) P(F)$$

$E_1, E_2, ..., E_n \in \mathcal{A}$ are \textbf{independent events}:\\
For every combination of $E_i$, P(intersection) = P(first) *P(second) ...\\
"intersection distributes over multiplication"\\


Random variables are \textbf{independent} if for $F$ CDF:
$$F_{x_1, ... x_n}(x_1, ...., x_n) = F_{x_1}(x_1) F_{x_2}(x_2) ...F_{x_n}(x_n)$$


\section*{Pattern Classification}
In $(\Omega, \mathcal{A}, P)$, the process is: \\
1. seed from $\omega \in \Omega$ (elementary event\\
2. Feature Vector (observable trait/s)\\
3. L classification: maps $\omega$ to a state (like cancer no cancer)\\


Feature vector: $(x_1, ..., x_n)$.\\

\textbf{Probability Distribution (combined discrete and continuous)}\\
\centerline{$F_{x,l}(x, l) = P(\omega \in \Omega : X \leq x \text{ and } L = l)$, joint probability}\\

Using Baye's Theorem we have, 
$$F(x | l) = \frac{F(x, l)}{P(l)}.$$

e.g., say we have l = 0 is sick, 1 healthy; x is diameter of cell.\\
$P(x | l = 0):$ normal distribution centered around c\\
$P(x | l = 1):$ normal distribution centered around -c.\\

note: 
\begin{align*}
p(x) & = p(x, l=0) + p(x, l=1) \\
& = p(x | l=0) p(l=0) + p(x | l=1)p(l=1)
\end{align*}
What's the probability someone is sick based on x?\\

$$P( l = 1 | x = c_0) = \frac{p(x = c_0 | l = 1)p(l = 1)}{p(x=c_0)}$$

use expanded formula of normal curve to solve.

To classify a patient, compute propability of l=1 given x and l=0 given x. Higher probability leads to classification (called \textbf{Baye's Classifier})

\subsection*{Denisties from the above}

\textbf{Joint Density}\\
$P(x, l) = \frac{d}{dx} F_{x, L} (x, l)$\\
from joint denisty, we can derive all other info like conditional, or prior probability for class l.\\
\ \\
\textbf{prior class probability}\\
(class probability density) \\
$P(l) = \lim_{x \rightarrow \infty} F_{x, L} (x, l)$\\


$P(l |x)$ \bt{posterior probability}

\textbf{Marginal Denisty }
$$p(x) = \sum_{l=0}^1 p(x, l)$$

\textbf{Class Conditional Distribution}\\
$$F(x | l) = \frac{P(X \leq x, L = l)}{P(L = l)}$$
probability of $x$ falling within range given a value for $l$.

Using Baye's Theorem, we can reformulate as
$$P(l | x) = \frac{P(x, l)}{P(x)}$$

look up Baye's Classifier

\subsubsection*{Deriving other densities from join density}
prior probability for class l
$$P(l) = \int_{-\infty}^\infty p(x, l) dx$$

$$P(x | l) = \frac{ P(x, l)}{p(l)}$$

\subsection*{Probability of misclassifying}

Suppose $R_0$ is a region of $X$ the feature vector associated with $L =0$. 
Similarly, $R_1$ is associated with $L=1$.
Then, 
$$P(\text{error}) = P(x \in R_0, L=1) + P(x \in R_1, L-0) $$
$$= \int_{R_0} p(x, 1) dx + \int_{R_1} p(x, 0) dx$$

we can related the limits of integration via 
$$P(1) = \int_{-\infty}^\infty p(x, 1) dx = \int_{R_0} p(x, 1) dx + \int_{R_1} p(x, 1) dx$$

So, 
$$P(\text{error} = P(1) + \int_{R_1} P(x, 0) - P(x, 1) dx$$




\section*{Spam Email Classifier: Naive Baye's}

Trick: $p^x ( 1 - p)^{1-x} $\\
if x, the feature vector is 1, brings in $p^x$; else just $1-p$.\\


From data we try to estimate $p(x, l)$, enabling us to compute
$$p(l | x) = \frac{p(x, l)}{p(x)}$$


For multidimensional $x_1, \dots, x_n$ feature vectors: 
$$p(l | x) = p(l | x_1) * \dots p(l | x_n)$$
by \bt{assuming each feature is independent} (that's the big assumption!).


\section*{K-nearest Neighbor Classifier}

Need to select : 
\begin{enumerate}
    \item k 
    \item metric (similarity function)\\
    e.g., euclidean distance
    \item Training set
\end{enumerate}

Idea is to find the $k$ elements form the training data that are closest to an input $x$.

Then, classify input $x$ based on the class label of the nearest $k$ elements appearing most.
\subsection*{Discussion}

Main assumption: proximity in feature space corresponds to similarity in class. \\

very popular algo

\bt{Trade-offs}: Large $k$ reduces variances (more samples), but makes classification less accurate.\\

*note method is \bt{non-parametric} (doesn't assume an underlying distribution).\\

*useful rule of thumb: $k$ scales with $\sqrt{\text{ sample size }}$\\



Cover and Hart showed the error rate of K-nearest neighbors is no more than 2*Baye's error (missclassification based on conditional) rate for infinitely large samples.

\section*{Triangle Problem}

\begin{displaymath}
   p(x | 0) = \left\{
     \begin{array}{lr}
       2x &  0 \leq x \leq 1 \\
       0 & \text{ otherwise}
     \end{array}
   \right.
\end{displaymath} 


\begin{displaymath}
   p(x | 1) = \left\{
     \begin{array}{lr}
       2 - 2x &  0 \leq x \leq 1 \\
       0 & \text{ otherwise}
     \end{array}
   \right.
\end{displaymath} 


Looks like triangle on graph.

\textcolor{red}{Cover and Hart, 1967 mathematical error in paper}

Theorem Stone 1997: \\
sample size and nearest neighbor $\rightarrow \infty$, \\
\centerline{k-neighbor error approaches baye's error}

\part*{Regression}

\section*{Linear}
$y = \beta_0 + \beta_1 x$

we want find $\beta_0, \beta_1$ such that 
\eq{
\text{ squared error = } \sum_{i=1}^n (\beta_0 + \beta_1 x_i - y_i)^2
}
is \bt{minimized.}

\subsection*{Optimizing slope, intercept}

Look at partial derivatives: 
\eq{
&\frac{\partial}{\partial \beta_0} = 2 \sum_{i=1}^n (\beta_0 + \beta_1 x_i - y_i) = 0\\
&\frac{\partial}{\partial \beta_1} = 2 \sum_{i=1}^n (\beta_0 + \beta_1 x_i - y_i)x_i = 0
}
then solve this system of equations.


\begin{bmatrix}
 \beta_0 \\
 \beta_1 
\end{bmatrix}
=  
\begin{bmatrix}
n & \sum x \\
\sum x & \sum x^2 
\end{bmatrix}^{-1} 
\begin{bmatrix}
    \sum y\\
    \sum xy
\end{bmatrix}. 

\subsection*{Multivariable Case}
$B = (X^T X)^-1 X^T y$
where all are vectors.


\subsection*{Linear Algebra Aside}
\bt{gradient vector}, denoted $\vec\nabla f$, is a vector, 
$$<\frac{\partial}{\partial x}, \frac{\partial}{\partial y}, \dots>$$


a \bt{hyperplane} is an equation describing an \bt{n-1 dimensional object} in n-dimensional \bt{space} dividing the space into \bt{two pieces}.\\

e.g., a point in 1-d,  a line in 2-d, a plane in 3-d.

\section*{Fittting via any function}

write as $Y = B * f(X)$, where $f(x)$ is the function we're using to fit\\

same results apply.\\
Solution: $B = (X^T X)^{-1} X^T y$ (y is point in n-space)\\
\ \\

e.g., \bt{cubic fit}: $y = b_0 + b_1 x + b_2 x^2 + b_3 x^3$\\
we optimize by \\
letting $x1, x_2, x_3, \dots$ data points.\\
and $f_1(x) = x, f_2(x) = x^2, f_3(x) =x^3$.\\
\ \\
Then, 
$X =$ 
\begin{bmatrix}
    1 & f_1(x_1) = x_1 & f_2(x_1) = x_1^2 & x_1^3 \\
    1 & x_2 & x_2^2 & x_2^3 \\
    \dots \\
    1 & x_n & x_n^2 & x_n^3
\end{bmatrix}
\ \\

Finally solve $B = (X^T X)^{-1} X^T y$. 

\section*{Variations}

\bt{Local Regression} increase weight of points near a point you're intested in predicting.

score * 1.25 

\section*{Exam Solutions}

\begin{enumerate}
    \item 
 \begin{displaymath}
   p(x, l) = \left\{
     \begin{array}{lr}
       \alpha / \beta & \text{ if } 0 \leq x \leq \beta \text{ and } l = 1\\
       (1 - \alpha) /\beta & if $1 - \beta \leq x$ \text{ and } l = 2 \\
       0 & \text{ otherwise}
     \end{array}
   \right.
   \end{displaymath}
   with $\alpha, \beta \in [0, 1]$ and $L = 1$ or 2.\\
   (a) got it; can also use b*h (base times height)\\
   (b) Compute the posterior probability.
   We want to find 
   $$P(l | x).$$
   By Baye's, 
   $$P(l | x) = \frac{p(x, l)}{p(x)}$$

   First let's compute $p(x)$. 
   \eq{ p(x) = p(x, 1) + p(x, 2)}
   Two cases: \\
   1. $\beta > 1/2$: \\
 \begin{displaymath}
     p(x) = \left\{
     \begin{array}{lr}
        \alpha / \beta & :  0 \leq x \leq 1 - \beta \\
        \frac{1 - \alpha}{\beta} + \alpha / \beta  + & : 1 - \beta \leq x \leq \beta \\
       \frac{1 - \alpha}{\beta} & : \beta \leq x \leq 1
        
     \end{array}
   \right.
   \end{displaymath}
        
   2. $\beta < 1/2$: 

 \begin{displaymath}
     p(x) = \left\{
     \begin{array}{lr}
        \alpha / \beta & :  0 \leq x \leq \beta \\
       \frac{1 - \alpha}{\beta} & : 1 - \beta \leq x \leq 1
     \end{array}
   \right.
   \end{displaymath}

\end{displaymath} 


\end{enumerate}
 ---

3. 
\textcolor{red}{review gaussian, including second moment}

\section*{Perceptron Algorithm}

\bt{Linearly separable} classification problem: (we can draw a line to seperate groups)

\subsection*{Vectors Review}
Recall: \\
\bt{line} in $\R^2$: ax + by = c \\
alternately written as \\
\begin{bmatrix}
a & b 
\end{bmatrix}
\begin{bmatrix}
x \\
y
\end{bmatrix}\\
\bt{plane} in $\R^3$: ax + by + cz = d\\
\begin{bmatrix}
a & b & c 
\end{bmatrix}
\begin{bmatrix}
x \\
y \\
z
\end{bmatrix}\\

In general \bt{n-dimensions}: \\
(suppose vectors are all column vectors)\\
$W^T = [w_1, w_2, w_3, \dots, w_n]$
and
$X =$
\begin{bmatrix}
x_1 \\
x_2 \\
\dots \\
x_n
\end{bmatrix}

\noindent Then, the equation of a \bt{plane in n-dimensions} is $W^T X = d$\\

\noindent \bt{dot} product of ($W X) = ||W|| ||X||$ cos $\theta$, where $\theta$ angle between W, X.\\
(also just sum of $x_1w_1 + x_2 w_2 + \dots x_n w_n$)


Therefore, 
$$W^T X = \text{dot prod} = ||W|| ||X|| \cos \theta$$

\subsection*{Phrasing the problem}

Given W and d; then the set of solutions $x$ forms a plane in n-space. \\
(think about "projections" of $W$ (by varying length and angle))

Let $g(x) = W^Tx - d$, called "discriminant". 

Our goal is to find values for $W$ so that 
\eq{g(x) > 0 & \text{ if x belongs to 1st class}\\
g(x) < 0 & \text{ otherwise}.
}

If we can find this, it defines a line (or plane) potentially seprating our dataset).\\
\includegraphics{discriminant}

To the right side of blue line is one class to the left is another.

Suppose data is \bt{linearly seperable} then, we can find $W$ and $d$ to solve.\\

Choose classes $l_1 = 1, l_2 = -1$.\\

Idea (with $\alpha$ learning rate): 

\fbox{
  \parbox{\textwidth}{
    if $w*x < 1$ and l = 1: \\
     -- update: w = w + x $(\alpha)$ \\
    if $w*x > -1$ and l = -1: \\
     -- update: w = w - x * $(\alpha)$\\
  }
}


The pseudo-code for algo:

\begin{verbatim}
initialize W, d at some values 

    for every element in training set: 
        if not correclty classified:
            W += li Xi 
            d += li
            (restart loop again)
            #called batch process since loop over entire set again

#idea: W tweaked down or up (depending on l = 1 or -1)
#turns out tweaking leads to convergence (if linearly seperable)
\end{verbatim}

If data is separable and we find one boundary separating the data, we can always tilt the boundary a bit to find another boundary (actually infinitely many boundaries exist).

\subsection*{Why does tweaking always lead to convergence?}

idea: as you tweak, you move away from initial weights, W, and the \bt{cone of solutions} gets larger. Therefore, as we increment, we are 'guaranteed' to fall within cone.\\
to imagine cone of solutions: think of \bt{rays} (corresponding to boundaries) going out into space. The area they cover gets bigger.

\cb{\bt{Theorem} algo converges to a correct solution after finite number of weight corrections.}
called " Perceptron Convergence Theorem"

\gray{\href{http://www.cs.utsa.edu/~bylander/cs5233/perceptron-proof.pdf}{click for proof}}

proof from class: \\
\gray{
GOAL: Find weights $w_0, w_1, w_2$ such that the discriminant function 
\eq{g(x_1, x_2) = w_0 + w_1 x_2 + w_2 x_2} divides the data set ($g(x) > 0$ if $l = 1, g(x) < 0 $ if $l =1-1)$. \\
Define \bt{normalized augmented} feature vector: \\
$\hat{x}' = l \hat{x} = $
\begin{bmatrix}
l \\
l x
\end{bmatrix}\\
Let $w[k]$ be the adjusted weights at the kth step.\\
Then, 
\eq{w[1] = w[0] + n\hat{x}'(1) \\
w[2] = w[1] + n \hat{x}'(2) \\
}
meaning
\eq{w[k] - w[0] = n(x[1] + \dots + x[k])}
}
\gray{
Therefore, 
\eq{w^* (w[k] - w[0] ) = n w^* (x(1) + \dots + x(k)) }
where $w^*$ is the hypothetical solution.\\
}
\gray{
We want to show 
\eq{AK^2 \leq || w[k] - w[0] ||^2 \leq BK}
for some constants $A, B$.
The thing sandwiched is the error think; area between parabola and line.\\
(means converges after a max of $k = B/A$ updates)\\
\ \\
By Cauchy Schwarz we have the left inequality: 
\eq{[w^* ( w[k] - w[0])]^2 \geq ||w^*||^2 ||w[k] - w[0]||^2}\\
\hline
\ \\
Next, to bound above: \\
first, square the equations for $w[1], w[2], etc.$.\\
next, sum up all the equations to get lots of cancellation:
\eq{||w[k] - w[0]||^2  &\leq n^2 (||x'(1)||^2 + ||x'(2)||^2 + \dots + ||x'(k)||^2)\\
&-2nw(0^T(x'(2) + \dots + x'(k))\\
}
Next let $M = \max\{ \text{ all } ||x'||^2\}$, then 
\eq{ \leq n^2 M -n \text{ stuff })K}
To simplify further we take $u =$ min of stuff: 
\eq{ \leq (n^2 M - n u ) K }
et voila.\\
*note: u is always negative, since u > 0 implies no data was misclassified!
}

\subsection*{Variation: augmented feature vector}
make each feature vector x sit in n+1 dimension by adding 1 as a component. Then, the algebra cleans up and we get 
$$||W|| ||X||$$
This makes updating quicker, since both values are part of W.

\subsection*{Another Variation: Augmented Normalized Feature Vector}
agument the feature vector and normalize values.\\

To normalize: multiply by 1 if in class 1 and -1 if class 2 (so flips sign)\\

The \bt{augmented normalized} feature vector is denoted: $\hat{x}'$ \\
Idea: make an error for either class look the same.



\section*{LMS: Least Mean Squares}
part of \bt{linear regression}\\
\bt{Cost Function} is the sum of squares (goal is to minimize this)\\
also called \bt{objective function}

We have linear function
\eq{y = w_0 + w_1 x_1 + w_2x_2}
where output y, depends on two variables with weights $w_1, w_2$.

Goal is to use tweak $w_i$ to minimize \bt{cost function}.\\

Use Calculus to minimize. 

Set gradient (partial derivatives with respect to each variable $x_i$). \\
Then, the optimal solution (after some algebra) is 
\eq{optimal = (X^TX)^{-1} X^T Y}
where $Y$ are the set of outputs for the training data.\\

Often although the above is analytically convenient, it can consume quite a bit of memory. Other methods such as \bt{steepest descent} (also called gradient descent) are used, because their implementation is \bt{more pratical} in terms of memory and processing power required.

\subsection*{Steepest (Gradient) Descent}
We use \bt{steepest descent} (also called gradient descent) to do so.
Gradient descent works by updating $w_i$ using
\eq{\text{new } w_i = w_i - \alpha \frac{\partial}{\partial w_i} * \text{cost function}}
where $\alpha$ is the learning rate.\\

Stop when gradient (partials) are zero.
\subsection*{multivariable chain rule (aside)}

\eq{\frac{d}{dt} f(x(t), y(t)) = \frac{\partial}{\partial x} \frac{dx}{dt} + \frac{\partial}{\partial y} \frac{dy}{dt}}





\end{document}
