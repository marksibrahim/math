
\documentclass[12pt]{article}

\usepackage{amsmath, amssymb, color, enumitem, fullpage}
% \fbox{\parbox{0.8\linewidth}{
\title{Abstract Algebra II}
\author{Mark}
\date{}

%word spacing
\usepackage{microtype}


%useful shortcuts
\def\R{\ensuremath{\mathbb{R}}} %\ensuremath adds math mode, if forgotten
\def\Q{\ensuremath{\mathbb{Q}}}
\def\N{\ensuremath{\mathbb{N}}}
\def\Z{\ensuremath{\mathbb{Z}}}

%shorcuts with arguments
\newcommand{\abs}[1]{\left\vert#1\right\vert} %nice absolute values
\newcommand{\bt}[1]{\textbf{#1}} %bold
\newcommand{\eq}[1]{\begin{align*}#1\end{align*}} %aligned equations
\newcommand{\cb}[1]{\centerline{\fbox{#1}}} %centered box
\newcommand{\bp}[1]{\fbox{\parbox{0.8\textwidth}{#1}}} %box paragraph
\newcommand{\notimplies}{% does not imply
  \mathrel{{\ooalign{\hidewidth$\not\phantom{=}$\hidewidth\cr$\implies$}}}}
\newcommand{\gray}[1]{\textcolor[gray]{0.5}{#1}} %gray text

\newcommand{\sn}[1]{\reversemarginpar\marginpar{\gray{\large #1}}} %side bar cornell
%\newcommand{\bt}[1]{\textbf{#1}}

\definecolor{indigo(dye)}{rgb}{0.0, 0.25, 0.42}
\color{indigo(dye)}

\begin{document}
\maketitle

\section*{Chapter 7, recall}
Ring is an \textbf{abelian} additive group with multiplication that's \textbf{associative} and \textbf{closed}, linked by \textbf{distribution}.\\
\ \\
\noindent\textbf{Division Ring}: a ring with mult inverses \\
\textbf{Zero Divisor}: if there is another element so product is zero\\
\textbf{Integral Domain}: commutative ring with unit and no zero divisors.\\

\ \\
\textbf{Ideal}: subring $I$ such that $ir$ and $ri \in I$ for all $r$ in ring.\\
For $I, J$ ideals, 
$$IJ = \text{ set of finite sums of ij}$$
\ \\
\textbf{principle ideal}: ideal generated by one element using + and *\\
\textbf{maximal ideal}: ideal not contained in any other proper ideal.\\
\textbf{prime ideal}: $ab$ in ideal, then $a$ or $b$ is.\\

\centerline{\fbox{
$P$ is prime $\iff R/P$ is an integral domain}
}
\textcolor{red}{see notes for proof.}


\centerline{\fbox{$M$ is maximal $\iff$ $R / M$ is a field}}\\
\textcolor[gray]{0.5}{$R / M$ is field means no ideals; by lattice iso, no ideals between R and M.}

\fbox{\parbox{0.8\textwidth}{
\centerline{maximal ideal $\rightarrow$ prime ideal}
\textcolor[gray]{0.5}{
\begin{center}
max ideal $\rightarrow R/P$ is a field. \\
field is an integral domain.
\end{center}}
}}

\subsection*{Quadratic Fields and integer rings}

Define a \textbf{quadratic field} as 
$$\mathbb{Q}(\sqrt D) = \{ a + b \sqrt D : a, b \in \mathbb{Q}\}$$
for $D \in \mathbb{Q}$ and not divisible by a perfect square ('square-free').

\textcolor[gray]{0.5}{can show this is a field using usual checks.\\
Inverses * involves a trick : \\
key: $(a + b \sqrt D)(a - b \sqrt D) = a^2 - Db^2$\\
If $a$ and $b$ are not both zero, then, $a^2 - Db^2$ can't be zero.\\
(this would imply $D = \frac{a}{b}^2$, a contradiction)
Thus, $$(a + b \sqrt D)\frac{a - b \sqrt D}{a^2 - Db^2} = 1$$
}

For $D = -1$, $\mathbb{Z}[D]$ is $\mathbb{Z}[i]$, the set $a + bi$ with integer coefficients, called the \textbf{Gaussian Integers}.\\
\medskip

\noindent Define the \textbf{quadratic ring of integers}, $\Theta_D$, in the quadratic field $\mathbb{Q}(\sqrt D)$ as\\
 \begin{displaymath}
      \left\{
     \begin{array}{lr}
     \mathbb{Z}[\sqrt D], & \text{ if } D = 2, 3 \text{ mod } 4\\
     \mathbb{Z}[\frac{1 + \sqrt D}{2}], & \text{ if } D = 1 \text{ mod } 4\\
     \end{array}
     \right.
 \end{displaymath} 
 *note both $\mathbb{Z}[\sqrt D]$ and $\mathbb{Z}[\frac{1 + \sqrt D}{2}]$ are subrings of the field $\mathbb{Q}(\sqrt D)$.\\
\medskip

The \textbf{field norm} $N$ is a function from $\mathbb{Q}(\sqrt D) \rightarrow \mathbb{Q}$ defined 
$$N(a + b \sqrt D) = (a + b \sqrt D)(a - b \sqrt D) = a^2 - D b^2$$
(as noted above norm is never zero if both $a$ and $b$ are not zero)\\

For the ring of integers ("quadratic integer rings"), the norm is more genrally defined as
$$N(a + b\omega) = (a + b\omega)(a + b \bar{\omega}) $$

where $\bar\omega$ is the Galois conjugate (-) is attached to $\sqrt D$.\\

\fbox{\parbox{0.8\textwidth}{Norm is \textbf{multiplicative}: $N(\alpha \beta) = N(\alpha)N(\beta)$.}}

\ \\

a number in $\mathbb{Q}[\sqrt D]$ is an \textbf{algebraic integer} if it's the root of a monic polynomial with integer coefficients.\\

    $\alpha$ is a unit implies there exists $\beta$ such that $\alpha \beta = 1$.\\

\fbox{\parbox{0.8\textwidth}{An element $\alpha$ in the ring of integers is a \textbf{unit} if and only if $N(\alpha) = \pm 1$.}}

\textcolor[gray]{0.5}{proof: 
$\rightarrow)$ Suppose $\alpha$ is a unit. Then, 
$$\alpha \beta = 1, \text{      for some } \beta \in \Theta_D$$
So, $N(\alpha \beta) = N(1) = 1$.\\
$\leftarrow)$ $\alpha \bar \alpha = \pm 1$, so either $\alpha \bar \alpha = 1$ or $\alpha (- \bar \alpha) = 1$.\\ }

e.g., for $\mathbb{Z}[i]$ (aka $D = -1$), the units are $\{\pm 1, \pm i\}$ as they are the only option satisfying $a^2 + b^2 = 1$.\\

For rings, $A, B, AB = \{a_1b_1 + a_2b_2 + a_3b_2 + ...\}$ "finite sums of elements"\\

\bt{Ideal generated} by a subset of $R$, A is denoted $(A)$. \\
it's the "smallest ideal containing A"


\bt{Kernel} of a ring homomorphism is set of elements mapping to 0 (additive id).

\cb{Kernel of ring hom is an ideal}
\gray{ For $s \in Kernel$, any $r \in R$, consider 
$\varphi(sr)$ still maps to kernel. \\
hence ker is ideal}
%------------Exercises----------------------------------------------------------
\subsection*{Exercises}
\begin{enumerate}
    \item What are the units of $\Theta_{-3} = \mathbb{Z}[\frac{1 + \sqrt {-3}}{2}]$?\\
\textcolor[gray]{0.5}{$\alpha = a + b \frac{1 + \sqrt{-3}}{2}$ is a unit 
\iff\\
$N(\alpha) = a^2 + ab + b^2 = 1$ \\
TRICK: complete the square! \\
$(2a + b)^2 + 3b^2 = 4 $\\
only options for $b$ are $0, 1,$ or $-1$.\\
units are $\{\pm 1, \pm \frac{1}{2}, \pm \frac{\sqrt{-3}}{2}\}$.
}
    \item Prove $\mathbb{Z}[i]$ with $N(a+bi) = a^2 + b^2$ is a Euclidean Domain.\\
   \textcolor[gray]{0.5}{We need to show Division Algo works. \\
   For $\alpha, \beta \in \mathbb{Z}[i]$, ($\beta \neq 0$)\\
   $$\frac{\alpha}{\beta} = \frac{a+bi}{c+di} = \frac{a}{c^2+d^2} + \frac{bi}{c^2+d^2} = r + si \ \ (r, s\in \mathbb{Q})$$
   Let $p, q$ be the closest integers to $r,s$ in turn. Then, 
   $$N((r+si)-(p + qi)) = (r-p)^2 + (s-q)^2 \leq \frac{1}{2}$$
   Then, we define Algo as
   $$\alpha = (p + qi) \beta + R$$ 
   Remains to show $N(R) < N(\beta)$.  \\
   Define some other variable $\theta = (r - p) + (s - q)i$, with $N(\theta) < \frac{1}{4} + \frac{1}{4} = \frac{1}{2}$.\\
   Then, 
   $$N(R) = N(\theta)N(\beta) \leq \frac{1}{2} N(\beta)$$
   }
    \item Find the ideal generated by $(3-i, 2 + 11i)$.\\
\textcolor[gray]{0.5}{idea is to find the gcd using Euclidean Algo.\\
First, 
    $$\frac{2 + 11i}{3-i} = \frac{-1}{2} + \frac{7}{2} i.$$
Select closests integers $p = -1, q = 3$. Then remainder $R$ is
$$= 2 + 11i - (-1+3i)(3-i) = 2 + i.$$
We have
$$2 + 11i = (-1 + 3i)(3-i) + (2+i).$$
Next, 
$$\frac{3-i}{2+i} = (1-i) $$
Thus, 
$$3-i = (1-i)(2+i) + 0.$$
Meaning, the gcd = $2+i$ (last nonzero remainder).\\
Thus, ideal is $((2+i))$.
}
    \item Show $\mathbb{Z}[\sqrt{-5}]$ is not a Euclidean Domain.\\
\textcolor[gray]{0.5}{idea is to show it's not a PID (hence not a Euclidean Domain).\\
Consider $I = (2, 1 + \sqrt{-5})$.\\
Suppose $I$ is a principal ideal with generator $\alpha$.\\
Then $ 2 = k_1* \alpha$ and $1 + \sqrt{-5} = k_2 \alpha$.\\
Then, $N(\alpha)$ divides $4$ and divides $6$ $\rightarrow N(\alpha) = 1 \text{ or } 2.$ \\
Case 1: $N(\alpha) = 2$\\
Then, $2 = a^2 + 5 b^2$, which is impossible for $a, b \in \mathbb{Z}$.\\
Case 2: $N(\alpha) = 1$\\
...somehow contradiction
}

\item What are zero divisors of $/Z$? What are units?\\
\gray{no zero divisors; units are $\pm 1$. So $\Z^* \ \{\pm 1\}$}

\end{enumerate}


\section*{Chapter 8: Euclidean Domains}
\subsection*{Norm}

For $R$ an integral domain, a \textbf{norm} is a function $N$ such that
$$N: R \rightarrow \mathbb{Z}_{\geq 0} \text{ and } N(0) = 0$$

A norm is a measure of size in $R$.

e.g., $R = F[x]$, norm is generally the degree of the polynomial.\\

*possible for same integral domain to have more than one norm. Often, statements are with respect to a particular norm.\\

\subsection*{Eucliean Domain}

A \textbf{Euclidean Domain} is an integral domain, $R$, with a division algorithm such that for any two elements $a, b \in R (b \neq 0)$,\\
there exists $q, r \in R$ where 
$$a = qb + r \text{ and } r = 0 \text{ or } N(r) < N(b)$$

$q$ is the \textbf{quotient} and $r$ is the \textbf{remainder}.\\

e.g., fields (with any norm), $\mathbb{Z}$ with $N(a) = |a|$, $F[x]$ with norm = degree of polynomial.  \\

\centerline{\fbox{\parbox{0.8\textwidth}{Every ideal in a Euclidean Domain is \textbf{principal}}}}\\
\textcolor[gray]{0.5}{proof: consider $d$ in an ideal $I$, such that $d$ has minimum norm in $I$. (exists by Well ordering principal)
\begin{enumerate}
    \item[(1): ] $(d) \subset I$, by closure.
    \item[(2): ] $I \subset (d)$, since for $a \in I$, \\
    $$a = qd + r$$
    with $N(r) < N(a)$ (impossible) or $r = 0$. Thus, $a \in (d)$.
\end{enumerate}
}

*useful to show NOT Euclidean Domain, if some ideal is not principal.\\


A Euclidean Domain allows for the use of the \textbf{Euclidean Algorithm}.\\

\centerline{If $(a, b)$ (ideal generated by a, b) = $(d)$, then $d = gcd(a, b)$}\\
\centerline{\indent\textcolor[gray]{0.5}{because $d = ax + by$ by Euclidean Algo.}}

\subsection*{Principal Ideal Domains, PIDs}

A \textbf{ Principal Ideal Domain} is an integral domain where every ideal is principal.

\centerline{ Euclidean Domain $\rightarrow$ PID}
\centerline{\textcolor[gray]{0.5}{since every ideal in Euclidean Domain is principal}}
\ \\

\centerline{\fbox{In a PID, irreducible $\rightarrow$ prime.}}
\textcolor[gray]{0.5}{\noindent For r irreducible, wwts (r) is a prime ideal.\\
Suppose $(r) \subset (m)$. \\
Then r = am for some a, then a is a unit or m is a unit since r is irreducible.
$a$ unit $\rightarrow (r) = (m)$\\
$m$ unit $\rightarrow (m) =$ entire ring.
}

\subsection*{Unique Factorization Domains, UFDs}

For $R$ an integral domain, 
\begin{itemize}
    \item $r \in R$ is \textbf{irreducible} if whenever, $r=ab$ ($a, b \in R$), $a$ or $b$ is a unit. (otherwise, $r$ is \textbf{reducible})
    \item $p \in R$ ($\neq 0$), is \textbf{prime} if $(p)$ is a prime ideal. \\
    i.e., normal notion of prime $p | ab$, $p | a$ or $p | b$ (aka a or b in ideal).
    \item $a, b \in R$ are \textbf{associate} if $a = ub$ for some unit $u \in R$.
\end{itemize}

\centerline{\fbox{prime element $\rightarrow$ irreducible}}\\
\textcolor[gray]{0.5}{p, prime. 
If $ab = p \in (p)$, then $a \in (p)$ or $b \in (p)$.\\
Next, show $a$ or $b$ is a unit. \\
Note without loss of generality, $p = ab = prb \rightarrow rb = 1$, meaning b is a unit. \\
}

irreducible $\not \rightarrow$ prime: e.g., $2 | (1 + \sqrt{-5})(1- \sqrt{-5})$, but $2$ does not divide $1+\sqrt{5}$.\\

\centerline{\fbox{ In PID, prime element $\iff$ irreducible}}\\
\textcolor[gray]{0.5}{above proves $\rightarrow)$. \\
$\leftarrow)$ see previous page.
}



A \textbf{Unique Factorization Domain} is an integral domain in which for every $r \neq 0$ and not a unit:\\
(1) r can be written as a \textbf{finite product} of irreducible elements. \\
    *not necessarily distinct.\\
(2) above is \textbf{unique} up to associates.\\
    *any factorization has same number of products and elements are associates with elements of composition in (1).\\

easiest example is any field, since every element in a field is a unit (hence nothing to verify in order be a UFD).\\

examples of UFDs: $\mathbb{Z}$, $\mathbb{F}[x]$, $\mathbb{Z}[i]$.\\
\textcolor[gray]{0.5}{$\mathbb{Z}[i]$ showed it's a Euclidean Domain $\rightarrow$ PID $\rightarrow$ UFD. Similar proof for $F[x]$.}

$\mathbb{Z}[x]$ is a UFD, but not a PID.\\

$\mathbb{Z}[\sqrt{-5}]$ is not a UFD\\
\textcolor[gray]{0.5}{$6 = 2 *3$, but also  $6 = (1 + \sqrt{-5})(1-\sqrt{-5})$. Each produc is made of irreducible terms.}

\subsection*{Ascending Chain Condition (ACC), Notherian}

A commutative ring with unit $R$ is \textbf{Noetherian} if it satisfies ACC: \\
\fbox{\parbox{0.8\textwidth}{every increasing sequence of ideas: \\
$$I_1 \subset I_2 \subset ... $$
terminates, eventually.}}

\ \\
Equivalent to say: 
\begin{enumerate}
    \item ACC
    \item every nonempty collection of ideals has a maximal element
    \item every ideal is finitely generated
\end{enumerate}
\textcolor[gray]{0.5}{proof\\
$(1 \rightarrow 2)$ Suppose $A$ is any nonempty collection of ideals.\\
If no maximal ideal $I_n$ existed in the collection, we can construct an infinite chain, hence not ACC, a contradiction.\\
\ \\
$(2 \rightarrow 3)$ Let $A$ be a nonempty collection of ideals with a maximal element, say $I_0$.\\
Thus, for $I_i$ in chain: \\
(a) $I_0 \subset I_i$, since $I_0$ is maximal.\\
(b) $I_i \subset I_0$, since \textcolor{red}{unclear}
$(3 \rightarrow 1)$ Suppose $I_1 \subset I_2 \subset I_3 \subset ...$ is a chain of ideals\\
Then, $\displaystyle \union_{i}^\infty I_i$ is an ideal, say $I$.\\
Since, every ideal is finitely generated, so is $I$, meaning the chain terminates.
}

e.g., $\mathbb{Z}[x_1, ..., x_n, ...]$ is Netherian. \\
\textcolor[gray]{0.5}{$(x_1) \subset (x_1, x_2) \subset$ ... infinite.}

\centerline{\fbox{PID is Netherian\\}}
\textcolor[gray]{0.5}{since every ideal is generated by 1 element, by (3) above, PID is Netherian.}

\fbox{\centerline{ PID $\rightarrow$ UFD}}

\textcolor[gray]{0.5}{proof: \\
IDEA: factor such as integers. \\
Suppose $r \neq 0$ in $R$, a PID.\\
Then we can factor $r$ as $r_1*r_2....$. \\
Suppose a branch of the factorization continued, then we'd have a chain: \\
$(r_1) \subset (r_2) \subset (r_3)$\\
along the branch, which contradicts ACC (since PID has ACC).\\
Is this product unique?\\
\ \\
Suppose $r = p_1 p_2 ..p_n = q_1 q_2 ..q_m$. \\
Then, $p_1 | q_1 q_2 ... q_m$, hence, $p_1 | q_i$ for some $i$.\\
Thus, $q_i = p_1 k$, but $q_i$ is irreducible, hence $p_1, q_i$ are associates.\\
Next repeat for $p_2$, to show $n=m$ and all are associates.
}

\centerline{In UFD, irreducible $\iff$ prime.} \\
\textcolor[gray]{0.5}{proof: 
$\rightarrow)$ $P$ is irreducible.
If $P |ab$ with $a = p_1 .. p_n, b = p_1'...p_m'$, then $P$ is associate to some $p_i$, hence divides $a$ or $b$. \\
$\leftarrow)$ true in general.}

Field is a ED, is a PID, is a UFD, is an Integral Domain.\\
(nicest to less)\\

\subsection*{GCD}

An \textbf{ideal is a gcd} d of $a, b$ if \\
(1) If $(a) \subset (d)$ and $(b) \subset (d)$ implies $(a, b) \subset (d)$.\\
(2) If $(a, b) \subset (c)$, then $(d) \subset (c)$.\\
"gcd(a, b) is a generator for smallest principal ideal containing (a, b)"\\

*(2) is a bit counterintuitive, careful.\\

gcds exist in UFD\\
\textcolor[gray]{0.5}{gcd(a, b) = min power of primes in $a, b$}

In PID, $(a, b) = (d)$. (exists since PID is UFD).\\
(but no Euclidean Algo!)\\

*gcd is not always a linear combo if not in PID.\\
e.g., $\mathbb{Z}[x]$ (UFD not PID) \\
a = 2, b = x: gcd(2, x) = 1 \\
but $1 \neq 2s + xt$ for any $s, t$.\\

Euclidean domain for gcd, is even better: linear combo and algo (euclidean) to find it!\\
\subsection*{Davenport-Hasse Norm}

$R$ has a \textbf{Davenport-Hasse} norm $N$ ifj: \\
For $a, b \neq 0$, $a \in (b)$ or $N(as+bt) < N(b)$ for some $s, t$.

e.g., Euclidean Domain has a Davenport-Hasse norm, since $N(R) = N(a -qb) < N(b)$.

\subsection*{Arithmetic, applying gcd}

Recall, for integer rings $\theta_D$: PID $\iff$ UFD\\
D $< 0$: almost never a PID\\
D $> 0$: unknown when they're PID. 

$\theta_D$ has \textbf{no unique factorization} for elements; it does have unique factorization for ideals. (every ideal can be written as a product of prime ideals)\\


\fbox{\parbox{0.8\textwidth}{
$I = (a_1 , ..., a_n) = r_1 a_1 + ... + r_n a_n$ (r in ring)\\
"linear combos of generator elements"}}\\
For $J = (b_1, ..., b_m)$.\\
\centerline{$IJ = ra_1b_1 + .... + ra_1b_m$ \\}
\centerline{$+ ra_2 b_1 + ... + r_a_2 b_m$}\\
\ \\


e.g., $R = \mathbb{Z}[\sqrt{-5}]$, recall not PID.\\
$P = (2, 1 + \sqrt{-5})$ we showed was not principal, BUT \\
$P^2 = (2, 1 + \sqrt{-5})(2, 1 + \sqrt{-5})$\\
$ = ( 4, 2 + 2\sqrt{-5}, -4 + 2\sqrt{-5})$ by def of $P*P$ as linear combos \\
$= (4, 2, 2\sqrt{-5}) = (2).$\\



\subsection*{Irreducibles in Integer Rings}

For $\pi$ in integer ring, $\theta_D$, \\
\centerline{\fbox{If $N(\pi) = p$ (for $p$ prime in $\mathbb{Z}$), $\pi$ is \textbf{irreducible}}}
\textcolor[gray]{0.5}{Suppose $N(\pi) =p$.\\
Then for $\pi = ab$, $p = N(a)N(b)$\\
$\rightarrow N(a) =1 \text{ or } N(b) =1$, meaning $a$ or $b$ is a unit.}\\

What are the irreducible elements in $\mathbb{Z}[i]$?\\
look at $p \in \mathbb{Z}$ and see how they factor in $\mathbb{Z}[i]$.
\textcolor{red}{read and take notes on end of section 8.3}\\



\centerline{\fbox{$p$ factors in $\mathbb{Z}[i]$ into two irreducibles $ \iff p = a^2 + b^2$ for $a, b \in \mathbb{Z}$}}
\textcolor[gray]{0.5}{idea is to think about norm of elements factoring $p$}


Use tool from Number Theory: \\
\centerline{prime $p \in \mathbb{Z}$ divides $n^2 + 1 \iff$ $p \cong 1 mod 4$ or $p =2$}
\textcolor[gray]{0.5}{look at elements of order 4 in $\mathbb{Z}/p\mathbb{Z}$ \textcolor{red}{look at again}}

\subsubsection*{Fermat's Sum of Squares}
\centerline{\fbox{$p = a^2 + b^2 \iff p \cong 1 mod 4$ or $p=2$}}
 \ \\
Furthermore, the sum of squares representation is \textbf{unique} up to sign changes.

\subsubsection*{What are irreducibles in $\mathbb{Z}[i]?$}

$1 + i$, $p \cong 3 mod 4$ for prime in $\mathbb{Z}$, and \\
$a \pm bi$ which form $p = a^2 + b^2$ for $p \cong 1 mod 4$ (p prime)
\textcolor{red}{reread 8.3 end to understand proof}\\


For $n = 2^k p_1^{a_1} p_2^{a_2}...p_r^{a_r}q_1^{b_1}...q_s^{b_s}$, \\
\fbox{\parbox{1.2\textwidth}{
\centerline{if p, q are distinct primes with}\\
\centerline{$p_i \cong 1 \mod 4$ and $q_i \cong 4 \mod 4$, then $n$ can be written as the sum of squares}
}}
\ \\
*the number of representations of $n$ as a sum of squares is\\ $4(a_1 + 1)(a_2 + 1)...(a_r + 1)$.
\textcolor[gray]{0.5}{proof at end of 8.3}



%--------------------------------------------------------------------
\subsection*{Exercises}
\begin{enumerate}
    \item Show $\mathbb{Z}[2i]$ is not a UFD\\
\textcolor[gray]{0.5}{find an element written as product of different irreducibles.\\
$4 = 2*2 = 2i(-2i)$, all irreducible.}
    \item Is p = $(2, 1 + \sqrt{-5})$ a prime ideal in $\mathbb{Z}[\sqrt{-5}]$?\\
\textcolor[gray]{0.5}{consider quotient $\mathbb{Z}[\sqrt{-5}] / (2, 1 + \sqrt{-5}$.\\
Is it an integral domain? \\
note in quotient, $\overline{1 + \sqrt{-5}} = \bar 0 \rightarrow \overline{\sqrt{-5}} = \overline{-1}.$\\
Thus, $\overline{a + b \sqrt{-5}} = \overline{a - b}$.\\
So, $\mathbb{Z}[\sqrt{-5}] / (2, 1 + \sqrt{-5}) \cong \mathbb{Z} / (2)$ ( by previous work).
Thus, it is an integral domain.
}
\end{enumerate}

\ \\
\ \\
%--------------------------------------------------------------------
\part*{Chapter 9: Polynomial Rings}

\textbf{Constructing $\mathbb{Q}$ from $\mathbb{Z}$}\\
set: $(a, b)$ with $a, b \in \mathbb{Z}$\\
equivalence: $(a, b) \equiv (c, d) \iff ad - bc = 0$
can confirm operations are well-defined as expected (based on representatives from equivalence class)\\

\subsection*{R UFD}
\textbf{Gauss's Lemma}\\
\fbox{\centerline{p(x) reducible in $F[x] \implies$ p(x) reducible in $R[x]$}}
\textcolor[gray]{0.5}{proof idea: use unique factorization in UFD}\\

\centerline{$p(x)$ is irreducible in $R[x]$ $\iff$ it is irreducible in $F[x]$}
\gray{Part by Gauss's other by looking at gcd of coefficients of $p(x)$.}\\

\centerline{$R$ is UFD $\iff$ $R[x]$ is UFD}\\
\textcolor[gray]{0.5}{proof from notes and in section 8.2}

\subsection*{Rational Roots Test}

If polynomial with integer coefficients has a \bt{rational root} $r/s$, $r$ divides constant term and $s$ divides leading coefficient.\\
\gray{think about factoring}\\

\subsection*{Smaller Fields}
For $I$ ideal, \\
\fbox{\centerline{If the image of $p(x)$ is \bt{irreducible} in $R/I [x]$, then it's \bt{irreducible} in $R[x]$}}

*careful: reducible in modulo doesn't imply reducible in ring.\\

Take away: \cb{$p(x)$ is irreducible in say $\Z / p \Z$ $\implies p(x)$ irred in $Z[x]$}.


\centerline{\bt{content} of $p(x) \in R[x],$ UFD = gcd of coefficients, "ideal generated by coef"}

\subsection*{Roots of Polynomials}

\bt{degree n} polynomial has \bt{n roots} in F, a field. \\
*not true in rings: $x^2 -1$ in $\Z / 8 \Z[x]$ has only 4 roots.

\subsection*{Eisentein's Irreducibility Criterion}
$p(x)$ is \bt{irreducible} in $\Z[x]$ if \\
\centerline{there is $p$, \bt{prime dividing} all coefficients, but $p^2$ \bt{doesn't divide} constant}\\

More generally true for \bt{integral domain} $R$: $p(x)$ irreducible in $R[x]$ if\\ coefficients are elements of prime ideal $P$, but constant is not element of $P^2$.


\cb{Tip: $f(x)$ is irreducible $\iff$ f(x+1) is }

%------------Exercises----------------------------------------------------------
\section*{Exercises}
\begin{enumerate}
    \item Show $x^3 - 3x -1$ is irreducible in $\Z[x]$. \\
    \gray{Since any rational root has to divide $1$, the only candidates for roots are $\pm 1$.\\
    Neither is a root. \\
    So, polynomial is irreducible.}

    \item Is $x^3 + 5x -17$ irreducible in $\Z[x]$?\\
    \gray{check mod 2: $x^3 + x + 1$, which has no roots so irreducible!}


\end{enumerate}

%------------Chapter 10----------------------------------------------------------
\part*{Chapter 10: Modules}

\section*{Linear Algebra Revisted}

 \bt{Linear Transformation} is a homomorphism $\varphi: V \rightarrow W$ both vector spaces:\\
 1) $\varphi(V + W) = \varphi(V) + \varphi(W)$\\
 2) $\varphi(\alpha V) = \alpha \varphi(V)$\\

$T: V \rightarrow W$ a linear transformation can be \bt{written as a matrix}: \\
\centerline{$M_{b}^\epsilon$ where $b$ is a basis of $V$ and $\epsilon$ is a basis of $W$. }\\

*note T depends on the basis chosen for V and W. \\
 \ \\


\bt{Big Theorem}: every vector space has a basis.\\
(same number of elements as dimensionality of vector space)\\
\ \\

Ker(T) = null space

\section*{Module}

An R-module $M$ is an \bt{abelian group} with $R$, a ring, acting on $M$ by: \\
1) $r(m+n) = rm + rn$\\
2) $(r+s) m = rm + sm$\\
3) $(rs)m = r(sm)$\\

\noindent *if $R$ has unit, then additional requirement: 1m = m.

Examples
\begin{itemize}
    \item F-module is a vector space over $F$ 
    \item $\Z$-module is an abelian group
    \item $F[x]$-module is a vectorspace $V$ over field $F$ with a linear transformation
\end{itemize}


\section*{Quotient Modules}
For any $N, M$ R-modules with $N \subseteq M$, 
$$M / N \text{ is a quotient}$$

"all quotients are submodules"\\
why? 
\textcolor[gray]{0.5}{
1. $N \trianglelefteq M$ since $N$ is abelian. so "+" makes sense\\
2. $r(m + N) = rm + N$ just need to check it's well-defined.
}

\section*{Generators}
Idea in general "blah" \bt{generated by} $m_1, m_2, ..., m_n$ means the \bt{smallest "blah" stucture} containing all $m_i$.\\

\bt{Sub-module} generated by $m_1, m_2, ..., m_n \in M$ is 
$$R m_1, + ... + Rm_n$$

\textcolor[gray]{0.5}{since closure is over addition and scalar mult, both captured above by linear combo}

\bt{cyclic} R-module if generated by a single element.

\section*{submodule}
N is a sub-module of a module $M$ if for $n_1, n_2 \in N$, 
\begin{enumerate}
    \item $n_1 + n_2 \in N $ "closure +"
    \item $r m_1 \in N$ "scalar closure"
\end{enumerate}


\section*{Homomorphisms}
$\varphi: M \rightarrow N$

\bt{R-module homomorphism} $\varphi$ is what you'd expect:\\ $\varphi(x+y) = \varphi(x) + \varphi(y)$ and $\varphi(rx) = r \varphi(x)$.


Ker$(\varphi)$ and Image($\varphi$) are both submodules (in $M$, $N$ in turn)\\

\subsection*{Isomorphism Theorems}
\begin{itemize}
    \item $M / $ Ker$\varphi \cong $ Image($\varphi$)
    \item $A + B / B \cong A / A \cap B$
    \item $(M / N) / (M' / N) \cong M / M'$   \ \ \ for $N \subseteq M' \subseteq M$
\end{itemize}

and lattice bien sur. 

\section*{Cyclic Modules}
a \bt{module} $M$ is \bt{cyclic} means there exists $m \in M$  such that $R*m = M$.\\
\ \\
an element $a$ of $M$ is \bt{torsion-free} if $r a \neq 0$ for any $r \in R$.\\
(a is \bt{torsion} element if there is some r such that $ra = 0$)\\

*torsion module implies every element is a torsion element.\\

\subsection*{Natural Map for Cyclic Modules (over PIDs)}
$\varphi: R \rightarrow M$ by $r \rightarrow r m$ where $m$ is generator.\\

Ker($\varphi$) = left ideal in $R$, call it $I$. 
Then, 
$$R / I \cong M$$
by first iso theorem. 
\gray{idea: $M = R*m$, so it's isomorphic to left cosets of $R$: $R / (r)$.}

*idea: nicer ring, yields nicer r-module $M$.

\bt{"annihilator"} of $M$ in $R = \{r \in R : r m = 0 \text{ for all } m\}$ 

\section*{Properties of Determinants}
\begin{itemize}
    \item $det(I_n) = 1$
    \item $det(A^T) = det(A)$
    \item $det(A^{-1}) = det(A)^{-1}$ ( i.e., 1/ det(A))
    \item $det(AB) = det(A)det(B)$ (= det(BA)) "commutative"
    \item $det(cA) = c^n det(A)$ for c a constant
    \item for $A$ triangular (upper or lower right entires all zero), \\
    \eq{det(A) = \text{ product of diagonal entries}}
\end{itemize}

\subsection*{Find determinant using cofactors}
What's det(A)?
\centerline{
\begin{bmatrix}
    2 & -1 & 1 & 0 \\
    3 & 5 & 0 & -2 \\
    1 & 1 & 0 & -3 \\
    4 & 0 & 3 & -1 
\end{bmatrix}
}

Easiest to go down the third column (b/c of the zeros): \\
det(A) = 
$1^{1+3} * 1$ det(
\begin{bmatrix}
    3 & 5 & -2 \\
    1 & 1 & -3 \\
    4 & 0 & -1
\end{bmatrix}
)+0 + 0 + $(-1)^{4 + 3} * 3$ det(
\begin{bmatrix}
    2 & -1 & 0 \\
    3 & 5 & -2 \\
    1 & 1 & -3
\end{bmatrix}
) = -50 + 99 = 49.


%==============================Chapter 12 =========================
\part*{Chapter 12: Modules over PID}

A $\Z$-module is an \bt{abelian group}.\\
Thus, \\
\centerline{$\Z$-module = $\Z \oplus \Z / n \Z \oplus \Z / n \Z \dots$}
\gray{by FTFGAG}

Any $n \Z$ is an \bt{ideal} of $\Z$ (the ideals are precisely $n \Z$).\\

The idea is to \bt{generalize} the above by replacing $\Z$ with any PID, R.\\


\subsection*{in context of linear algebra}
for vectors space $V$  with a linear transformation $A$, we find a \bt{different basis}.\\
This means we find $B$ such that $B = P^{-1} A P$ for some matrix $P$.\\
This  allows us to write transformation in \bt{unique forms}: 
\begin{itemize}
    \item \bt{Jordan Canonical} Form: as close to a diagonal matrix as possible
    \begin{itemize}
        \item requires eigenvalues to be in field $F$.
    \end{itemize}
    \item \bt{Rational Canonical Form}: similar but doesn't require eigenvalues to be in $F$.
\end{itemize}


\section{Fundamental Theorem of Finitely Generated Modules}
(FTFGM) over PIDs \\
recall, PID = integral domain (commutative ring with unit and no zero divisors) where every ideal is principal. \\

For, \\
R = PID\\
M = finitely generated R-module. \\

There are two ways of \bt{uniqely} decomposing a finitely generated module $M$:\\
\bp{
\begin{enumerate}
    \item Invariant Factor way:\\
\eq{M = \underbrace{R \oplus \dots \oplus R }_{\text{rank r}} \ \ \oplus \ \ \underbrace{R / (r_1) \oplus \dots \oplus R / (r_n)}_{invariant factors}}
where $r_1 | r_2 | \dots |r_n$.
    \item Elementary Divisor way: \\
\eq{M = \underbrace{R \oplus \dots \oplus R }_{\text{rank r}} \ \ \oplus \ \ \underbrace{R / (p^s_1) \oplus \dots \oplus R / (p^s_n)}_{elementary divisors}}
where $p_1, p_2 \dots$ are prime elements (not necessarily distinct).\\
\end{enumerate}
}\\
*recall: $p \in R$ is \bt{prime} if $(p)$ is a \bt{prime ideal} (ab in (p) $\implies $ a or b in $(p)$ );\\
implies traditional def: $p | ab \implies p | a \or p | b$.\\

*recall: $A \oplus B = \{(a, b) : a \in A, b \in B\}$.\\

\gray{proof after Chineses Remainder and Noetherian R-modules}\\

Note: Fundamental TFG Modules $\implies$ FTFGAG.\\

\subsection*{Chinese Remainder Theorem for $R$-modules}
Let $R$ be commutative with 1.\\
For $A, B$ \bt{comaximal} ideals in R, 
$$A \cap B = AB \text{ and } R / AB \cong R / A \oplus R / B$$

\noindent \bt{comaximal} means $A + B = R$ " sum gives entire ring."
\gray{proof in notes}

\subsection*{Noetherian R-Modules}
$M$ is a Noetherian R-Module  is equivalent to any of the following: 
\begin{itemize}
    \item $M$ satisfies ACC on R-submodules
    \item Every R-submodule is finitely generated
    \item Every collection of submodules has a maximal element
\end{itemize}(eerily similar to Noetherian ring)
 \ \\


\bp{
$M$ is Noetherian $\iff$ $M'' \text{ and } M' = M / M''$ are Noetherian \\
"submodules and quotients of Noetherian are Noetherian"}

\gray{proof}

\noindent \bt{Torsion}\\
For $R$ integral domain and $M$ an R-module, 
\eq{Tor(M) = \{x \in M : rx = 0 \text{ for some r } \in R\}}
*M is torsion free if Tor(M) = 0\\

\noindent \bt{Annihilator of M} is the ideal of R such that
\eq{Ann(N) = \{ r \in R : rn = 0 \text{ for all n} \in N\}}

\section{Rational Canonical Form}
\subsection*{Eigenvalue}
The \bt{eigenspace} of a linear transformation T is 
\eq{\{ v \in V : T(v) = Av = \lambda v\}}

The \bt{characteristic polynomial} of T, denoted $Ch_A(x)$ is det(xI - A).\\
\gray{often written A - xI, but above produces a nice monic polynomial.}\\
\bt{degree} n of $Ch_A(x)$ is the \bt{dimension of V}.\\
The set of eigenvalues is precisely the set of \bt{roots of the characterstic polynomial.} (at most n eigenvalues)

\subsection*{Minimal Polynomial}
The unique monic polynomial, $m_A(x)$, of \bt{smallest degree} such that $m_A(A) = 0$.\\
--can also think of $m_A(x)$ as \bt{generator of Ann(V)} in $F[x]$

The minimal polynomial is the \bt{largest invariant factor} (all invariant factors divide $m_A(x)$).

\subsection*{Characteristic Polynomial}
\begin{itemize}
    \item characteristic polyn is the \bt{product} of all invariant factors
    \item \bt{Cayley Hamilton}: min poly \bt{divides} char poly
    \item char divides some power of the min polyn (meaning char and min have same roots)
\end{itemize}

SAME Charactersitic polynomial is a \bt{necessary} but \bt{not sufficient} condition to conclude two matrices are similar (they need to have the same RCF or JCF)
\subsection*{Companion Matrix of a polynomial}
For any $a(x) \in F[x] = x^k + b_{k-1} x^{k-1} + b_{k-2} x^{k-2} + \dots + b_00$, \\
the \bt{companion matrix} of $a(x)$ is \\
\centerline{\begin{bmatrix}
    0 & 0 &\dots & -b_0 \\
    1 & 0 & \dots & -b_1 \\
    \dots \\
    0 & 0 & \dots 1 & -b_{k-1}
\end{bmatrix}}


\subsection*{Rational biz}
A matrix is in \bt{rational canonical form} if the companion matrices of some polynomials $a_1(x) | a_2(x) | \dots | a_m(x)$ form the matrix.\\
"$a_i(x)$ are the \bt{invariant factors}"\\

\centerline{RCF of any matrix is \bt{unique}}\\
\bt{Two matrices are similar} if and only if they have the \bt{same RCF}.\\

To find all similar matrices, consider different possible minimal polynomials and invariant factors.


\section{Jordan Canonical Form}

Jordan form is as \bt{close} as possible to a \bt{diagonal matrix} (often simpler matrix than rational form).

To obtain the JCF, we use the \bt{elementary divisor form} of the fundamental theorem.

Suppose for an $F[x]$-module of V with invariant factors $a_1(x) | a_2(x) | \dots | a_m(x)$, all monic polynomials.
Then, the \bt{elementary divisors} are powers of $(x - \lambda)^k$ (under the assumption the field F contains all eigenvalues of A).

The k x k \bt{elementary Jordan matrix} with eigenvalue $\lambda$ is\\
\begin{bmatrix}
\lambda & 1 & 0 & \dots & 0 \\
0 & \lambda & 1 & 0 \dots & 0 \\
\dots\\
0 & 0 & 0 & \dots & 1 \\
0 & 0 & 0 & \dots & \lambda\\
\end{bmatrix}

\bt{Jordan Canonical Form} is a block diagonal matrix (square matrices along diagonal, zero elsewhere) with Jordan Blocks (above) along the diagonal.

\bt{unique} up to permuting the Jordan Blocks. 

\bt{Theorem} if $A$ contains all eigenvalues, then $A$ is \bt{similar to} a matrix in Jordan Canonical Form (JCF = $P^{-1}AP$ for some P).\\
\cb{A similar to JCF $\iff$ $m_A(x)$ has no repeated roots}

\cb{dim of eigenspace = \# invariant factors = \# Jordan Blocks}
\cb{A can be \bt{diagonalized} $\iff$ Jordan blocks are of size 1 }
equivalent ot $m_A(x)$ having distinct roots


For matrices of size 2 or 3, knowing $m_A(x)$ and $ch_A(x)$ determines JCF.\\

For larger matrices (say 4x4), we can use \bt{rank} to computer Jordan Blocks: \\
\cb{\# Jordan Blocks of size k = $r_{k-1} - 2r_k + r_{k+1}$}
where $r = $ rank of matrix computed as rank $(A - I)^k$ (which is the number of linearly independent rows/columns.

\gray{
e.g., number of Jordan blocks of size 2 = $r_1 - 2r_2 + r_3$ (compute $(A-I)^0 = I$ has rank 4 (for A 4x4), $(A-I)^1$ see $\#$ of independent rows ..)}


\subsection*{computing JCF}

\begin{enumerate}
    \item Put charactersitic polynomial into form: $(x-\lambda)^k$ \\
    (if can't, we won't be able to put into JCF)
    \item size of JB for $\lamba$ = dim null space (A - $\lambda I$)?
\end{enumerate}

Linear fact \cb{dim null space + dim column space (rank, = row space) = n}

%====================Exercises================================================
\section*{Exercises}
\begin{enumerate}
    \item What are the submodules of $\R[x]$ for $V = \R^2$ and $T:$ rotation (counterclockwise) by $\pi / 4$? \\
\textcolor[gray]{0.5}{
possibilities are
dimension 0: point at center\\
dimension 1: lines thorugh the center \\
dimension 2: entire plane\\
\ \\
dimension 1 is not closed when rotating a point by $\pi / 4$. \\
So, 0 and whole thing are only submodules.
}
    \item How about for $S:$ rotation by $\pi$?\\
\textcolor[gray]{0.5}{0, lines through the origin, and whole plane}

    \item Is $M =$ set $\R^2$ with $T :$ rotation by $\pi$ inside $\R[t]$-module cyclic? 
    \gray{No, think polynomial $a + bt + ct^2$ acts by multiplication where $tv = T(v)$.\\
    Span of $v, tv, t^2v, t^3v$ doesn't yield all of $\R^2$.
    }

    \item How about with $T: $ rotation by $\pi / 4$?
    \gray{yes!, v = (1, 0, then $T^2(v) = (0, 1)$ which spans all of $\R^2$.}


    \item Show $A, B$ similar matrices have the same charactersitic polynomial. \\
    \gray{$ch(B) = det(xI - B) = det(xI - P^{-1}AP) = det(P^{-1}x P - P^{-1} AP)$\\
    $= det(P^{-1})det(x - A)det(P)$ = det(x - A).}

    \item Show the constant term in the characteristic polynomial of $A$ (nxn) is $(-1)^n det A$. 
    \gray{char poly = det(xI - A). The constant term is where $x=0$, so we have constant term = det(-A) = $(-1)^n det(A)$ (depending on n even or odd)}
    \item Show the coefficient of $A$ is the negative of trace(A).
    \gray{Note product of the diagonal: $(x-a_1)(x-a_2) \dots (x- a_n)$; \textcolor{red}{unclear}}
\end{enumerate}


%====================CHAPTER13================================================


\part*{Chapter 13: Field Extensions}

\bt{Goal}: if $a(x)$ has no roots in $F$, how do we enlarge $F$ so $a(x)$ has a root? \\

an element $c$ is \bt{algebraic} over $F$ if it is the root of some nonzero polynomial in $F[x]$.(else it's \bt{transcendental})\\

a field $K$ is \bt{algebraically closed} if every polynomial $f(x) \in K[x]$ has at least one root in $K$.

recall: $F[x]$ is a ring, a particularly nice ring: Euclidean Domain. \\
Thus, every ideal in $F[x]$ is principal 
\gray{since $F[x]$ is a Euclidean Domain, hence a PID}\\

\section*{Extensions as a Map over Polynomials}
Consider 
\eq{\varphi_c : F[x] \rightarrow F \text{ by } a(x) \rightarrow a(c)}

$\varphi_c$ is a homomorphism! \\
\begin{itemize}
    \item Ker($\varphi$) is an \bt{ideal}, so it must be principal
    \begin{itemize}
        \item it's generated by the minimum* polynomial of $c$
    \end{itemize}
    \item Image($\varphi$): turns out to be a field! 
    \begin{itemize}
        \item it's the \bt{smallest field} containing $F$ and $c$; call it $F(c)$.
        \item Image$(\varphi) \cong \frac{F[x]}{<m(x)>}$
    \end{itemize}
\end{itemize}

*the \bt{minimum polynomial} $p(x)$ of $c$ over $F$ is the polynomial of lowest degree in $F[x]$ such that $p(c) = 0$ (note by making $p(x)$ monic, we can ensure it's unique).

Any homomorphism $\varphi: F_1 \rightarrow F_2$ between fields is an \bt{isomorphism} (or 0 map).
\section*{Extensions as Vector Spaces}

For $F \subseteq K$, $K$ can be thought of as a vector space over $F$.\\

The degree of $K$ is denoted $[K: F]$.


It turns out $F(c) = span \{1, c, \dots, c^{n-1}\}$ where $n$ is the degree of the \bt{minimal polynomial}.\\
\gray{proof: take any $a(c) \in F(c)$, then $a(x) = q(x) m(x) + r(x)$. \\
Evaluate at $c$, then $a(c) = 0 + r(c)$ where $r(c)$ has degree < n.}


Furthermore,
\eq{[E:F] = [E:K][K:F]}\\
*when $[E:F] = n, [K:F]=m$ with $gcd(n, m) = 1$, $[EK:F] = nm$.\\
\ \\
\cb{Any polynomial of degree n in $F[x]$ has \bt{n roots} in an extension of F.}
\gray{proof idea: the extension is $\frac{F[x]}{(p(x))}$, where $p(x)$ is the irreducible polynomial in F. This extension is a field by the work above, where a root $c$ of $p(x)$ exists}

    If both $a, b$ are roots of some irreducible $p(x) \in F[x]$, then 
    \eq{F(a) = \frac{F[x]}{(p(x))} = F(b)}
    implying $a, b$ are \bt{algebraically indistinguishable!}

$\alpha$ \bt{algebraic} over F $\iff$ $F(\alpha) / F$ is finite degree extension.\\

$\alpha, \beta$ algebraic: carries over sums, products, division: $\alpha + \beta$ algebraic etc.\\

\bt{algebraic closure} of a filed, say $\Q$, denoted $\overline{\Q}$, is $\Q$ plus all algebraic elements in $\Q$.

\cb{Every element of a \bt{finite} field is algebraic}
\gray{Take $F \subseteq K$ and $c \in K$ (deg k =n), then $1, c, \dots, c^n$ is a linearly dependent set. \\
Thus, $a_0 + a_1 c + a_2 c^2 + \dots + a_nc^n = 0$ for some $a_i \in F$.
}

\cb{Every Finite extension is a \bt{simple} extension (adjoins one element)}


\subsection*{Characteristic of a field}
The smallest number $n$ such that $\underbrace{1 + \dots 1}_{n} = 0$\\
(else ch(F) = 0, if no finite n exists, e.g. $\Q$)

Note $ch(F)$ must be \bt{zero} or \bt{prime} 
\gray{(if not prime, then ab = 0, implying a=0 or b=0, a contradiction of requirement for ch to be smallest!)}\\
\centerline{so finite fileds must have \bt{prime order!}}

In characterstic $p$, 
\eq{(a + b)^p = a^p + b^p}
"against every inclination"


\section*{Splitting Fields}

For $f(x) \in F[x]$, $K$ is called a \bt{splitting field} for $f(x)$ if 
\begin{itemize}
    \item $f(x)$ has all its roots in $K$ ($f(x)$ splits into linar factors in $K$)
    \item it's the smallest such extensions (no subextension of $K$ contains all roots of $f(x)$)
\end{itemize}

a polynomial is called \bt{separable} if it has distinct roots in some splitting field. (if polynomial a repeated root, it's \bt{inseparable})\\

\cb{f(x) has distinct roots $\iff (f(x), d/dx f(x)) = 1$}
\gray{$\alpha$ is a root of $f'(x) \iff \apha$ is a multiple root of $f(x)$, thus minimal polynomial divides both f and $f'$, meaning gcd $\neq 1 \Box$}

\cb{ch(F) = 0 $\implies$ any irred p(x) has \bt{distinct} roots}
\gray{why? if $\alpha$ is a multiple root of $p(x)$, then $p'(x)$ would have the same root and be of lower degree!}

\cb{Splitting fields are unique}
\gray{proof: division algo and induction by looking at map between two splitting fields to show they're iso }

\bt{What's the degree of $K$ over $F$ ($[K:F]$)?}\\
\gray{
Suppose $\alpha_1, \dots, \alpha_n$ are the roots of $f(x)$. Then, $F(\alpha_1) / F \leq n$, $\F(\alpha_1, \alpha_2) / F \leq n-1$, etc. \\}
Since, degree of extensions are multiplicative, $[K:F] \leq n!$.\\

For $K_1, K_2$ extensions of $F$ of degrees n and m, 
\eq{[K_1K_2 : F] = nm \tag{if (n, m) = 1}}

If $p(x)$ irred in $F[x]$ has one root in a splitting field $K$, it must have \bt{all its roots} in $K$.




\subsection*{Roots of Unity}
The nth \bt{roots of unity} in a field are elements $a_i$ such that $a_i^n = 1$.
the roots of unity divide a unit circle into arcs of equal length.

$a$ is a \bt{primitive nth root} of unity if $n$ is the \bt{smallest} integer such that $a^n = 1$.

\subsection*{Cyclotomic Polynomials}
The nth \bt{cylotomic polynomial} is
\eq{\Phi_n(x) = \prod_{1 \leq k \leq n, (n, k) = 1} (x - e^{\frac{2i \pi k}{n} })}

For $n$ prime, 
\eq{\Phi_n(x) = 1 + x + x^2 + \dots + x^{n-1}}

For $n = 2p$, 
\eq{\Phi_{2p}(x) = 1 - x + x^2 - \dots + x^{p-1}}

The first few cyclotomic polynomials are
\eq{\Phi_1(x) = x-1 & \Phi_2(x) = x+1 & \Phi_3(x) = x^2 + x + 1\\
\Phi_4(x) = x^2 + 1 & \Phi_5(x) = x^4 + x^3 + x^2 + x +1 & \Phi_6(x) = x^2 -x + 1.
}


\cb{The degree of $\Phi_n(x) = \varphi(n)$}

The cyclotmoic polynomial $\Phi(n)$ is the \bt{minimal} polynomial of any nth root of unity $\zeta_n$.



\subsection*{Cyclotomic Fields}

Field obtained by joining any primitive nth roots of unity.\\
For any field F, $F(\zeta_n)$ is called a \bt{cylotomic field} for $\zeta_n$ the nth root of unity. The field is \bt{cylic} !




\section*{Exercises}
\begin{enumerate}
    \item What is $(1 + \sqrt[3]{2})^{-1}$ in $\Q(\sqrt[3]{2})$ ? \\
    \gray{1. min poly is p(x) = $x^3 - 2$, since $\sqrt[3]{2}$ is a root and p(x) is irreducible by Eisenstein. \\
    2. Thus, $\Q[x] / p(x) = \Q(\sqrt[3]{2}$. \\
    3. Inside field, $p(x)$ is zero.\\
    idea: use euclidean algo to find $a(x) (1+x) + b(x)(x^3-2) = 1$.\\
    evaluate $a(x)$ at $\alpha$ to find inverse of $(1+ \alpha)$, since right term goes to zero!}

    \item What is $[\Q(\sqrt[n]{2}) : \Q]$?\\
    \gray{min poly: $x^n-2$, so degree of extension is n.}

    \item What's the degree of the splitting field for $(x^2 -2)(x^2-3)$?\\
    \gray{It's the degree of $\Q(\sqrt{2}, \sqrt{3})$ over $\Q$, which by previous work is $4$.}

    \item What's the degree of $\Q(\sqrt[4]{2}, \sqrt{2})$? \\
    \gray{it equal to the degree of $\Q(\sqrt[4]{2})$}

    \item For $p(x) = x^3 + 9x + 6$ and $\theta$ a root, find $\frac{1}{1+\theta} \in \Q(\theta)$.\\
    \gray{since $p(x)$ is irreducible, we know 
    $$a(x) (1+x) + b(x)(x^3+9x+6) = 0$$
    by gcd. At $x = \theta$, $a(\theta)(1+\theta) + 0 = 0$, meaning $a(\theta)$ is the inverse we seek. \\
    To find $a(x)$ use Euclidean algo.}

    \item In general to find $\theta^{-1}$, consider factoring $\theta$ as in page 516 of Dummit.
\end{enumerate}

Strategies:
\begin{itemize}
    \item To find minimal polynomial, try multiplying out complex conjugates of root given (x - root)(x - complex conju of root). Hopefully polynomial is irreducible , done.
    \item Another way to determine degree is to take something like $\Q(\sqrt{3+2\sqrt{2}})$ show it equals a simpler field ($\Q(\sqrt{2})$ in this case) by squaring the element.
\end{itemize}

\section*{Field Automorphisms and Galois}

\eq{Aut(K/F) = \{\sigma \in Aut(K) | \sigma \text{ fixes } F\}}

Automorphisms of K/F only take \bt{roots to roots} of same poly.\\

\cb{$|Aut(K/F)| \leq [K:F]$}
(equality if the polynomial of the splitting field is \bt{separable})\\

A field extension $K$ is \bt{Galois} only if $|Aut(K/F)| = [K:F]$.

Since automorphisms of a splitting field $K$ of $p(x)$ permute roots of $p(x)$, \\
\centerline{$|Aut(K/F)| = [K:F]$ = degree of K over F}\\

e.g., 2 = $|Aut(\Q(\sqrt{2})/\Q)|$ since autmorphisms can take $\sqrt{2}$ to itself or $-\sqrt{2}$. \\

e.g., 4 = $|Aut(\Q(\sqrt{2}, \sqrt{3})/\Q)|$ similar argument with options for both $\sqrt{2}$ and $\sqrt{3}$.

\subsection*{Automorphisms as a Group: Galois Groups}

\centerline{Gal(K/F) = permutations of roots of $a(x)$= Aut(K/F)}
(not every permutation; just ones uniquely identifying an automorphism)

These automorphisms are a group, called the \bt{Galois Group}, under composition.

\subsection*{Fixed Fields}
For $K$ a splitting field of $p(x)$ over $F$, one-to-one between\\
\cb{ subgroups of Gal(K/F) $\iff$ subfields of $K$}

e.g., what's the splitting field of $x^p-2$ over $\Q$?

\gray{roots are $\sqrt[p]{2}, \zeta_p\sqrt[p]{2}, \dots \zeta_p^{p-1}\sqrt[p]{2}$
which are contained in $\Q{\sqrt[p]{2}, \zeta_p}$.
}

\section*{Review Galois Theory}

\begin{itemize}

\item For $p(x) \in F[x]$ \bt{irreducible}, then $p(x)$ has a root in some extension of $F$.

\item For $\alpha$ a \bt{root} of $p(x)$, $F(\alpha) \cong \frac{F[x]}{(p(x))}$.

\begin{itemize}
    \item if $\alpha, \beta$ roots of $p(x)$ $F(\beta) \cong F(\alpha)$ "algebracailly indistinguishable"
\end{itemize}

\item $[F(\alpha):F] $ = degree of minimal polynomial

\item a homomorphism between fields is an isomorphism or 0.

\item \bt{quadratic extensions} equivalent to adjoining $\sqrt{D}$ (square-free)

\item $K_1K_2$, \bt{composite extension}, is the smallest field containing $K_1, K_2$.

\begin{itemize}
    \item $[K_1:F]=n, [K_2:F]=m$, then $[K_1K_2:F] \leq nm$ (= if (n, m)=1).
\end{itemize}

\item a field $K$ is \bt{algebraically closed} if every poly in $K[x]$ has a root in $K$. (this in fact means every root of any poly is in $K$ by factoring argument)

\item the \bt{splitting field} $K$ of a polynomial is the smallest field containing all its roots.
    \begin{itemize}
        \item for $K$ splitting field, $[K:F] \leq n!$
    \end{itemize}

\item $f(x) \in F[x]$ is \bt{separable} $\iff (f(x), f'(x)) = 1$.

\item In $ch(F)=0$ or for $F$ finite field, 
    \begin{itemize}
        \item $p(x)$ irreducible $\implies p(x)$ is separable.
        \item $p(x)$ is separable $\iff$ $p(x)$ is the product of distinct irreducibles.
    \end{itemize}
\end{itemize}

Questions:
\begin{itemize}
    \item Does $\zeta_n = e^{2\pi i / n}$ ?
\end{itemize}


\end{document}
