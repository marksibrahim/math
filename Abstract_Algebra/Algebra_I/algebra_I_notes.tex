

\documentclass[12pt]{article}

\usepackage{amsmath, amssymb, color, enumitem}
% \fbox{\parbox{0.8\linewidth}{
\title{Final Review}
\author{Mark}
\date{}

\definecolor{indigo(dye)}{rgb}{0.0, 0.25, 0.42}
\color{indigo(dye)}

\begin{document}
\maketitle
\centerline{\LARGE \textbf{Groups}}
\medskip


\centerline{Tools}
$\varphi(n)$, \textbf{Euler's Phi Function}, "totient function," is the number of natural numbers $a \leq n$ such that $(a, n) = 1$\\

\centerline{\fbox{$$\varphi(ab) = \varphi(a) \varphi(b) \text{ if } (a, b) =1$$}}
\ \\
\centerline{\fbox{$$\varphi(p^a) = p^a - p^{a-1} \text{ for prime p}$$}}

\textcolor[gray]{0.5}{
$p^a$ candidates \\
$\frac{p^a}{p} = p^{a-1}$ are divisible by $p$ \\
(10/2 yield \# numbers $\leq 10$ divisble by 2)} 
\medskip

example: $\varphi(15) = \varphi(5) \varphi(3) = 4*3 = 12$, since gcd(3, 5) = 1. \\

\textbf{Fun}: \textbf{last digit} of a number is the remainder when dividing by 10, aka mod 10
(for last three digits use mod 1000)\\

To compute \textbf{$a^{-1}$ mod n}: 
 For gcd(a, n) = 1, use euclidean algo, to find ax + ny = 1, meaning ax = 1 mod n. Thus, x is $a^-1$



\section*{Dihedral $D_{2n}$, $Z_n$ and $S_n$}
\subsection*{$D_{2n}$}
The \textbf{elements} of $D_{2n}$ are $\{1, r,...,r^{n-1}, s, sr, ..., sr^{n-1}\}$.\\

\textbf{Orders} of elements are $|s| = |sr^i| = 2$ and $|r| = n$

\textbf{Property}: $r^i s = sr^{-i}$\\

$D_{2n}$ can be described by  
$$\langle r, s: r^n = 1 = s^2, rs = sr^{-1} \rangle$$

Which elements commute with all of $D_{2n}$? (aka center)\\

\[Z(D_{2n}) = \left\{
  \begin{array}{lr}
    {1} & \text{for $n$ odd} \\
    {1, r^{n/2} &  \text{for $n$ even}
  \end{array}
\right.
\]


\subsection*{Symmetric Groups, $S_n$}

Permutations of $\{1, 2, 3, ..., n\}$\\
\indent each \textbf{uniquely} written as a product of disjoint cycles \\
\textcolor[gray]{0.5}{idea of proof is to write any permutation $(a_1, ..., a_n)$ by closing once a loop is reached; this is guaranteed to happen since permutation is a bijective function!}\\
\textbf{order}, $|S_n| = n!$

\begin{itemize}
    \item \textbf{cycle order} is the \textbf{lcm} lengths of disjoint unique cycles
    \item an element has \textbf{order p}, prime, in $S_n$ if and only if its cycle decomposition if a \textbf{product of p-cyles}.
    \textcolor[gray]{0.5}{from homework, section 1.3 problem 14}
\end{itemize}
*least common multiple is found by writing down the multiples of two numbers and finding the first common match.\\

can think of any permutation in $S_n$ as acting on polynomial:
$$S_4 : (x_1 - x_2)(x_1 - x_3)(x_1 - x_4)(x_2-x_3)(x_2-x_4)(x_3 - x_4)$$

$(1 2 3)$ sends $x_1$ to $x_2$ and so on; then, rewrite so each subscript $i < j$.\\

If, (-1) is present, the permutation is \textbf{odd}. \\
\indent \textbf{Sign} of $(12)$ is -1; sign of $(123)$ is 1.\\

\subsubsection*{Conjugating in $S_n$}
$\sigma = (12)(345)(6789)$ and $\tau = (1357)(2468)$, \\
$$\tau \sigma \tau^{-1} = (\tau(1) \tau(2) ) (\tau(3) ...) = (3\ 4) (5...$$

hence two elements are conjugates only if they have the same cycle type.\\

Any element of $S_n$ can be written as a \textbf{product of 2-cycles} (not uniquely!!)
Whether the number of transpositions is odd or even, though, is unique!

\subsubsection*{Alternating, $A_n$}

set of even permutations 

$$|A_n| = \frac{n!}{2}$$


$A_n$ is simple for all $n \geq 5$.\\

$A_n$ is the kernel of $\varphi: S_n \rightarrow \{\pm1\}$ \\

$S_5$ is not solvable (by looking at composition series)\\

\noindent \textbf{Curiosities} \\
Fermat's Little Theorem: $a^p \equiv a$ mod p
\subsection*{Cyclic, $Z_n$}

$Z_3 = \{0, 1, 2 \}$ is a group with $+$ mod 3\\

\noindent What are the \textbf{generators}?
\begin{mdframed}
find a generator $<a>$, \\
\centerline{all others are $<a^i>$ such that $(i, n) = 1$}
\end{mdframed}
example: $Z_8 = <1> = <3> = <5> = <7>$\\

\noindent For $|a| = n$, \\
$$|a^k| = \frac{n}{(n,k)}$$

\noindent What are \textbf{subgroups} of $Z_n$?
\begin{mdframed}
subgroups are $<a^i>$ for each $i$ a divisor of $n$
\end{mdframed}
\textcolor[gray]{0.5}{else, $i$ is relatively prime, hence generates entire group}\\

\noindent How many elements of order $k$ in $Z_n$?\\
\ \\
\centerline{$\varphi(k)$, if $k | n$}

\textcolor[gray]{0.5}{elements of order $k$ generate a cyclic group of order $k$.\\ Thus, the number of generators is $\varphi(k)$.}


\subsection*{Isomorphisms}

To show two groups are not isomorphic consider: 
\begin{itemize}
    \item abelian?
    \item elements have the same orders
\end{itemize}

\noindent For $\varphi$ a homomorphism,\\
\centerline{ Ker $\varphi = 1 \iff \varphi$ is \textbf{injective} }
\ \\

Curious isomorphic groups: $D_6 \cong S_4$\\

Precisely 2 groups of order 4: $V_4$ and $Z_4$



\subsection*{Exercises}
\begin{enumerate}
    \item How many 3-cycles in $S_4$?
\textcolor[gray]{0.5}{
        $$(a\ b\ c)$$
count options for $a * b * c = 4*3*2$, which can be written in 3 ways. \\
So, total is $\frac{4*3*2}{3} = 8$}


    \item Elements of order 8 in $Z_{8,000,000}$ ?\\
\textcolor[gray]{0.5}{there are only 4 elements, since $\varphi(8) = 4$\\
Note, $<1 m>$ is an element of order 8. \\
Thus, the others $<1 m>$ raised to 3, 5, 7 (relatively prime to 8)}

    \item Find the lattice of $Z_{p^2q}$ 
\textcolor[gray]{0.5}{subgroups: $Z_p, Z_q, Z_{p^2}, Z_{pq}$}

    \item Number of divisors of 45?
    \textcolor[gray]{0.5}{
$$45 = 3^2 5$$
divisors = (2+1)(1+1) = 6\\
(add 1 to powers and multiply)}
    
    \item For $A, B$ subgroups of $G$, $A \cap B$ is a subgroup too.

    \item Show $\sigma^2$ is even for all permutations $\sigma$
    \textcolor[gray]{0.5}{look at the $\varphi: S_n \rightarrow \{\pm 1\}$;\\
    $\varphi(\sigma^2) = \varphi(\sigma)^2 = 1$, thus $\sigma^2$ is in the kernel of $\varphi$ hence in $A_n$.}

    \item Show $S_n$ is generated by $\{(1\ \ i): i \leq n\}$. \\
    \textcolor[gray]{0.5}{every permutation can be written as a transposition; use fact that $(ij) = (1\ i)(1\ j)(1\ i)$.}


\end{enumerate}

*to check homomorphism it sufficies to check generators and relations are preserved\\

$GL_n(F)$: set of all invertible (det$\neq 0$) with entries $\in F$, a field.\\


\section*{Quotient Groups}

\subsection*{Cosets}

For $H \leq G$, \\
cosets of $H$ ($gH$ for $g \in G$) \textbf{partition} $G$, each containing the same number of elements.\\

e.g., $\mathbb{Z}/\mathbb{3Z}$ partitions the integers into three cosets.\\

\fbox{\parbox{0.8\linewidth}{
How?\\
Notice, \\
1. If $a \in bH$, $aH = bH$. \\
\textcolor[gray]{0.5}{
For all $x \in aH$, $x = ah_1 = bh_2h_1$ \\
$\rightarrow x \in bH$.} \\

2. $aH, bH$ are disjoint or precisely the same. \\
\textcolor[gray]{0.5}{Suppose $x \in aH \cap bH$. Then, \\
$x = ah_1 = bh_2$ \\
$\rightarrow a \in bH$\\
Thus, $aH = bH$.}\\

3. Every element of $G$ is in some coset.\\
\textcolor[gray]{0.5}{trivially, $aH$, for any $a \in G$.}\\

4. $|H| = |aH|$ for any $a \in G$. \\
\textcolor[gray]{0.5}{consider function f(h) = ah, for any $a \in G$. \\
$f$ is bijective, meaning all cosets have the same number of elements.}

}}
\ \\

Lagrange: for $H \leq G$,
$$|G| = |H|\ [G:H]$$


The {\bf order} of any {\bf element} has to divide the order of the group.\\
\textcolor[gray]{0.5}{consider subgroup generated by the element, whose order then has to divide that of G.}\\

Any \textbf{group of prime} order is cyclic (any element generates entire group).\\

Any group of order $2p \cong Z_{2p}$ or $D_p$ \\
($Z_{2p}$ if it contains an element of order $2p$)


\subsection*{Normality}
Can cosets be a group?\\

For $H \leq G$, define operation: $aH bH= abH$ (for $a, b \in G$).\\
*operation is on cosets

Key question: when is $aH bH = abH$ well-defined?\\

\textcolor[gray]{0.5}{need to check whether two representatives from a coset, say $a$ and $a'$ (and $b$, $b'$), produce the same result under the operation.\\
Does $aH bH = a'H b'H$?\\
If $gHg^{-1}$, then yes! }\\

precisely when $H \trianglelefteq G$.\\

Thus, a subgroup $N$ is \textbf{normal} in $G$ if \\

\centerline{$gNg^{-1} \in N$ for all $g \in G$}
\centerline{equivalently, $gNg^{-1} = N$}
\ \\

A subgroup with \textbf{index 2} is normal
\textcolor[gray]{0.5}{only two cosets}
\textcolor{red}{reasoning...}\\

An \textbf{element of order 2} a, 
$$<a>\ \trianglelefteq G \iff a \in Z(G)$$
\textcolor[gray]{0.5}{$<a>\ \trianglelefteq G$, so \\
$xax^{-1}$ is $e$ or $a$ \\
can't be $e$ if $a \neq e$.}\\

If $G / Z(G)$ is cyclic, \textbf{$G$ is abelian}.
\textcolor[gray]{0.5}{section 3.1 problem 36}



\subsection*{Exercises}
    
\begin{enumerate}
    \item Show $H = <(1\ 2)>\ \leq S_3$, but not normal.\\
    \textcolor[gray]{0.5}{If $H \trianglelefteq S_3$, then 
$$N_{S_3}(H) = S_3.$$
But, 
$$(13)(12)(13)^{-1} = (23) \not \in H$$}

    \item Show $S_4$ has no normal subgroup of order 8 \\
\textcolor[gray]{0.5}{Suppose $H \trianglelefteq S_4$ with $|H| = 8$. \\
Then, $S_4 / H$ has order 3, meaning\\
$$(gH)^3 = H, \text{ for all $g$}$$
thus, $g^3 \in H$ for all $g \in G$.\\
all elements of order 2 raised to the third are themselves; thus, all elements of order 2 are in H, too many.}

    \item If $G / Z(G)$ is cyclic, show $G$ is abelian. 
    \textcolor[gray]{0.5}{ Let $xZ(G)$ be a generator of $G / Z(G)$ for some $x \in G$. \\
    Then for $a, b \in G$, \\
    $a = x^nc_a$ for some $c_a \in Z(G)$\\
    $b = x^m c_b$\\
    Since $c_a, c_b$ commute with any element, $ab = x^nc_ax^mc_b = ba$.}
    


\end{enumerate}


\subsection*{Cauchy's Theorem}
$p$ divides the order of $G$, then $G$ has an element of order $p$.
\textcolor[gray]{0.5}{proof later}


\subsection*{HK in G}
note $HK$ need not be a subgroup; when is it?\\
\begin{center}
$H, K \leq G$ and $K \trianglelefteq G$, then\\
$HK \leq G$\\
\textcolor[gray]{0.5}{
$e \in HK$. \\
For $a = h_1 k_1 \in HK$, a^{-1} = k_1^{-1} h_1^{-1} \in HK$}\\
\end{center}


a set $A$ \textbf{normalizes} $K$ if it's a subset of $N_G(K)$.\\

What's the order of $HK$? \\
$$|HK| = \frac{|H||K|}{|H \cap K|}$$


For $H$, $K$ $\leq G$,\\
\centerline{ \textbf{$HK$ is a subgroup} $\iff HK = KH$.}
\ \\

If $H, K \leq G$, finite, with relatively prime orders, 
$$H \cap K = 1$$
\indent \textcolor[gray]{0.5}{(problem 8 section 3.2) proof: look at orders of elements in H and K}


\subsection*{Isomorphism Theorems}
For $\varphi$ a homomorphism: $$\varphi : G \rightarrow H$$

\noindent \textbf{Fundamental Homomorphism} \\
$$G / Ker(\varphi) \cong \varphi(G)$$
"cosets of ker isomorphic to image"

\noindent \textbf{Diamond Iso}\\
$A, B \leq G$ and $A \trianglelefteq N_G(B)$\\

\centerline{$AB$}\\
\> \> \> \> \> \> \>\> \> \> \> \> \> \> \> \> \> \> \> \> \> \> \> \> \> \> \> \> \> \> \> \> \> \> \> \> \> \> \> \> \> \> \> \> \> \> \> \> \>  $A$ \ \ \ \ \ \ \ \ \ \ \ \ \ \ \ \ \ \ \ $\trianglelefteq B$\\
\centerline{$\trianglelefteq A \cap B$}\\

$$AB / B \cong A / (A \cap B)$$

\noindent \textbf{Lattice Iso}\\
$N \trianglelefteq G$ \\

The structure of the subgroups of G/N is exactly the same as the structure of the subgroups of G containing N, with N collapsed to the identity element.

"$G/N$ is all subgroups of $G$ above $N$ in lattice."

\subsection*{Composition Series}

For a group $G$, construct 
$$1 \leq N_1 \leq N_2 \leq ... \leq G$$

with $N_i \trianglelefteq N_{i + 1}$ and $N_{i+1} / N_{i}$ is \textbf{simple} \\

*simple: no non-trivial normal subgroups 

Then, \\
 \centerline{each $N_{i+1} / N_i$ "composition factor" is \textbf{unique}}
\ \\
 \indent as is the number of $N_i$\\

*factorization is not necessarly unique\\

\textbf{$G$ is solvable} if each $N_{i+1} / N_i$ is abelian.\\
\indent*$N_{i+1} / N_i$ need not be simple


\section*{Group Actions}


a group $G$ \textbf{acting} on a set $A$ is \\

\noindent Two conceptions:\\

\noindent \textbf{operation} such that\\
\indent a) $1a =a$ \\
\indent b) $g_1g_2a$ is assocaitive\\


\noindent a \textbf{homo map} $G \rightarrow S_{|A|}$ (from $G$ to the \textbf{symmetries} of $A$).\\

\noindent an action if \textbf{faithful} if its kernel is the identity\\

\subsection*{Conjugates of an element}

There is a 1-to-1 correspondence between\\
\begin{center}
conjugacy class of $a * \in G$ \\
and \\
cosets of $C_a(G)$ (centralizer of $a$ in $G$) \\
\end{center}
\textcolor[gray]{0.5}{Note $xax^{-1} = yay^{-1} \iff xC_a(G) = yC_a(G)$\\
since this implies $y^{-1}x \in C_a$\\
Consider $f: x C_a \rightarrow \text{ conjugate of a}$, defined\\
$f(x C_a) = xax^{-1}$\\
injective: by above. \\
surjective: for any $yay^{-1}$, there is $yC_a$ producing it.}\\

Thus, number of conjugates of $a$ equals the index of $C_a$ in $G$.\\

\centerline{\fbox{\text{ conjugates of a = [G:C_a]}}}
\ \\
*\textbf{conjugacy class} of $a$ means the set $xax^{-1}$ as $x$ ranges over $G$.\\

\subsection*{Class Equation}
\textcolor{red}{see chp 24  in Gali}
For $H \leq G$, 

there is a 1-to-1 correspondence between 
\begin{center}
conjugates of $H$ \\
and\\
cosets of N
\end{center}

\fbox{note, for $a \in N(H), aHa^{-1} = H}\\
\textcolor[gray]{0.5}{$aHa^{-1} \subseteq H$ \\
since for $h \in H$, $aha^{-1} \in H$.\\
$H \subseteq aHa^{-1}$ \\
since $aHa^{-1}$ contains as many elements as H\\
why? }


By previous result, \textbf{number of conjugacy classes}
\begin{center}
= size of orbits = index of stabilizer \\
= index of normalizer.

(for any element, the number of conjugates = index of its centralizer)
\end{center}

$$|G| = |Z(G)| + \text{ sum elements in each conjugacy class }

\subsection*{Orbits}
\textbf{Orbit} of $a \in A$ by is $\{g a : g \in G\}$.\\
"Hit $a$ with all $g \in G$", "spin $a$"\\

\noindent Orbits create \textbf{equivalence classes} in $A$\\

\centerline{$G_a =$ \textbf{stabilizer} of $a$ in $G$}
"elements of $G$ such that $ga = a$"\\

\centerline{\fbox{size of $Orb(a) = [G: G_a]$} = # different $g \{G_a\}$ }
\medskip
\centerline{= index of stabilizer}

Ker of action on set $aH$ "coset" = largest normal subgroup of $G$ contained in $H$

\textbf{Cayley's Theorem}\\
any finite group $G$ is isomorphic to a subgroup of $S_n$.\\
\textcolor[gray]{0.5}{Consider function, $\pi_a(x)$ for $a \in G$ by ax. This function is bijective, hence permutes $G$. Therefore, we consider composition of functions $\pi$ to see the permutations form a subgroup of $S_n$.}\\


For \textbf{$p$ the smallest} prime dividing $|G|$,\\ 
\centerline{any subgroup of index $p \trianglelefteq G$.}
\ \\

\subsection*{Fundamental Theorem of finitely generated abelian groups}
"FTFGAG"

\centerline{every FGAG is the direct product of cyclic groups}
\bigskip
\centerline{\indent aka, $\displaystyle G \cong \underbrace{\mathbb{Z} \times \mathbb{Z} \times \mathbb{Z}}_r...\underbrace{\frac{\mathbb{Z}}{n\mathbb{Z}} \times ...\times \frac{\mathbb{Z}}{n\mathbb{Z}}}_{invariant}$}\\
\medskip
\centerline{invariants: $n_1 | n_2...| n_k$}
\bigskip

r is called \textbf{rank}\\
\indent \textbf{invariant factors} are unique\\

recall,\\ 
any cyclic group is isomorphic to:\\
\begin{itemize}
\item $\mathbb{Z}$ is $(\mathbb{Z}, +)$ which is infinite\\
 \centerline{or}
\item $\displaystyle \frac{\mathbb{Z}}{n\mathbb{Z}}$ over $+$ is finite\\
\end{itemize}

\noindent So, $r=0$ if $G$ is finite.
\bigskip

\noindent Two FGA groups are \textbf{isomorphic} $\iff$ \textbf{same rank and invariant factors}\\

e.g., $|G| = 8$\\
possible expression as cyclic groups: \\
$\frac{\mathbb{Z}}{8\mathbb{Z}}$\\

\noindent$\frac{\mathbb{Z}}{2\mathbb{Z}} \times \frac{\mathbb{Z}}{4\mathbb{Z}}$\\
\noindent$\frac{\mathbb{Z}}{2\mathbb{Z}} \times \frac{\mathbb{Z}}{2\mathbb{Z}} \times \frac{\mathbb{Z}}{2\mathbb{Z}}$$\\

\centerline{\textbf{Chinese Remainder Theorem}}
$$\frac{\mathbb{Z}}{mn\mathbb{Z}} \cong \frac{\mathbb{Z}}{m\mathbb{Z}} \times \frac{\mathbb{Z}}{n\mathbb{Z}}$$
$$\iff$$
$$(m,n) = 1$$

Traditionally written as $x \equiv a \mod n$ and $x \equiv b \mod m$ implies there is only one solution to $x \mod mn$.
\textcolor[gray]{0.5}{direct congurence proof}\\

Hence, any group $G$ can also be written in elementary divisor form: 

$$G \cong \mathbb{Z}^r \times \prod \frac{\mathbb{Z}}{p_i^a \mathbb{Z}}$$


*elementary divisors are not invariant factors!

* $Z_2 \times Z_2 \neq Z_4$ (different invariant factors)

\subsection*{Sylow's Theorem}
$G$ has order $p^\alpha m$, with $p$ not dividing $m$, then $G$ has a subgroup of order $p^\alpha$.

Further,
\begin{itemize}
    \item any 2 sylow p-groups are conjugate
    \item $n_p$ the number is Sylow p-groups: $n_p \equiv 1$ mod p and $n_p | m$
    \item a unique Sylow p-group is \textbf{normal}
\end{itemize}


For $|G = 5*7|$, often \textbf{useful} to consider \\
\indent \hspace{20mm} \textbf{quotients}: $G / P_7$ ( G mod a Sylow 7-subgroup) \\
\indent \hspace{20mm} \textbf{subgroup}: $H = P_7 P_5$


\subsection*{Exercises}

\begin{enumerate}
    \item Show $G$ with $|G| = pq$ is cyclic \\
    \textcolor[gray]{0.5}{
    number of Sylow p groups: $n_p \equiv 1$ mod p and $n_p | q$\\
    $\rightarrow$ $n_p = 1$\\
    Similarly, $n_q \equiv 1$.
    Thus, the unique p and q subgroups are abelian, as they're of prime order.\\
    Further, both are normal $\rightarrow$ commutes.}


    \item Determine groups of order 99 \\
    \textcolor[gray]{0.5}{unique Sylow 11-subgroup and Sylow 3-subgroup. Thus, can show group is abelian, hence is $Z_{99}$ or $Z_3 \times Z_{33}$}

    \item Find the invariant factors of all abelian groups of order $270 (=2*3^3*5)$\\

    \textcolor[gray]{0.5}{First find elementary divisors:}
    $$ Z_2 \times Z_{3^3} \times Z_5$$
    $$ Z_2 \times Z_3 \times Z_{3^2} \times Z_5$$
    $$ Z_2 \times Z_3 \times Z_3 \times Z_3 \times Z_3 \times Z_5$$

    \textcolor[gray]{0.5}{Then, for each abelian group, write powers in descending order: }
        \begin{center}
            \begin{tabular}{ l | c | r }
            p=3 & p=2 & p=5 \\
            \hline
            3^3 & 2 & 5 \\
            1 & 1 & 1 \\
            1 & 1 & 1
            \end{tabular}
        \end{center}
    \textcolor[gray]{0.5}{Each row yields an invariat factor. Here it's: $Z_{270}$ \\}

    For $$ Z_2 \times Z_3 \times Z_{3^2} \times Z_5$$, 
        \begin{center}
            \begin{tabular}{ l | c | r }
            p=3 & p=2 & p=5 \\
            \hline
            3^2 & 2 & 5 \\
            3 & 1 & 1 \\
            1 & 1 & 1
            \end{tabular}
        \end{center}
    the invariant factors are $Z_{90} \times Z_3$.}

    \item G with $|G| = 105$ has a normal Sylow 5-subgroup\\
    \textcolor[gray]{0.5}{105 = 3*5*7 so \\
    $n_5 \equiv 1$ mod 5 and $n_5 | 3*7$\\
    meaning $n_5 = 21$ or $1$...
    }
\end{enumerate}

\centerline{\LARGE \textbf{Rings}}

an \textbf{abelian group} with multiplication such that
\begin{itemize}
    \item * is associative and closed
    \item Distribution
\end{itemize}

a ring with multiplicative inverses (for non-zero elements) is called a \\
\centerline{ "Division Ring" (or Skew field)}
\ \\
*field is a commutative division ring\\

$u$ is a \textbf{unit} of $R$ if there exists $v$ such that $uv = vu = 1$.\\
\indent $u$ is a \textbf{zero divisor"} if there exists $v$ such that $uv=0$ or $vu =0$. \\

In $\mathbb{Z} / n \mathbb{Z}$ an element is a unit (if relatively prime to n) or a zero divisor. \\

$R^*$ is the set of all units of $R$.\\

an \textbf{Integral Domain} Ring is a ring with 
\begin{itemize}
    \item unit
    \item commutative
    \item no zero divisors
\end{itemize}
e.g., $\mathbb{Q}$, $\mathbb{Z} / n \mathbb{Z}$ if n is prime \\



\subsection*{Quotient Rings}

a ring \textbf{homomorphism} $\varphi$ preserves + and * \\

Ker($\varphi$) = elements mapping to $0$

\subsubsection*{Ideals}
analogous to normal subgroups.
a subgring $I$ is an \textbf{ideal} of $R$ if 
$$ir \in I \text{ and } ri \in I \text{\  \ \  for all r } \in R \text{ and } i \in I$$

Thus, we defined a \textbf{quotient ring} as sets $r + I$, for $r$ in ring with operations:
$$(r_1 +I)+(r_2+I) = (r_1+r_2) + I \text{ and } (r_1 + I) * (r_2 + I) = (r_1r_2)+ I$$

any ideal $I$ is the \textbf{Ker} of homo $\varphi(r) = r I$. \\

Ker of $\varphi$ is always an ideal.\\

\hspace{10mm} prototypical example of ideal: $\mathbb{Z}$ with ideals $n\mathbb{Z}$\\

For $I, J$ ideals of $R$, 
\begin{itemize}
    \item $I \cap J$ is an ideal
    \item $IJ$ is defined as $\{$ finite sums $ij \}$
\end{itemize}

only ideals of a field are trivial \\
\indent \textcolor[gray]{0.5}{for $a \in I$, there is $a^{-1} \in F$, so that $1 \in I$.}\\

\textbf{Lattice} Isomorphism also preserves ideals between ring and quotient.

\subsection*{Special Ideals}

an ideal is \textbf{principle} if it's generated by \textbf{one element}.\\
\indent \hspace{70mm} (using both + and *)\\

For $R$ a commutative ring with unit, \\

an ideal $M$ is \textbf{maximal} in $R$ if \\
\centerline{no other proper ideal contains $M$}\\

\centerline{\fbox{$M$ is maximal $\iff$ $R / M$ is a field}}\\
\textcolor[gray]{0.5}{$R / M$ is field means no ideals; by lattice iso, no ideals between R and M}\\

(again assume commutative ring with unit)\\
a proper ideal is \textbf{prime} if $ab \in P$, then $a$ or $b \in P$.\\

\centerline{\fbox{P is a prime ideal $\iff R / P$ is an integral domain}}\\

\textcolor[gray]{0.5}{apparently follows from def, but unclear} \\

Thus, max ideal $\rightarrow$ prime ideal (filed is an integral domain)\\

*monic polynomial means leading coefficient is 1

\subsection*{Finite Fields}

For $p(x)$ \textbf{irreducible} in $F[x]$, \\
\ \\
\centerline{\fbox{$F[x]\ /\ (p(x))$ is a field}}

quadratics and cubics are reducible only if reduction contains linear factor. \\

only irreducible quartics: \\
$$x^4 + x + 1$$
$$x^4 + x^3 + 1$$
$$x^4 + x^3 + x^2 + x + 1$$

\textbf{characteristic} of a field $F$ is the smallest integer $n$ such that 
$$1^n = 1 + ... + 1 = 0$$

the characteristic of a field is either 0 or p.\\

All finite fields have order $p^n$ for some prime p.



\subsection*{Exercises}
\begin{enumerate}
    \item Given an exmaple of a division ring that's not a field\\
    \textcolor[gray]{0.5}{Quaternions $\mathbb{H}$, since $ij \neq ji$, hence not abelian;\\ is a division ring since * inverses exist (complicated looking fraction)}

    \item What are the ideals of $\mathbb{Z}$?\\
    \textcolor[gray]{0.5}{$n\mathbb{Z}$ for $n \in \mathbb{Z}$: $0, \mathbb{Z}, 2\mathbb{Z}$ }

    \item What are the max ideals of $\mathbb{Z}$?\\
    \textcolor[gray]{0.5}{n$\mathbb{Z}$ is maximal when $\mathbb{Z} / n \mathbb{Z}$ is a field, meaning n prime.}

    \item What the prime ideals of $\mathbb{Z}$?
    \textcolor[gray]{0.5}{all of the above AND 0}

\end{enumerate}


\subsection*{Curiosities}

Compose bijective functions bijection\\
This explains why composing cycles produces a cycle, aka a perumtation.
\textcolor[gray]{0.5}{prove directly by thinking about steps of composition.}
\section*{Open Questions}

\subsection*{Open Questions}
\centerline{\fbox{P is a prime ideal $\iff R / P$ is an integral domain}}\\
\textcolor{red}{why?}

\end{document}
