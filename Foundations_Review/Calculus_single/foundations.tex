
\documentclass[12pt]{article}   	
\usepackage{graphicx}				
\usepackage{amssymb, amsmath, mathtools}

\title{Foundations for Calculus}
\date{}	

\begin{document}
\maketitle

\section{Properties of Numbers}
\subsection{Division by Zero}

$0 * b = 0$ for all numbers b. \\
Also, for all numbers $a \neq 0$, $\exists \ a^{-1}$ such that, $ a * a^{-1} = 1$.\\

\noindent Therefore, since $0 * b = 0$, we exclude $0^{-1}$ from the multiplicative inverse above, 
meaning division by zero can not exist. \\

note: $(x - 1) (x - 2) = 0$, implies $x = 1$ or $x =2$ or BOTH.


\subsection{Negative and Positive Numbers}
0 is neither positive nor negative,\\ 
since positive is defined as all numbers greater than zero (negative less than). 

\subsection{Absolute Value}
For any number a, \\
\[
|a| =  
    \begin{dcases*}
        a, & if $a \geq 0$ \\ 
        -a, & if $a < 0$
    \end{dcases*}
\]

When solving equations, straightforward appraoch is to use definition.\\

Geometric Interpretation of $|a|$ is the distance from the origin of a. \\

\noindent To solve $|a| = 4$, we can use the geometric interpretation: numbers that have a distance of 4 from the origin, a = 4 or a = -4. \\

\noindent More generally, for 
\begin{itemize}
    \item $b > 0,  \text{if } |a| = b\text{, then } a = b \text{ or } a = -b.$ 
    \item $b = 0, a = 0$
    \item $b < 0, \text{ then no a exists, since distance can't be negative!} $
\end{itemize}

LOOK MORE INTO ABSOLUTE VALUES.


\subsection{Triangle Inequality}

\section*{Limits}
\centerline{\textbf{$f$ approaches the limit near a}}
\bigskip

\centerline{for every $\epsilon > 0$, there exists $\delta >0$ such that for all $x$\\}
\medskip
\centerline{if $0 < |x -a| < \delta$, then $|f(x) - l| < \epsilon$}

note $x \neq a$ !

Observations: 
\begin{itemize}
    \item a function can't approach two different limits
    \textcolor[gray]{see proof with picture}
    \item $\lim_{x \rightarrow a} f(x) + g(x) =$ limit of each sumed (same for products)
\end{itemize}

\textcolor{red}{recall polynomial factor and remainder theorems (long division). see
mathisfun.com polynomials remainder factor}

\end{document}  
