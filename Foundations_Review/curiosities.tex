

\documentclass[a4paper, 12pt]{article} 
\usepackage{amsmath, amssymb, color, graphicx, enumitem}
\usepackage{fullpage} %smaller margins

%font
\usepackage[sc]{mathpazo}
\linespread{1.05}         % Palladio needs more leading (space between lines)
\usepackage[T1]{fontenc}

%word spacing
\usepackage{microtype}


%useful shortcuts
\def\R{\ensuremath{\mathbb{R}}} %\ensuremath adds math mode, if forgotten
\def\Q{\ensuremath{\mathbb{Q}}}
\def\N{\ensuremath{\mathbb{N}}}
\def\Z{\ensuremath{\mathbb{Z}}}
\def\C{\ensuremath{\mathbb{C}}}

%shorcuts with arguments
\newcommand{\abs}[1]{\left\vert#1\right\vert} %nice absolute values
\newcommand{\bt}[1]{\textbf{#1}} %bold
\newcommand{\eq}[1]{\begin{align*}#1\end{align*}} %aligned equations
\newcommand{\cb}[1]{\centerline{\fbox{#1}}} %centered box
\newcommand{\bp}[1]{\fbox{\parbox{0.8\textwidth}{#1}}} %box paragraph
\newcommand{\notimplies}{% does not imply
  \mathrel{{\ooalign{\hidewidth$\not\phantom{=}$\hidewidth\cr$\implies$}}}}
\renewcommand{\eq}[1]{\begin{align*}#1\end{align*}} %aligned equations


%colors
\definecolor{javagreen}{rgb}{0.25,0.5,0.35} %dark green color
\newcommand{\green}[1]{\textcolor{javagreen}{#1}} %command for green
\newcommand{\gray}[1]{\textcolor[gray]{0.5}{#1}} %gray text

%environment


\title{}
\date{}
%==tips====
%part
    %section, sub, sub
%\begin{enumerate}[resume] %continues counting
\begin{document}


\section*{Curiosities}

\subsection*{Fibonacci Sequence}
the Fibonacci sequence is $F_1 = 1, F_2 = 1, F_3 = 1+1, F_4 = 2 + 1 \dots$

has a closed form expression:
$$F_n = \frac{\theta - (-\theta)^{-n}}{\sqrt{5}}
\text{, for $\theta = \frac{1 + \sqrt{5}}{2}$.}$$

where $\theta$ is the \bt{golden ratio}. 

Also, 
\eq{
    \lim_{n \rightarrow \infty} \frac{F_{n+1}}{F_n} = \theta.
}

\subsection*{Binomial Theorem}
$$(a+b)^n = \sum_{k=0}^n \binom{n}{k} a^{n-k}b^k$$

$$\binom{n}{k} = \frac{n!}{k!(n-k)!}$$

\subsection*{Polynomials}

$$1 + r + r^2 + ... + r^n = \frac{1 - r^{n+1}}{1-r}$$
\textcolor[gray]{0.5}{set $S = 1 + r + ... + r^n$, multiply by r, and solve}


\subsection*{Cauchy Schwarz}

$$\sum x_i y_i \leq \sqrt{\sum x_i^2} \sqrt{\sum y_i^2}$$
\textcolor{red}{look into proof}


\subsection*{Complex: Birth of Sin, Cos}
Define, for $z \in \mathbb{C}$: 
$\displaystyle e^z = \sum_{n=0}^\infty \frac{z^n}{n!}$\\
$\displaystyle \cos(z) = 1 - \frac{z^2}{2!} + \frac{z^4}{4!} - \frac{z^6}{6!} + ...$\\
$\displaystyle \sin(z) = \frac{z}{1!} - \frac{z^3}{3!} + \frac{z^5}{5!} - \frac{z^7}{7!} ...$\\
\ \\

We derive Euler's formula, 
$$e^{ix} = \cos(x) + i\sin(x)$$

\textcolor[gray]{0.5}{
Set $x$ above to $2a$, then 
\begin{align*}
    e^{i2a} & = (e^{ia})^2\\
    & = (\cos(a) + i\sin(a))^2 \\
    & = \cos^2(a) + 2 i \cos(a) \sin(a) - \sin^2(a)\\
\end{align*}
$e^{i2a}$ also equals $\cos(2a) + i \sin(2a)$.
So real parts/imaginary parts are equal, 
$$\cos(2a) = \cos^2(a) + \sin^2(a)$$
$$\sin(2a) = 2 \cos(a) \sin(a)$$
}
Can prove $\sin(a + b)$ identity similarly.

\subsection*{Complex Numbers}
for $a, b \in \mathbb{C}$, 
$$a*b = \text{number with length a*b, angle a + b}$$


\subsection*{Trig}

\textbf{Master Identities}: \\
$\sin(x + y) = \sin(x)\cos(y) + \sin(y)\cos(x)$ \\
$\cos(x + y) = \cos(x)\cos(y) - \sin(x)\sin(y)$\\

*remember $\sin$ is odd $(\sin(-x) = -\sin(x))$; $\cos$ is even.\\

from these derive the rest.\\

\subsubsection*{Exercsies}
\begin{enumerate}
    \item $\tan^2(x) \sin(x) = \tan^2(x)$\\
\textcolor[gray]{0.5}{$\tan^2(x) (\sin(x) - 1) = 0$}\\
$\tan^2(x) = 0$ or $\sin(x) = 1$

    \item $2\cos^2(x) + \sin(x) -2 = 0$\\
\textcolor[gray]{0.5}{use $\cos^2(x) = 1 - \sin^2(x)$}.\\
\end{enumerate}

\subsection*{Log}
$\log(ab) = \log(a) + \log(b)$\\
$\log(a^m) = m\log(a)$\\
\textcolor[gray]{0.5}{
why?
\begin{flalign*}
   \log(a^m) &= \log(a...a),& \\
     & = \log(a) + \log(a) + ... + \log(a),& \\
     & = m\log(a).&
\end{flalign*}
}

\subsection*{Derivatives}
$\displaystyle \frac{d}{dx} \arcsin(x) = \frac{1}{\sqrt{1+x^2}}$\\
$\displaystyle \frac{d}{dx} \arctan(x) = \frac{1}{1+x^2}$\\
\textcolor{red}{proofs?}\\

\noindent $\displaystyle \frac{d}{dx} a^x = a^x ln(a)$\\
\ \\
$\displaystyle \int \frac{1}{x} = ln(|x|) + c$ \\

\subsection*{Integration Tricks}

$$\int_{- \infty}^\infty e^{-x^2} dx$$
is nontrivial.\\
Idea is to consider $\int_{- \infty}^\infty\int_{- \infty}^\infty e^{-x^2}e^{-y^2} dx dy$ and use polar coordinates to integrate.\\
(see machine leanring hw1)\\

For \textbf{Integrating a Polynomial over a sphere} see trick in paper by Folland.

\subsection*{Min and Max in $\R$}

$$max(a, b) = \frac{a+b}{2} + \frac{|a-b|}{2}$$\\
\gray{(a+b)/2 takes you to midpoint. |a-b|/2 adds the remaining half to the larger number}\\

$$min(a, b) = \frac{a+b}{2} - \frac{|a-b|}{2}$$ \\
\gray{similar idea, except goes down by the half}

\section*{Geometry}
\subsection*{area of an equilaterla triangle}
$A = \sqrt{3}/4 s^2$
\gray{derive by bisecting angle, using 30, 60, 90 triangle}

\subsection*{Number of diagonals in an n-sided figures}

\eq{\frac{n(n-3)}{2}}
\end{document}
